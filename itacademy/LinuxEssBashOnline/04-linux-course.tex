% xetex compatible variant that support TTF fonts according to company rules
\documentclass[ignorenonframetext, professionalfonts, hyperref={unicode}]{beamer}

\usetheme{Epam}

\usepackage{fontspec}
\setsansfont{SourceSansPro-Regular}
%\setbeamerfont{frametitle}{family=\fontspec{Oswald}}
\setbeamerfont{frametitle}{family=\fontspec{Oswald}}
\setbeamerfont{block title}{family=\fontspec{Oswald}}

%\setmainfont{Times New Roman}
\defaultfontfeatures{Mapping=tex-text}
\defaultfontfeatures{Ligatures=TeX}

%\setsansfont{Arial}
%\setromanfont{Trebuchet MS}

\usepackage{cmap}
\usepackage{graphicx}

\usepackage{textcomp}

\usepackage{beamerthemesplit}

\usepackage{ulem}

\usepackage{verbatim}
\usepackage{import}

\usepackage{listings}
\lstloadlanguages{bash}

\lstset{escapechar=`,
	captionpos=b,
	extendedchars=false,
	language=sh,
%	frame=single,
	tabsize=2, 
	columns=fullflexible, 
%	basicstyle=\scriptsize,
	keywordstyle=\color{blue}, 
	commentstyle=\itshape\color{brown},
%	identifierstyle=\ttfamily, 
	stringstyle=\mdseries\color{green}, 
	showstringspaces=false, 
	numbers=left, 
	numberstyle=\footnotesize, 
	breaklines=true, 
	inputencoding=utf8,
	keepspaces=true,
	morekeywords={u\_short, u\_char, u\_long, in\_addr}
	}

\definecolor{darkgreen}{cmyk}{0.7, 0, 1, 0.5}

\lstdefinelanguage{diff}
{
    morekeywords={+, -},
    sensitive=false,
    morecomment=[l]{//},
    morecomment=[s]{/*}{*/},
    morecomment=[l][\color{darkgreen}]{+},
    morecomment=[l][\color{red}]{-},
    morestring=[b]",
}

\author[Epam]{{\bf Epam}\\Low Level Programming Department}

%\institution[EPAM]{EPAM}
%\logo{\includegraphics[width=1cm]{logo.png}}

\graphicspath{{../../slides/cmdline/clipart/}{../../slides/bash/clipart/}}

\bibliographystyle{unsrt}
\setbeamertemplate{bibliography item}{\insertbiblabel}

\AtBeginSection[]{%
  \begin{frame}<beamer>
    \frametitle{}
    \tableofcontents[
        sectionstyle=show/shaded, hideallsubsections ]
  \end{frame}
  \addtocounter{framenumber}{-1}% If you don't want them to affect the slide number
}

% \regex for regular expressions
\newcommand{\regex}[1]{ %
\expandafter{$\ulcorner{\color{blue}\texttt{#1}}\lrcorner$} %
}



\title{Введение в GNU/Linux}


%%%%%%%%%%%%%%%%%%%%%%%%%%%%%%%%%%%%%%%%%%%%%%%%%
%%%%%%%%%% Begin Document  %%%%%%%%%%%%%%%%%%%%%%
%%%%%%%%%%%%%%%%%%%%%%%%%%%%%%%%%%%%%%%%%%%%%%%%%

\begin{document}

\begin{frame}
	\frametitle{}
	\titlepage
	\vspace{-0.5cm}
	\begin{center}
	%\frontpagelogo
	\end{center}
\end{frame}


%%%%%%%%%%%%%%%%%%%%%%%%%%%%%%%%%%%%%%%%%   
%%%%%%%%%% Content starts here %%%%%%%%%%
%%%%%%%%%%%%%%%%%%%%%%%%%%%%%%%%%%%%%%%%%

\section{Процесс загрузки ОС Linux}
\mode<all>{\begin{frame}{Процесс загрузки GNU/Linux}
	\scriptsize
	\begin{enumerate}
		\item BIOS
		\item Master Boot Record (MBR)
			\pause
		\item Загрузка загрузчика 
		\begin{itemize}
		\footnotesize
			\item Stage 1 -- Первичный загрузчик
			\item Stage 1,5 -- Загрузка ядра загрузчика и драйвера ФС
			\item Stage 2 -- загрузчик читает конфигурацию, загружает ядра и образ initrd (initial-RAM disk) в память
                        \item Передает управление ядру
		\end{itemize}

		\item Запуск программы инициализации в initrd, загрузка драйверов файловых систем (LVM, RAID, NFS)
			\pause
		\item Нахождение и монтирование корневого раздела
			\pause
		\item Запуск программы init
		\begin{itemize}
		\footnotesize
			\item Монтирование оставшихся разделов ФС
			\item Запуск демонов для заданного уровня загрузки (runlevel)
			\item Выдает приглашение пользователю. 
		\end{itemize}

	\end{enumerate}
\end{frame}
}
\subsection{Остановка системы}
\mode<all>{\begin{frame}[fragile]
\frametitle{How to correctly stop system.}
    \begin{itemize}       
        \item reboot
        \item shutdown now
        \item shutdown –h now
        \item shutdown –h 20:00
        \item shutdown -c
        \item shutdown -r now
        \item shutdown +15 "Upgrading hardware, downtime should be minimal“
    \end{itemize}
\end{frame}
}

\section{Package management system}
\mode<all>{\begin{frame}{Software installation}
 How to install software to computer? Please describe process step by step. 
\pause
    \begin{enumerate}
        \item Find application 
        \pause
        \item Download installation package or source code
        \pause
        \item Run installer on complie
    \end{enumerate}
\end{frame}

%\begin{frame}{.rpm package installation}
%    Установим программу из .rpm пакета
%    \begin{enumerate}
%        \item Найдем пакет vim-enhanced и загрузим его на хост 
%        \pause
%        \item Посмотрим информацию о пакете
%        \pause
%        \item Установим его
%        \pause
%    \end{enumerate}
%    Ошибка о зависимостях, т.к. не хватает установленных программ/библиотек в системе.
%\end{frame}
}
\mode<all>{\begin{frame}{Задачи системы управления пакетами.}
\begin{itemize}
 \item избежать Dependency hell 
 \item Общие задачи пакетного менеджера:
   \begin{itemize}
     \item Проверка целостности пакетов
     \item Проверка зависимостей пакетов
        \item Поддержание списка установленных пакетов
        \item Автоматическое удаление пакетов
     \item Предоставление доступа к репозиторию пакетов
     \item Разрешение зависимостей
   \end{itemize}
\end{itemize}
\end{frame}
}
\mode<all>{\newcounter{tmpc}

\begin{frame}{Репозиторий}
	\begin{block}{Репозиторий пакетов}
		Место, где хранятся и поддерживаются пакеты, а также сопутствующая мета-информация, предназначенное для использования пакетным менеджером.
	\end{block}
	\begin{block}{Пример: Fedora Core}
		\begin{itemize}
			\item Packages/*.rpm
			\item RPM-GPG-KEY-*
			\item repodata
			\begin{itemize}
				\item множество сжатых и несжатых XML файлов для YUM
			\end{itemize}
		\end{itemize}

		Описание репозтория для YUM на локальной системе хранится по пути
		{\tt /etc/yum.repos.d/*.repo}
	\end{block}
		
\end{frame}

\begin{frame}{Apt: команды}
	\begin{block}{Установка/обновление пакета}
		{\tt apt-get install pkgname }

                {\tt apt-get -f install}
	\end{block}
	\begin{block}{Обновление данных о пакетах}
		{\tt apt-get update }
	\end{block}
	\begin{block}{Удаление пакета}
		{\tt apt-get remove pkgname }
	\end{block}
	\begin{block}{Поиск}
		{\tt apt-cache search pkgname }
	\end{block}
\end{frame}

\begin{frame}{YUM: команды}
	\begin{block}{Установка/обновление пакета}
		{\tt yum install pkgname }
	\end{block}
	\begin{block}{Обновление всех пакетов}
		{\tt yum update }
	\end{block}
	\begin{block}{Удаление пакета}
		{\tt yum remove pkgname }
	\end{block}
	\begin{block}{Поиск}
		{\tt yum list pkgname }\\
		{\tt yum search pkgname }
	\end{block}
\end{frame}


\begin{frame}[fragile]{Упражнение}
%  \begin{enumerate}
%      \item Создать на {\tt /dev/sda} раздел размером примерно 10Gb
%      \item Создать на этом разделе ext3 ФС и смонтировать раздел в {\tt /mnt/chroot}
%      \item Развернуть {\tt /media/nfs/pub/CentOS/precreated/centOS.tar.gz} в {\tt /mnt/chroot}
%      \item Смонтировать {\tt proc, sysfs} а также {\tt /dev} в соответствующие места {\tt /mnt/chroot}
%      \item {\tt chroot /mnt/chroot}
%      \item Отредактировать {\tt /etc/resolv.conf} -- скопировать туда информацию из {\tt resolv.conf} основной системы
%      \item Отредактировать {\tt /etc/yum.conf} Добавить следующий раздел
%\begin{minipage}{0.5\textwidth}
%\begin{verbatim}
%[base]
%  name = CentOS 6
%  baseurl = ftp://192.168.11.15/CentOS
%  gpgcheck = 0
%\end{verbatim}
%\end{minipage}
%\setcounter{tmpc}{\theenumi}
%\end{enumerate}
%\end{frame}
%\begin{frame}{Продолжение упражнения}
  \begin{enumerate}
      %\setcounter{enumi}{\thetmpc}
      \item Обновить информацию о пакетах {\tt update}
      \item Удалить пакет vim
      \item Установить заново пакет vim
      \item Посмотреть списки файлов для пакетов {\tt rpm, vim}
      \item Найти, к какому пакету относится команда {\tt ls, top}
      \item Найти пакет предоставляющий сервис ssh и установить его
    \end{enumerate}
\end{frame}
}
\mode<all>{\begin{frame}{RPM: структура пакета}
	\begin{itemize}
		\item Метаданные
			\begin{itemize}
				\item Имя
				\item Версия/Релиз
				\item Группа
				\item Описание
                                \item Зависимости
				\item ...
			\end{itemize}
		\item Архив с файлами
			\begin{itemize}
				\item cpio
			\end{itemize}
		\item Скрипты
			\begin{itemize}
				\item Pre Install
				\item Post Install
				\item Pre Uninstall
				\item Post Uninstall \bigskip
				\item Triggers
			\end{itemize}
	\end{itemize}
\end{frame}

\begin{frame}{Система управления пакетами: для чего это нужно}
\begin{itemize}
 \item ''DLL Hell''
 \item Dependency hell
 \item Общие задачи пакетного менеджера:
   \begin{itemize}
     \item Проверка целостности пакетов
     \item Проверка зависимостей пакетов
        \item Поддержание списка установленных пакетов
        \item Автоматическое удаление пакетов
     \item Предоставление доступа к репозиторию пакетов
     \item Разрешение зависимостей
   \end{itemize}
\end{itemize}
\end{frame}

\begin{frame}{Debian-based и RedHat-based системы управления пакетами}

\begin{center}
 \textbf{Два уровня пакетных менеджеров}
\end{center}

\begin{tabular}{| l | c | r |}
      \hline
          Level &  RedHat-based & Debian-based \\ 
      \hline
          Low & rpm & dpkg \\ 
      \hline
          High & yum, dnf & apt, aptitude \\
      \hline
    \end{tabular}

    Низкоуровневые используются для установки, удаления, получения информации о пакете. \\
    Высокоуровневые предоставляют дополнительные функции такие как поиск по репозиторию, копирование пакета из репозитория, разрешение зависимостей, обновление системы.

\end{frame}

}
\mode<all>{\begin{frame}{Команды пакетных менеджеров}
        \begin{tabular}{ll}
            \multicolumn{2}{c}{Установка пакета }   \tabularnewline
            Debian & {\tt apt-get \alert{install} pkgname } \\
            CentOS & {\tt yum \alert{install} pkgname } \\
            \multicolumn{2}{c}{Обновление пакета }  \tabularnewline
            Debian & {\tt apt-get \alert{install} pkgname } \\
            CentOS & {\tt yum \alert{update} pkgname }  \\
            \multicolumn{2}{c}{Удаление пакета }   \tabularnewline
            Debian & {\tt apt-get \alert{remove} pkgname } \\ 
            CentOS & {\tt yum \alert{remove} pkgname }  \\
            \multicolumn{2}{c}{Поиск. По имени пакета}   \tabularnewline
            Debian & {\tt apt-cache \alert{search} pkgname } \\
            CentOS & {\tt yum \alert{list} pkgname }  \\
            \multicolumn{2}{c}{Поиск. По строке.}   \tabularnewline
            Debian & {\tt aptitude \alert{search} '\alert{\textasciitilde d}tmux' } \\
            CentOS & {\tt yum \alert{whatprovides} tmux} 
        \end{tabular}
\end{frame}
}
% when to use low level package managers
%\mode<all>{\begin{frame}{RPM: команды}
	\begin{block}{Установка пакета}
		{\tt rpm -i [rpm-file1] ... [[url://]rpm-fileN] }
	\end{block}
	\begin{block}{Удаление пакета}
		{\tt rpm -e pkgname1 ... pkgnameN }
	\end{block}
	\begin{block}{Обновление пакета}
		{\tt rpm -U [rpm-file1] ... [[url://]rpm-fileN] }
	\end{block}
	\begin{block}{Проверка пакета}
		{\tt rpm -V pkgname1 ... pkgnameN }
	\end{block}
\end{frame}
}
%\mode<all>{\begin{frame}{RPM -q: часто используемые опции опроса}

	\begin{itemize}
		\item {\tt pkgname} -- выбор пакета, установленного в системе
		\item {\tt -a} -- все пакеты, установленные в системе
		\item {\tt -p} -- использовать файл RPM
	\end{itemize}


	\begin{itemize}
		\item {\tt -i} -- показать информацию пакета\\
			{\tt rpm \alert{-q} \alert{-i} glibc }
		\item {\tt -l} -- показать список файлов пакета \\
			{\tt rpm \alert{-q -l} glibc }
		\item {\tt -{}-whatprovides} -- \\
			{\tt rpm \alert{-q --whatprovides} java}
		\item {\tt -{}-whatrequires} -- \\
			{\tt rpm \alert{-q --whatrequires} /bin/bash}
		\item {\tt -{}-queryformat} -- формат вывода\\
			{\tt rpm \alert{-q -{}-whatrequires} /bin/bash \alert{-{}-queryformat ''\%\{name\} ''} }

	\end{itemize}

\end{frame}
}
%\mode<all>{\begin{frame}{ Упражнение .rpm package installation}
    Установим программу из .rpm пакета
    \begin{enumerate}
        \item Найдем пакет vim-enhanced и загрузим его на хост 
        \pause
        \item Посмотрим информацию о пакете
        \pause
        \item Установим его
        \pause
    \end{enumerate}
    Ошибка о зависимостях, т.к. не хватает установленных программ/библиотек в системе.
\end{frame}
}


\end{document}
