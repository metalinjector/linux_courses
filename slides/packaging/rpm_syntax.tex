\begin{frame}
%damn tilde, best that i found
\def~{$\sim$}
	\frametitle{Подготовка окружения для сборки}	
			\begin{itemize}
				\item {\tt rpmdevtools} -- разные полезные помощники
				\item {\tt rpmdev-setuptree} -- создает структуру каталогов
				\item {\tt ~/.rpmmacros} -- информация о пакетировщике, локальные
			\end{itemize}			
				\begin{table}
					\begin{tabular}{l | l | l }
					Каталог & Макрос & Описание\\
					\hline
					~/RPM/SPECS/ & \%\_specdir & Сборочные .spec файлы\\
					~/RPM/SOURCES/ & \%\_sourcedir & Исходные архивы и патчи\\
					\hline
					~/RPM/BUILD/ & \%\_builddir	& Сборка\\		
					~/RPM/BUILDROOT/ & \%\_buildrootdir & Установка\\
					\hline
					~/RPM/RPMS/ & \%\_rpmdir & Собранные .rpm\\
					~/RPM/SRPMS/ & \%\_srcrpmdir & ``Собранный'' .srpm \\
					\end{tabular}
				\end{table}			
\end{frame}

\begin{frame}
\def~{$\sim$}
\frametitle{Пример 0: Hello, World!}

	\begin{block}{Упражнение: hello.spec}
		\begin{itemize}
			\item Скопировать файл описания сборки {\tt hello.spec} в {\tt ~/RPM/SPECS/}
			\item Скопировать файл c исходным кодом в {\tt ~/RPM/SOURCES/}
			\item Собрать пакет с помощью {\tt rpmbuild -ba ~/RPM/SPECS/hello.spec}
			\item Посмотреть получившуюся файловую структуру {\tt find ~/RPM/}
		\end{itemize}
	\end{block}

\end{frame}

\begin{frame}
	\frametitle{Шаги сборки}
	
			\begin{table}
				\begin{tabular}{l | l | l | l }
				Шаг & Чтение & Запись & Описание \\
				\hline 
				\%prep & \%\_sourcedir & \%\_builddir & Подготовка\\
				\%build & \%\_builddir & \%\_builddir & Сборка (configure, make)\\
				\%install & \%\_builddir & \%\_buildrootdir & Установка (make install)\\
				\%check &	\%\_builddir & \%\_builddir & Проверка (make test)\\
				\hline 
				bin & \%\_buildrootdir	&\%\_rpmdir & Упаковка RPM\\
				src & \%\_sourcedir & \%\_srcrpmdir & Упаковка SRPM\\
				\end{tabular}
			\end{table}
\end{frame}

\begin{frame}
	\frametitle{Структура spec-файла}
	\begin{itemize}
		\item Тэги -- {\tt Tag: value}
		\begin{itemize}
		\item Все из открытого шаблона .spec файла
		\item Отсутствующие в шаблоне
			\begin{itemize}
			\item {\tt Patch0: } имя файла с патчем
			\item {\tt BuildArch/ExcludeArch/ExclusiveArch: } архитектура (noarch)
			\item {\tt BuildRoot: } BUILDROOT каталог устарело
			\end{itemize}
		\item Зависимости
			\begin{itemize}
			\item {\tt BuildRequires:}
			\item {\tt Requires:}
			\item {\tt Provides:}
			\item {\tt Conflicts:}
			\item {\tt Obsoletes:}
			\item {\tt AutoReqProv: (0|1)}
			\end{itemize}
		\end{itemize}
	\end{itemize}
\end{frame}

\begin{frame}
	\frametitle{Структура spec-файла}

	\begin{itemize}
		\item Секции --  {\tt \%prep, \%build, \%install, \%check, \%files, \%changelog} 
		\item Макросы -- {\tt \%define FOO bar} 
		\item Скриптлеты, выполняемые в различных условиях\\
			{\tt (pre-,post-)install, uninstall, trigger}
		\item Подпакеты -- {\tt \%package}
		\item Условности -- {\tt \%ifarch ARCH\_NAME, \%if TRUE\_OR\_FALSE}
	\end{itemize}
	\begin{block}{Hint:}
		 {\tt rpm -\-showrc }
	\end{block}
\end{frame}

\begin{frame}
	\frametitle{Пример 1: Hello, World!}

	\begin{block}{Упражнение: hello.spec}
		\begin{itemize}
			\item Собрать пакет с помощью {\tt rpmbuild -ba hello.spec}
			\item Посмотреть список зависимостей {\tt rpm -qRp hello*.rpm}
			\item Установить полученный пакет
			\begin{itemize}
				\item {\tt apt-cache dotty hello > hello.dot}
				\item {\tt dot -Tpng hello.dot -Gdpi=300 -o hello.png}
			\end{itemize}
			\item Удалить исходники и spec-файл
			\item Пересобрать пакет из SRPM {\tt rpmbuild -{}-rebuild *.src.rpm}
		\end{itemize}
	\end{block}

	\pause

	\begin{block}{Упражнение: расширяем функционал}
		Добавить компиляцию, установку и пакетирование для CPP-примера.
	\end{block}

\end{frame}


\begin{frame}
	\frametitle{Разработка и использование в реальной системе}
	
	\begin{block}{Build-time vs Run-time}

		\begin{enumerate}
			\item Не все, что нужно во время компиляции, должно быть установлено в конечной системе.
			\item Не все, что нужно для работы программы, необходимо устанавливать на сборочной системе.
		\end{enumerate}
	\end{block}
	\begin{block}{Чистое сборочное окружение}
		\begin{itemize}
			\item Воспроизводимость сборки 
			\item Контроль зависимостей
			\item Контроль автоматически "подхваченных" зависимостей
		\end{itemize}
	\end{block}
\end{frame}

\begin{frame}
	\frametitle{Версии}

	\Large{Чтение правил дистрибутива строго обязательно!}
	
	Hint: {\tt rpmdev-vercmp (rpmdevtools)}
	\begin{block}{ Версия -- составная}
		{\tt \%name-\%epoch:\%version-\%release}
		\begin{itemize}
			\item {\tt Name: == \%name}
			\item {\tt Epoch: == \%epoch}
			\item {\tt Version: == \%version}
			\item {\tt Release: == \%release}
			\begin{itemize}
				\item Release:
				\item \%\{?dist\} tag
				\item rebuild number -- автоматически (при использовании нормальных билд-систем)
			\end{itemize}
		\end{itemize}
	\end{block}

\end{frame}


\begin{frame}
	\frametitle{Пример 2}

	\begin{block}{Подпакеты, скрипты, зависимости и обновление}
		На примере {\tt template.spec}:

		\begin{itemize}
			\item Собрать пакет с помощью {\tt rpmbuild -ba template.spec}
			\item Посмотреть список зависимостей {\tt rpm -qRp template*.rpm}
			\item Установить полученные пакеты
			\item Изменить версию, пересобрать и обновить в системе
			\item Построить дерево зависимостей
			\item Удалить полученные пакеты
		\end{itemize}
	\end{block}
\end{frame}

%\begin{frame}
%	\frametitle{Задача на дом: Пакетируем linking}

%	\begin{block}{Задача:}
%		Создать и установить пакеты для примера с созданием библиотек
%		и их использования из предыдущих лекций.
%	\end{block}

%	Hint: libtestA, libtestB, libtestA-devel, libtestB-devel, mainpkg

%\end{frame}

\begin{frame}
	\frametitle{Задача на дом}

	\begin{block}{Пакетируем архиватор}
		Создать RPM и SRPM пакеты для своего архиватора
	\end{block}

\end{frame}
