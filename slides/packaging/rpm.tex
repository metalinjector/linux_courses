\begin{frame}{RPM: команды}
	\begin{block}{Установка пакета}
		{\tt rpm -i [rpm-file1] ... [[url://]rpm-fileN] }
	\end{block}
	\begin{block}{Удаление пакета}
		{\tt rpm -e pkgname1 ... pkgnameN }
	\end{block}
	\begin{block}{Обновление пакета}
		{\tt rpm -U [rpm-file1] ... [[url://]rpm-fileN] }
	\end{block}
	\begin{block}{Проверка пакета}
		{\tt rpm -V pkgname1 ... pkgnameN }
	\end{block}
\end{frame}

\begin{frame}{RPM -q: часто используемые опции опроса}

	\begin{itemize}
		\item {\tt pkgname} -- выбор пакета, установленного в системе
		\item {\tt -a} -- все пакеты, установленные в системе
		\item {\tt -p} -- использовать файл RPM
	\end{itemize}


	\begin{itemize}
		\item {\tt -i} -- показать информацию пакета\\
			{\tt rpm \alert{-q} \alert{-i} glibc }
		\item {\tt -l} -- показать список файлов пакета \\
			{\tt rpm \alert{-q -l} glibc }
		\item {\tt -{}-whatprovides} -- \\
			{\tt rpm \alert{-q --whatprovides} java}
		\item {\tt -{}-whatrequires} -- \\
			{\tt rpm \alert{-q --whatrequires} /bin/bash}
		\item {\tt -{}-queryformat} -- формат вывода\\
			{\tt rpm \alert{-q -{}-whatrequires} /bin/bash \alert{-{}-queryformat ''\%\{name\} ''} }

	\end{itemize}

\end{frame}


\begin{frame}{Команды пакетных менеджеров}
        \begin{tabular}{ll}
            \multicolumn{2}{c}{Установка пакета }   \tabularnewline
            Debian & {\tt apt-get \alert{install} pkgname } \\
            CentOS & {\tt yum \alert{install} pkgname } \\
            \multicolumn{2}{c}{Обновление пакета }  \tabularnewline
            Debian & {\tt apt-get \alert{install} pkgname } \\
            CentOS & {\tt yum \alert{update} pkgname }  \\
            \multicolumn{2}{c}{Удаление пакета }   \tabularnewline
            Debian & {\tt apt-get \alert{remove} pkgname } \\ 
            CentOS & {\tt yum \alert{remove} pkgname }  \\
            \multicolumn{2}{c}{Поиск. По имени пакета}   \tabularnewline
            Debian & {\tt apt-cache \alert{search} pkgname } \\
            CentOS & {\tt yum \alert{list} pkgname }  \\
            \multicolumn{2}{c}{Поиск. По имени файла}   \tabularnewline
            Debian & {\tt apt-file \alert{search} path } \\
            CentOS & {\tt yum \alert{provides} file} 
        \end{tabular}
\end{frame}
