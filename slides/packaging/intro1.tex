\begin{frame}{RPM: структура пакета}
	\begin{itemize}
		\item Метаданные
			\begin{itemize}
				\item Имя
				\item Версия/Релиз
				\item Группа
				\item Описание
                                \item Зависимости
				\item ...
			\end{itemize}
		\item Архив с файлами
			\begin{itemize}
				\item cpio
			\end{itemize}
		\item Скрипты
			\begin{itemize}
				\item Pre Install
				\item Post Install
				\item Pre Uninstall
				\item Post Uninstall \bigskip
				\item Triggers
			\end{itemize}
	\end{itemize}
\end{frame}

\begin{frame}{Два уровня пакетных менеджеров}

\begin{tabular}{| l | c | r |}
      \hline
          Level &  RedHat-based & Debian-based \\ 
      \hline
          {\bf Low} & rpm & dpkg \\ 
      \hline
          {\bf High} & yum, dnf & apt, aptitude \\
      \hline
    \end{tabular}
    \\


    \alert{Низкоуровневые} используются для установки, удаления, получения информации о пакете. \\
    \alert{Высокоуровневые} предоставляют дополнительные функции такие как поиск по репозиторию, копирование пакета из репозитория, разрешение зависимостей, обновление системы.

\end{frame}
