%\newcounter{tmpc}

\begin{frame}{Репозиторий}
	\begin{block}{Репозиторий пакетов}
		Место, где хранятся и поддерживаются пакеты, а также сопутствующая мета-информация, предназначенное для использования пакетным менеджером.
	\end{block}
	\begin{block}{Пример: Fedora Core}
		\begin{itemize}
			\item Packages/*.rpm
			\item RPM-GPG-KEY-*
			\item repodata
			\begin{itemize}
				\item множество сжатых и несжатых XML файлов для YUM
			\end{itemize}
		\end{itemize}

		Описание репозтория для YUM на локальной системе хранится по пути
		{\tt /etc/yum.repos.d/*.repo}
	\end{block}
		
\end{frame}



%\begin{frame}[fragile]{Упражнение}
%  \begin{enumerate}
%      \item Создать на {\tt /dev/sda} раздел размером примерно 10Gb
%      \item Создать на этом разделе ext3 ФС и смонтировать раздел в {\tt /mnt/chroot}
%      \item Развернуть {\tt /media/nfs/pub/CentOS/precreated/centOS.tar.gz} в {\tt /mnt/chroot}
%      \item Смонтировать {\tt proc, sysfs} а также {\tt /dev} в соответствующие места {\tt /mnt/chroot}
%      \item {\tt chroot /mnt/chroot}
%      \item Отредактировать {\tt /etc/resolv.conf} -- скопировать туда информацию из {\tt resolv.conf} основной системы
%      \item Отредактировать {\tt /etc/yum.conf} Добавить следующий раздел
%\begin{minipage}{0.5\textwidth}
%\begin{verbatim}
%[base]
%  name = CentOS 6
%  baseurl = ftp://192.168.11.15/CentOS
%  gpgcheck = 0
%\end{verbatim}
%\end{minipage}
%\setcounter{tmpc}{\theenumi}
%\end{enumerate}
%\end{frame}
%\begin{frame}{Продолжение упражнения}
%  \begin{enumerate}
%      %\setcounter{enumi}{\thetmpc}
%      \item Обновить информацию о пакетах {\tt update}
%      \item Удалить пакет vim
%      \item Установить заново пакет vim
%      \item Посмотреть списки файлов для пакетов {\tt rpm, vim}
%      \item Найти, к какому пакету относится команда {\tt ls, top}
%      \item Найти пакет предоставляющий сервис ssh и установить его
%    \end{enumerate}
%\end{frame}
