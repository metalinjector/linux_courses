\begin{frame}{Операционая система GNU/Linux}
        Включает в себя:
	\begin{itemize}
		\item Ядро (Linux)
                \item компилятор (GCC) 
		\item библиотеки (glibc)
		\item утилиты и приложения (Userspace)
	\end{itemize}
	\begin{block}{Дистрибутив}
		\only<1>{\center{\bf{?}}}
		\pause
		\only<2->{это набор программного обеспечения на базе ядра Linux, распространяющийся как единое целое.}
	\end{block}
\end{frame}


\begin{frame}{Задачи дистрибутива}
        Обычно дистрибутив состоит из ПО с открытым исходным кодом, который можно скачать из сети, собрать и установить на копмьютер самостоятельно,
        Задча трудозатратная по времени. Разрабочики дистрибутивов уже выполнили ее, поэтому пользователи используют готовый дистрибутив.
	\begin{itemize}
		\item Предоставление комплекта ПО. 
		\item Средства установки и настройки
		\item Средства обновления
	\end{itemize}
\end{frame}

\begin{frame}{А есть ли разница?}
	\begin{itemize}
		\begin{columns}
		\column{0.4\textwidth}
			\item Система управления пакетами (может отсутствовать)
			\item Формат распространения ПО
			\item Пути к файлам
			\item Система сборки ПО
                        \item Документация
		\column{0.4\textwidth}
			\item Инсталлятор
			\item Первичные настройки
			\item Средства управления
			\item Набор ПО
                        \item Поддержка, в т.ч. коммерческая
		\end{columns}
	\end{itemize}
Чем похожи:
    \begin{itemize}
        \item Все дистрибутивы предоставляют стандартный интерфейс управления. 
        \item Ядро Linux представляет собой Unix-like OS. Linux API совместимый со стандартом POSIX, Single UNIX Specification (SUS) 
        \item Имея навыки работы в одном дисрибутиве легко переключиться в окружение другого.
    \end{itemize}
\end{frame}

\begin{frame}{Примеры дистрибутивов и их классификация.}
	\begin{itemize}
		\begin{columns}
		\column{0.3\textwidth}
			\item RedHat
			\item Fedora Core
			\item CentOS
			\item Scientific Linux
			\item Oracle Unbreakable Linux
		\column{0.3\textwidth}
			\item Slackware 
			\item Gentoo
			\item Arch
			\item OpenSUSE
			\item ALT Linux 
		\column{0.3\textwidth}
			\item Debian
			\item Ubuntu
			\item Mint
			\item Knoppix
			\item BackTrack
		\end{columns}
	\end{itemize}
    \begin{itemize}
        \item По назначению серверный или десктопный; 
        \item По скорости обновления: стабильный или обновляющийся; 
        \item по типу установки: загрузка прямо с ISO или установка на диск; 
        \item поддерживаемые архитектуры
    \end{itemize}
\end{frame}
