\begin{frame}{Авторское право и лицензии}

	\begin{block}{Авторское право}
		 Возникает по факту создания ПО 

		\begin{itemize}
			\item Неимущественные права
			\item Имущественные права
		\end{itemize}
	\end{block}

	\pause

	\begin{block}{Лицензии}
		Лицензия -- средство передать какие-либо права на продукт либо его часть.

		Необходима для защиты авторских прав. 
		Средство для возможности законно пресечь несанкционирование копирование,  использование или распространение ПО. 
	\end{block}
\end{frame}


\begin{frame}{Лицензии: открытые и свободные}
	\begin{block}{ Р.Столлман: 4 свободы}
		\begin{itemize}
			\item Свобода 0: Свобода запускать программу в любых целях.
			\item Свобода 1: Свобода изучения работы программы и адаптация её к вашим нуждам. 
				Доступ к исходным текстам является необходимым условием.
			\item Свобода 2: Свобода распространять копии,  так что вы можете помочь вашему товарищу.
			\item Свобода 3: Свобода улучшать программу и публиковать ваши улучшения,
				так что всё общество выиграет от этого.
				Доступ к исходным текстам является необходимым условием.
		\end{itemize}
	\end{block}
\end{frame}


\begin{frame}{Лицензии: permissive}
	\begin{columns}
	\column{0.3\textwidth}
		\center\includegraphics[width=2cm,natwidth=144,natheight=144]{../../slides/intro/three-arrows@2x.png}

	\column{0.6\textwidth}

	\begin{itemize}
		\item BSD
		\item MIT
		\item Apache
	\end{itemize}
	\end{columns}

	\begin{block}{I want it simple and permissive.}
		\begin{itemize}
			\item практически не ограничивают свободу действий пользователей ПО и разработчиков, работающих с исходным кодом.
			\item По своему духу, распространение работы под пермиссивной лицензией схоже с помещением работы в общественное
				достояние, но не требует отказа от авторского права.
		\end{itemize}
	\end{block}

\end{frame}


\begin{frame}{\textcopyleft -- Copyleft}

	\begin{columns}
	\column{0.3\textwidth}
		\center\includegraphics[width=2cm,natwidth=144,natheight=138]{../../slides/intro/circular@2x.png}

	\column{0.6\textwidth}

	\begin{itemize}
		\item GPL
		\item LGPL
		\item AGPL
	\end{itemize}
	\end{columns}


	\begin{block}{I care about sharing improvements.}
	
	Авторское лево -- концепция и практика использования законов авторского права для обеспечения 
	невозможности ограничить любому человеку право использовать,  изменять и распространять как 
	исходное произведение,  так и произведения,  производные от него.
	\end{block}


	При копилефте все производные произведения должны распространяться под той же лицензией,
	что и оригинальное произведение.

\end{frame}



