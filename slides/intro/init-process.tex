\begin{frame}{init - система инициализации}


\begin{block}{Упражнение. Какие задачи выполняет система инициализации?}
init --help
\end{block}

\begin{block}{Упражнение. Найти процесс init в дереве процессов}
{\tt pstree -Ap}
либо, т.к. вывод может быть слишком длинным
{\tt pstree -Ap | head}
\end{block}

\pause
 Выполняет следующие задачи:
\begin{itemize}
    \item Инициализация системы
    \item Перезагрузка, остановка системы
    \item Загрузка в режиме восстановления 
\end{itemize}
\end{frame}

\begin{frame}{Процесс init}
\alert{init} - первый процесс который выполняется ядром. \\
        Является родителем для всех процессов в системе. \\
        Идентификатор процесса
	\center{\large PID = 1}

	\begin{block}{Наиболее известные системы инициализации}
		\begin{itemize}
			\item systemd - использует DSL язык
			\item upstart - альтернативная система CentOS5 
			\item SysVInit - система на shell скриптах
		\end{itemize}
	\end{block}
    
\end{frame}

\begin{frame}{SysVInit: runlevels}
	\begin{block}{Управление}
		\begin{itemize}
			\item kernel boot parameters: <N> -- runlevel
			\item утилита {\tt runlevel}
			\item утилита {\tt init}
		\end{itemize}
	\end{block}

	\scriptsize
	\begin{block}{Runlevel}
		\begin{table}
			\begin{tabular}{| c | l | }
			\hline
			Runlevel & Описание\\
			\hline
			0	& Выключить систему \\
			1,s,single & Однопользовательский режим \\
			2	& Многопользовательский режим без графики. Без сетевых сервисов.\\
			3	& Многопользовательский режим без графики. Полноценная сеть. \\
			4	& Определяется на хосте\\
			5	& Многопользовательский режим с графикой.\\
			6	& Перезагрузка\\
			emergency & Аварийная оболочка \\
			\hline
			\end{tabular}
		\end{table}
	\end{block}
\end{frame}

\begin{frame}{SysVInit: сервисы}
	\begin{block}{Управление}
		\begin{itemize}
			\item утилита {\tt service}
			\item утилита {\tt chkconfig}
		\end{itemize}
	\end{block}

	\begin{block}{Сервисы}
		\begin{itemize}
			\item {\tt /etc/rc.d/init.d}
			\item {\tt /etc/rc.d/rc.N}\footnote{N=runlevel}
		\end{itemize}
	\end{block}
\end{frame}

\begin{frame}{systemd: runlevels}
	\begin{block}{Управление}
		\begin{itemize}
			\item kernel boot parameters\\
				{\tt systemd.unit=rescue.target} \\
			\item утилита {\tt systemctl} \\
				{\tt systemctl isolate multi-user.target} \\
				{\tt systemctl set-default single.target}
		\end{itemize}
	\end{block}

	\begin{block}{targets}
		\tiny
		\begin{table}
			\begin{tabular}{| c | l | l | }
			\hline
			Runlevel & Описание\\
			\hline
			0	& poweroff.target & Выключить систему \\
			1,s,single & rescue.target  & Однопользовательский режим \\
			2, 4	& multi-user.target & Определяется на хосте. default 3\\
			3	& multi-user.target & Многопользовательский режим без графики. Полноценная сеть. \\
			5	& graphical.target & Многопользовательский режим с графикой.\\
			6	& reboot.target & Перезагрузка\\
			emergency & emergency.target & Аварийная оболочка \\
			\hline
			\end{tabular}
		\end{table}
	\end{block}
\end{frame}

\begin{frame}{systemd: сервисы}
	\begin{block}{Управление}
		\begin{itemize}
			\item утилита {\tt systemctl}
			\item утилита {\tt journalctl}
		\end{itemize}
	\end{block}

	\begin{block}{Сервисы}
		\begin{itemize}
			\item {\tt /lib/systemd/system/}
			\item {\tt /etc/systemd/system/}
		\end{itemize}
	\end{block}
\end{frame}

\begin{frame}{systemd: systemctl управление сервисами}
		\begin{itemize}
			\item {\tt systemctl status service\_name}
			\item {\tt systemctl start service\_name}
			\item {\tt systemctl stop service\_name}
			\item {\tt systemctl is-active service\_name}
			\item {\tt systemctl restart service\_name}
			\item {\tt systemctl reload service\_name}
			\item {\tt systemctl enable service\_name}
			\item {\tt systemctl disable service\_name}
			\item {\tt systemctl is-enabled service\_name}
		\end{itemize}
\end{frame}
