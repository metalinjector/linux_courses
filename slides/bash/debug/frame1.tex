\begin{frame}[fragile,allowframebreaks]
	\frametitle{Включение режима отладки}
	Используется команда {\tt set} для активизации различных режимов работы {\tt bash}.

	Включение режима производится с помощью ''-'',\\
	а отключение -- ''+''.

	\begin{block}{Примеры включения режима отладки.}
		\begin{lstlisting}[language=sh,frame=single]
bash -x ./script.sh # опция командной строки

#!/bin/bash -x # внутри скрипта опцией
[.. script ..]

#!/usr/bin/env bash
set -x # внутри скрипта командой 

#!/usr/bin/env bash
[..irrelevant code..]
set -x  # включаем отладку на часть кода
[..relevant code..]  
set +x # выключаем отладку
[..irrelevant code..]
\end{lstlisting}
	\end{block}
\end{frame}
