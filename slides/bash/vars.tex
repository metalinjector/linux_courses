%% Vars


\begin{frame}
	\frametitle{Переменные}
	\large\center{Нетипизированные!!!}

	Для прямого обращения необходимо использовать префикс \\
	\center{\Large{\tt \$}}

	Фигурные скобки используют для отделения от текста:\\
	\center{\tt \$VARrest != \$\{VAR\}rest}

	%\bigskip
	
	\begin{alertblock}{Используют без префикса}
		\begin{itemize}
			\item в объявлении declare var
			\item в присвоении declare var=10
			\item чтение в командe read var
			\item удаление unset var
			\item в арифметических операциях {\tt (( a=b+c ))}
		\end{itemize}
	\end{alertblock}
\end{frame}

\begin{frame}[fragile]
	\frametitle{Задание. Присвоить переменной значение.}

	\begin{lstlisting}
#!/bin/bash

VAR=string
echo $VAR
	\end{lstlisting}


	\begin{block}{Изменить и посмотреть на результат.}
		\begin{itemize}
			\item Добавить пробел до знака ''{\tt =}''
			\item Добавить пробел после знака ''{\tt =}''
			\item Присвоить переменной VAR значение: I love \$\$\$!
			\item Создать переменные с другим именем и значением var 1var \_var var1
		\end{itemize}
	\end{block}

\end{frame}

\begin{frame}[fragile]
	\frametitle{Косвенное обращение к переменной}

	Косвенное (indirect) обращение к переменной: {\tt \$\{!VARIABLE\}}

	\begin{block}{Пример}
		\begin{lstlisting}
#!/bin/bash 
num=$# 
lastarg=${!num} 
echo $num $lastarg
		\end{lstlisting}
	\end{block}

\end{frame}


\begin{frame}
	\frametitle{Типы переменных}
	\begin{itemize}
		\item Локальные\\
		    Область видимости -- текущая программа, функция или субшелл
		\item Переменные окружения (внешние)
		\item Позиционные параметры (передаются как аргументы команды)
	\end{itemize}
\end{frame}

\begin{frame}
	\frametitle{Внешние переменные}

	\center{Наследование внешней переменной}

	\begin{itemize}
		\item \alert{export}
		\item Переданное в командной строке \\
			\begin{block}{Пример}
				{\tt TEST=123 make}
			\end{block}
	\end{itemize}
\end{frame}

\begin{frame}[fragile]
	\frametitle{Специальные переменные}
	Часто используемые в скриптах:
\begin{tabbing}
\hspace{1cm}\= \alert{\$HOSTNAME}\quad \= имя хоста\kill
            \> \alert{\$IFS}\quad\> разделитель \\[5pt] 
            \> \alert{\$HOME}\quad \> домашняя директория\\[5pt] 
            \> \alert{\$PWD}\quad \> текущая директория\\[5pt]
            \> \alert{\$UID}\quad \> идентификатор пользователя \\[5pt]
            \> \alert{\$\$}\quad \> идентификатор процесса\\[5pt]
            \> \alert{\$HOSTNAME}\quad \> имя хоста\\[5pt]
            \> \alert{\$PS1}\quad \> вид командной строки\\[5pt]
            \> \alert{\$PS4}\quad \> вид командной строки в режиме отладки
\end{tabbing}

    
\end{frame}


