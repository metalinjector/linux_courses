\begin{frame}[fragile]
	\frametitle{Запуск группы команд в Subshell.}
	
	\begin{block}{Пример}
\begin{lstlisting}
( echo 1; echo 2) | tee file
\end{lstlisting}
	\end{block}
	\begin{block}{( cmd1; cmd2)}
	    Запускается новый shell
	\end{block}
\end{frame}

\begin{frame}[fragile]
	\frametitle{Примеры использования Subshell.}
	\begin{block}{Пример. Сохраняем директорию.}
\begin{lstlisting}
echo "$PWD"
( cd /usr; echo "$PWD" )
echo "$PWD"
\end{lstlisting}
	\end{block}
\end{frame}

\begin{frame}[fragile]
	\frametitle{Примеры использования Subshell.}
	\begin{block}{Пример. Сохраняем значение переменной TEST}
\begin{lstlisting}
TEST=42; (echo Subshell TEST=$TEST; TEST=0; echo Subshell TEST=$TEST ); echo External TEST=$TEST
\end{lstlisting}
	\end{block}
	\begin{block}{Пример. Отличие от анонимной функции.}
\begin{lstlisting}
TEST=42; { echo Subshell TEST=$TEST; TEST=0; echo Subshell TEST=$TEST ; }; echo External TEST=$TEST
\end{lstlisting}
	\end{block}
\end{frame}
