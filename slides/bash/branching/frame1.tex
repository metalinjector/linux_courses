\begin{frame}[fragile]
\frametitle{Синтаксис {\bf if}}

	\begin{columns}
		\column{0.5\textwidth}
	
	\begin{lstlisting}[language=bash]
if command1
then
    OTHER COMMANDS
elif command2
then
    OTHER COMMANDS
else
    OTHER COMMANDS
fi
\end{lstlisting}
		\column{0.5\textwidth}
Варианты форматирования
then отдельной строкой
	\begin{lstlisting}[language=bash]
if command
then OTHER COMMANDS 
fi
\end{lstlisting}

then и if на одной строке
	\begin{lstlisting}[language=bash]
if command; then 
    OTHER COMMANDS
fi
\end{lstlisting}
	\end{columns}

В зависимости от результата выполнения (exit code) command1 выполняется блок команд после then.   

Ключевое слово fi - обязательно

%	\pause
%	{\bf Практическое задание:} \\
%	\begin{itemize}
%
%		\item с помощью конструкции {\bf if} проверить существует ли файловый объект передаваемый в качестве параметра скрипту ({\tt man test})
%		\item если нет, то создать директорию с таким именем
%		\item если cуществует и файл является shell-скриптом ({\tt man file}), то запустить его
%		\item если существует и является директорией, то вывести на экран Top5 по размеру файлов из этой директории, 
%		    отсортированных в порядке убывания ({\tt man ls, man head})
%%	\end{itemize}
%	\end{columns}
\end{frame}
