\begin{frame}
	\frametitle{Домашнее задание}
	\begin{itemize}
		\item Создать библиотеку функций:
			\begin{itemize}
				\item функция {\tt help};
				\item функция {\tt count\_str} -- подсчитывает количество найденных строк в stdin.
				\item функция {\tt readfile}, в которую передается имя файла в качестве 1 параметра
					и строка, которую необходимо найти, в качестве 2-го.
					Задача функции -- вырезать из файла все пустые строки и комментарии, 
					и с помощью предыдущей функции вернуть количество вхождений искомой строки.
			\end{itemize}
		\item Создать скрипт, который обрабатывает файл переданный через параметр -f или -{}-file;
		\item Включить библиотеку в свой скрипт с помощью {\tt source};
		\item Найти количество сервисов {\tt tcp} в файле {\tt /etc/services};
		\item Найти количество сервисов {\tt udp} в файле {\tt /etc/services};
  \end{itemize}
\end{frame}
