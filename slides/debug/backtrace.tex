%%
%% 
\begin{frame}[fragile]
	\frametitle{Встроенные функции gcc}

	\begin{block}{Адрес возврата ({\tt gcc})}
		
		Встроенные в {\tt gcc} функции для получения адреса вызывающей функции:

	\begin{lstlisting}
void * __builtin_return_address (unsigned int level);
	\end{lstlisting}
	\end{block}

	\begin{block}{Пример}
	
	Добавить в функцию {\tt second()} код:

	\begin{lstlisting}
printf("Frame 0: %p\n",  __builtin_return_address(0));
printf("Frame 1: %p\n",  __builtin_return_address(1));
printf("Frame 2: %p\n",  __builtin_return_address(2));
printf("Frame 3: %p\n",  __builtin_return_address(3));
	\end{lstlisting}
	\end{block}

\end{frame}

\begin{frame}[fragile]
	\frametitle{{\tt addr2line}}

	\begin{block}{Перевод адреса в строку}
		{\tt addr2line <address>}

		\begin{itemize}
			\item {\tt -e filename} -- бинарный файл
			\item {\tt -f} -- показать имена функций 
			\item {\tt -i} -- для inline функций -- показывать первую не встроенную
			\item {\tt -C} -- нормальные имена для "искалеченных" (C++) функций
		\end{itemize}
	\end{block}

	\begin{block}{Упражнение}

		\begin{itemize}
			\item Вызвать {\tt addr2line}, передав туда адреса, полученные из {\tt \_\_builtin\_return\_address()}.
			\item Перекомпилировать с флагом {\tt -g}
			\item Повторить вызов {\tt addr2line}
		\end{itemize}

	\end{block}

\end{frame}

%\begin{frame}[fragile]
%	\frametitle{Пример}
%
%	\begin{block}{Вывод {\tt backtrace}}
%	\begin{verbatim}
%./backtrace() [0x400c3e]
%/lib64/libc.so.6(+0x350f0) [0x7fb5711450f0]
%./backtrace(fill_random_ascii_buffer+0x40) [0x400d11]
%./backtrace(main+0xc9) [0x400dfe]
%	\end{verbatim}
%
%	\end{block}
%	\begin{block}{Вызов {\tt addr2line}}
%	\begin{verbatim}
%addr2line -e backtrace -ifC 0x400c3e
%	\end{verbatim}
%
%	\end{block}
%
%\end{frame}

\begin{frame}[fragile]
	\frametitle{Добавим ошибку}

	\begin{block}{Ошибка}
	
	\begin{lstlisting}
void second (void) {
  char *buf;
  sprintf( buf, "Hello, world!\n");
}
	\end{lstlisting}
	\end{block}
	
\end{frame}
\begin{frame}[fragile]
  \frametitle{Обработчик}
	\begin{block}{И обработчик}
	\begin{lstlisting}
void sighandler( int signal) {
  printf("SIGSEGV received, exiting.\n");
  exit(1);
}
int main (void) {
  struct sigaction action;
  action.sa_handler = sighandler;
  action.sa_flags=0;
  sigemptyset(&action.sa_mask);
  sigaction (SIGSEGV, &action,NULL);
	\end{lstlisting}
	\end{block}

\end{frame}



\begin{frame}[fragile]
	\frametitle{BackTrace}

	\begin{block}{Backtrace}
		Список вызовов функций\footnote{{\tt backtrace\_symbols} -- может не работать при проблемах с памятью!}.

	\end{block}

	\begin{lstlisting}
#include <execinfo.h>

int backtrace (void **buffer, int size);
char ** backtrace_symbols (void *const *buffer, int size);
void backtrace_symbols_fd (void *const *buffer, int size, int fd);
	\end{lstlisting}
\end{frame}

\begin{frame}[fragile]
	\frametitle{Пример использования}

	Для использования с {\tt GNU ld} линковать с опцией {\tt -rdynamic}.

	\begin{block}{Упражнение: вызов {\tt backtrace} (backtrace.c)}

	\begin{lstlisting}
void *trace[TRACEDEPTH];
size_t trace_size,  i;
char **trace_msg;
trace_size = backtrace(trace,  TRACEDEPTH);
trace_msg = backtrace_symbols( trace,  trace_size);

for( i=0; i<trace_size; i++) printf("%s\n",  trace_msg[i]);

free( trace_msg);

backtrace_symbols_fd( trace, trace_size, STDERR_FILENO);
	\end{lstlisting}

	\end{block}
\end{frame}


