\begin{frame}[fragile]{Задание. Виды команд в оболочке}
Команда  \alert{type} отображает тип команды. Выполним ее для различных команд.
\begin{lstlisting}[language=bash]
type type cd help alias read
type dmesg rm
type if
type -a ls # отображает тип всех команд в системе
type -a echo pwd test
\end{lstlisting}
Два типа команд. А оно нам надо? Есть ли разница?
\pause
Встроенные и внешние команды
\begin{itemize}
    \item всегда присутствуют в интерпретаторе, внешних может и не быть на диске
    \item однообразный синтаксис на разных платформах (переносимость скриптов)
    \item как правило выполняются быстрее, т.к. код находится в памяти
    \item есть средства, чтобы выключить встроенные команды, либо использовать прямой путь
\end{itemize}
\end{frame}
