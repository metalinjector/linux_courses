\begin{frame}[fragile]{xargs}
			Утилита для создания и запуска команд из стандартного потока ввода:
		\begin{verbatim}
xargs [options] command [command options]
                \end{verbatim}
		\begin{itemize}
			\item {\tt -d} -- разделитель
			\item {\tt -0} -- null-terminated строки
			\item {\tt -I text} -- подстановка
			\item {\tt -n N} -- максимальное количество аргументов
			\item {\tt -P N} -- максимальное количество процессов
		\end{itemize}

%Для работы с разделителями в имени файла: пробелы, tab, символ новой строки. 
Использовать -print0 в команде find для замены на ASCII NUL в имени файла.
\end{frame}

\begin{frame}[fragile]{xargs}
	\begin{block}{Примеры}
		\begin{verbatim}
file /bin/*  | grep shell | cut -f 1 -d ':' | xargs wc -l 
# calculate number of strings in all shell scripts
                \end{verbatim}
		\begin{verbatim}
find /etc -type f -size -100k | \
 xargs tar -czf /tmp/archive-100k.tar.gz
                \end{verbatim}
		\begin{verbatim}
find /etc -type f | xargs -I {} echo "Найден {} файл"
                \end{verbatim}

		\begin{verbatim}
find . -type f -name "*.mp3" -print0 | \
 xargs -0 -n 1 -P 0 -I mp3 avconv -i mp3 mp3.ogg
                \end{verbatim}
	
	\end{block}
\end{frame}
