\begin{frame}[fragile]{Как правильно задавать вопросы}
From FAQ How To Ask Questions The Smart Way
Before You Ask
  \begin{itemize}
	  \item Try to find an answer by reading the manual.
	  \item Try to find an answer by reading a FAQ.
	  \item Try to find an answer by searching the archives of the forum you plan to post to.
	  \item Try to find an answer by searching the Web.
	  \item Try to find an answer by inspection or experimentation.
	  \item Try to find an answer by asking a skilled friend.
	  \item If you're a programmer, try to find an answer by reading the source code.
    \end{itemize}
\end{frame}


\begin{frame}[fragile]{Встроенная документация}
\begin{itemize}
    \item \textbf{man} - помощь по внешним командам
    \pause
    \item \textbf{info} - расширенная помощь по некоторым командам (texinfo format)
    \pause
    \item \textbf{find /usr/share/doc/} - файлы документации поставляемые вместе с приложением
    \item \textbf{-h, --help option} - встроенная в приложение справка
    \item \textbf{help} - встроенная помощь по внутренним командам bash (также man bash)
\end{itemize}
\end{frame}

\begin{frame}[fragile]{Основное о man}
      \textbf{man \textless command\_name\textgreater }

	\begin{block}{Example: show uptime manual page}
		{\tt man uptime}
	\end{block}

		\begin{itemize}
			\item Прочитайте {\tt man man} !
		\end{itemize}

\end{frame}

\begin{frame}[fragile]{man page navigation}
		\begin{itemize}
			\item \textbf{up, down} - scroll one line
			\item \textbf{q} - exit
			\item \textbf{/pattern} - search pattern
			\item \textbf{n} - next text pattern
			\item \textbf{N} - repeat search in back direction
			\item \textbf{h} - help
		\end{itemize}
\end{frame}

\begin{frame}[fragile]{Page structure}
		\begin{itemize}
			\item NAME
			\item SYNOPSIS
			\item DESCRIPTION
			\item EXAMPLES
			\item SEE ALSO
		\end{itemize}
\end{frame}

\begin{frame}[fragile]{Разделы помощи}
	\begin{itemize}
		\item[1] Основная секция(юзерские программы)
		\item[2] Syscalls
		\item[3] С library
		\item[5] Конфигурационные файлы
		\item[8] Системные службы
	\end{itemize}
\end{frame}

\begin{frame}[fragile]{More than one section of the manual}
	name(section)  \\ 
	\textbf{man(1)} and \textbf{man(7)}, or \textbf{exit(2)} and \textbf{exit(3)} \\
     \begin{block}{Example: show manual in section 5 and 1}
        \begin{lstlisting}
man -f passwd #or whatis passwd 
man 5 passwd; man 1 passwd; man -wa passwd
        \end{lstlisting}
    \end{block}
\end{frame}

\begin{frame}[fragile]{Поиск по страницам помощи}
     \begin{block}{Упражнение. Поиск страниц с ключевым словом.}
        \begin{lstlisting}
man -f passwd #or whatis passwd 
man -k passwd #or apropos passwd 
whatis  -l -w '*'
man -s 3 -Kw passwd
        \end{lstlisting}
    \end{block}
\end{frame}


%\begin{frame}[fragile]{Чему научились}
%  \begin{itemize}
%  \item Как спрашивать у сообщества
%  \item Умеем использовать 3 источника получения информации man, info, help
%  \item Как перемещаться по страницам помощи info и man
%  \item Иcкать в системе помощи man и запрашивать из одного из 8-ми разделов 
%  \end{itemize}
%\end{frame}
