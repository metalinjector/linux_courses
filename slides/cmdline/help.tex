\begin{frame}[fragile]{Как правильно задавать вопросы}
From FAQ How To Ask Questions The Smart Way
Before You Ask
  \begin{itemize}
	  \item Try to find an answer by reading the manual.
	  \item Try to find an answer by reading a FAQ.
	  \item Try to find an answer by searching the archives of the forum you plan to post to.
	  \item Try to find an answer by searching the Web.
	  \item Try to find an answer by inspection or experimentation.
	  \item Try to find an answer by asking a skilled friend.
	  \item If you're a programmer, try to find an answer by reading the source code.
    \end{itemize}
\end{frame}


\begin{frame}[fragile]{Источники получение помощи}
\begin{itemize}
    \pause
    \item \textbf{man} - помощь по внешним командам
    \pause
    \item \textbf{help} - встроенная помощь по внутренним командам bash (также man bash)
    \pause
    \item \textbf{info} - расширенная помощь по некоторым командам (texinfo format)
    \item \textbf{/usr/share/doc/} - файлы документации поставляемые вместе с приложением
\end{itemize}


     \begin{block}{Упражнение. Другие источники помощи.}
        \begin{lstlisting}
            help;  help help
            man -h ; info --help
            man -k intro or apropos intro
            man -f intro or whatis intro
            man 3 intro ; man 1 intro
            man -wa intro
        \end{lstlisting}
    \end{block}
\end{frame}

\begin{frame}[fragile]{Основное о man}
		\begin{itemize}
			\item Прочитайте {\tt man man} !
			\item apropos, аналог {\tt man -k <слово>}
                        \item whatis, аналог {\tt man -f <слово>} 
			\item Разделы (sections)
				\begin{itemize}
					\item[1] Основная секция(юзерские программы)
					\item[2] Syscalls
					\item[3] С library
					\item[5] Конфигурационные файлы
					\item[8] Системные службы
				\end{itemize}
		\end{itemize}
	  \textbf{Замечание}
	  Обычно внутри страницы работает поиск с помощью '/'
	
\end{frame}


%\begin{frame}[fragile]{Чему научились}
%  \begin{itemize}
%  \item Как спрашивать у сообщества
%  \item Умеем использовать 3 источника получения информации man, info, help
%  \item Как перемещаться по страницам помощи info и man
%  \item Иcкать в системе помощи man и запрашивать из одного из 8-ми разделов 
%  \end{itemize}
%\end{frame}
