 \begin{frame}{Редакторы}

  \begin{itemize}
    \item Любой редактор, с которым вы можете справится. \pause Но его может не быть в вашей системе\ldots \pause
    \item Редактор \alert{vi} присутствует как стандартный в любой Unix-подобной системе\footnote{В этом качестве он внесен в стандарт Single Unix Specification. Существует множество реализаций редакторов, совместимых c vi: vim, elvis, nvi, vi-mode в Emacs, Sublime Text 2, vi из busybox и т.д.} \pause
  \end{itemize}

\end{frame}

\begin{frame}{vi и vim}
  Перед стартом:

  \begin{enumerate}
    \item  Редактор vi изначально создавался как универсальный и переносимый\footnote{Обязан работать на любых типах терминалов и виртуальных консолей}. Все действия можно осуществить на алфавитно-цифровой части клавиатуры, без мыши. \pause
    \item \alert{Редактор командного стиля}\footnote{Командного стиля, а не меню-ориентированный}. Действия подачей прямых управляющих команд. \pause \newline
      3 основных режима: \sout{портить текст и противно бибикать}
      \begin{itemize}
        \item[-] \alert{Командный режим} (Normal mode) - по умолчанию при запуске.
        \item[-] \alert{Режим изменения текста} (Edit mode)
        \item[-] \alert{Режим построчного редактирования} (Ex mode) - операции над файлом целиком\footnote{сохранение, открытие файлов, выход, вставка файла в текущий и т.д.}.
      \end{itemize}
  \end{enumerate} \pause
  \alert{Упражнение}: проходим \alert{vimtutor}, встроенный в vim учебник\footnote{ export LANG='ru\_RU.UTF-8' - на русском языке }
\end{frame}

