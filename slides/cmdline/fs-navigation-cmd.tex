
\begin{frame}{Файловая система.}
  \begin{itemize}
    \item В UNIX (и Linux) файлы организованы в виде \emph{единой древовидной структуры} (дерева), называемой \alert{файловой системой}.
    \item \alert{Каждый файл имеет имя}, определяющее его расположение в дереве FS.
    \item Корнем дерева является \alert{корневой каталог} (root directory), имеющий имя \alert{"/"}.
  \end{itemize} \pause

  \begin{itemize}
    \item \alert{полный путь} начинается с \alert{/}(корневого каталога), каталоги разделяются также \alert{/}. \newline
      Пример: /home/user/.ssh/authorized\_key
    \item \alert{относительный путь} - от текущего каталога \newline
      Примеры: ../user10/.bashrc; ./script; ls script
  \end{itemize}

\end{frame}

\begin{frame}[fragile]{Имя файла.}
        Допустимые символы в имени файла:
                \begin{itemize}
                    \item регистр имеет значение \alert{fileA} и \alert{Filea} - два разных имени
                    \item \alert{/} разделяет директории
                    \item \alert{-} \textbackslash <Space> специальные символы 
                    \item \alert{.file} - скрытый файл
                \end{itemize}

      \begin{block}{Примеры. Cпециальные имена.}
        \alert{./} - текущая директория
        \alert{../} - предыдущая директория
      \end{block}
\end{frame}

\begin{frame}[fragile]{Навигация по файловой системе}
 Для каждой запущенной программы (в том числе и shell) в системе определен \alert{текущий каталог}. 
      \begin{itemize}
		  \item {\tt pwd} -- имя текущей директории (help pwd)
		  \item {\tt ls} -- список файлов в директории. По умолчанию в текущей (man ls)
		  \item {\tt cd} -- смена текущей директории (help cd)
      \end{itemize}
      \begin{block}{Упражнение. Заходим в /usr/bin/ и просматриваем список доступных команд.}
	\begin{lstlisting}
        pwd
        cd /usr/bin/
        pwd
        ls
        cd -
        pwd
\end{lstlisting}
      \end{block}
\end{frame}

\begin{frame}[fragile]{Просмотр типов файлов}
      \begin{block}{Упражнение. Символьное обозначение типа файла.}
Первая буква вывода команды ls -l обозначает тип файла. 
Tip. Команды file позволяет определить тип файла. Ключ -d для работы с
директорией. 

/dev/zero

/dev/sda 

.. 

/bin/sh 

/dev/log

/dev/stdout
      \end{block}
\end{frame}

\begin{frame}{Типы файлов в Unix}
  Для пользователя:
  \begin{itemize}
    \item \alert{Обычные файлы (regular file)}: .bashrc, /bin/bash
    \item \alert{Каталоги (directory)}: /home/user1, /usr, /, /usr/local  \pause
    \item \alert{Символические ссылки (symbolic links)}: /bin/sh, /dev/stdout
  \end{itemize} \pause
  Для администратора:
  \begin{itemize}
    \item \alert{Файлы устройств (device special file)}:
      \begin{itemize}
        \item \alert{блочные}: /dev/sda5, /dev/loop0, /dev/sr0
        \item \alert{символьные}: /dev/null, /dev/mem, /dev/tty
      \end{itemize}
  \end{itemize} \pause
  Для программиста:
  \begin{itemize}
    \item \alert{FIFO (named pipe)}: /dev/xconsole
    \item \alert{Socket}: /dev/log
  \end{itemize}

\end{frame}

\begin{frame}[fragile]{Создание (текстовых) файлов}
  \pause
  \begin{enumerate}
    \item Любимым редактором: \alert{vi/vim}, \alert{nano}, \alert{mcedit} \pause
    \item \alert{echo}
\begin{lstlisting}
echo какой-то текст >file
\end{lstlisting}\pause
    \item \alert{cat} с перенаправлением\footnote{До ``конца ввода'', т.е. нажатия Ctrl+D}
\lstinputlisting[basicstyle=\small]{../../sam-solutions/samples/cat-input-file} \pause
    \item \alert{touch} - создать пустой файл
\begin{lstlisting}
~$ touch file4
~$
\end{lstlisting}
  \end{enumerate}
\end{frame}

\begin{frame}[fragile]{Команды для работы с файлами}
Any ideas what is purpose of this commands? 
	\begin{itemize}
		\begin{columns}
		\column{0.2\textwidth}
			\item cp
			\item mv
			\item rm
			\item file
		\column{0.2\textwidth}
			\item touch
			\item ln
			\item mkdir
			\item mknod
			\item mkfifo
		\end{columns}
	\end{itemize}
\end{frame}


\begin{frame}[fragile]{Операции над каталогами (и файлами)}
  \begin{itemize}
    \item \alert{mkdir} - создать каталог
\begin{lstlisting}
~$ mkdir dir1 /tmp/somedir
~$ mkdir -p dir/and/existant/parts/in/path
\end{lstlisting} \pause
    \item \alert{rmdir} - удалить (пустой) каталог
\begin{lstlisting}
~$ rmdir dir1 /tmp/somedir
~$ rmdir -p deep/empty/dir/structure/
\end{lstlisting} \pause
    \item \alert{cp} - копирование файлов\footnotemark[8]
    \item \alert{mv} - перемещение и переименование файлов
    \item \alert{rm} - удаление файлов\footnotemark[17]
\begin{lstlisting}
~$ rm -rf dir1 /tmp/somedir
~$ cp /etc/passwd /tmp/passwd # copy file to file 
~$ cp -r /etc/  /tmp/ # copy directory to directory
\end{lstlisting} \pause
  \end{itemize}
\footnotetext[8]{с ключом \alert{-r}) и каталогов}
\end{frame}
