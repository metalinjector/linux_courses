\begin{frame}[fragile]{Путь к файлу.}
      \begin{itemize}
        \item Имя файла
                \begin{itemize}
                    \item чувствительно к регистру
                    \item / разделяет директории
                    \item - \textbackslash <Space> специальные символы 
                    \item . скрытый файл
                \end{itemize}
        \item \alert{Полное имя файла} (absolute pathname) начинается с /
        \item \alert{Относительное имя файла} (relative pathname) начинается с
        любого другого символа. Поиск файла производится относительно
        \alert{текущей директории} (current working directory).
      \end{itemize}

      \begin{block}{Примеры}
        ./ - текущая директория

        ../ - предыдущая директория

        ../../usr/bin/ls

        /usr/bin/ls
      \end{block}
\end{frame}

\begin{frame}[fragile]{Навигация по файловой системе}
      \begin{itemize}
		  \item {\tt pwd} -- имя текущей директории (help pwd)
		  \item {\tt ls} -- список файлов в директории. По умолчанию в текущей (man ls)
		  \item {\tt cd} -- смена текущей директории (help cd)
      \end{itemize}
      \begin{block}{Упражнение. Заходим в /usr/bin/ и просматриваем список доступных команд.}
	\begin{lstlisting}
	pwd
	cd /usr/bin/
	pwd
	ls
	cd -	
	pwd
	\end{lstlisting}
      \end{block}
\end{frame}

\begin{frame}[fragile]{Просмотр типов файлов}
      \begin{block}{Упражнение. Символьное обозначение типа файла.}
Первая буква вывода команды ls -l обозначает тип файла. 
Tip. Команды file позволяет определить тип файла. Ключ -d для работы с
директорией. 

/dev/zero

/dev/sda 

.. 

/bin/sh 

/dev/log

/dev/stdout
      \end{block}
\end{frame}

\begin{frame}[fragile]{Команды для работы с файлами}
	\begin{itemize}
		\begin{columns}
		\column{0.2\textwidth}
			\item touch
			\item ln
			\item mkdir
			\item mknod
			\item mkfifo

		\column{0.2\textwidth}
			\item cp
			\item mv
			\item rm
			\item rmdir
			\item file
			\item install

		\column{0.4\textwidth}
			\begin{block}{Упражнение}
				\begin{enumerate}
					\item Создать иерархию директорий
						\begin{lstlisting}
dir1/dir1.1/dir1.1.1
dir1/dir1.2/dir1.2.1
dir1/dir1.2/dir1.2.2
						\end{lstlisting}
					\item Внутри каждой создать файл
					\item Удалить все созданное
				\end{enumerate}
			\end{block}
			
		\end{columns}
	\end{itemize}
\end{frame}
