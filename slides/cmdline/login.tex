% пользовательская сессия
\begin{frame}
  \frametitle{Пользовательская сессия}

  \begin{center}
    \begin{block}<1->{Многопользовательская система?}
      Надо представиться системе. Логин и пароль.
    \end{block}

    \begin{block}<2->{Как может выглядеть}
      \includegraphics[height=2.4cm]{console-login-screenshot}
      \includegraphics[height=2.4cm]{gdm-login-screenshot}
      \includegraphics[height=2.4cm]{putty-login-screenshot}
    \end{block}

    \begin{block}<3->{Виды сессий}
      \begin{itemize}
        \item локальные и удалённые (сетевые)
        \item текстовые и графические
      \end{itemize}
    \end{block}

  \end{center}

\end{frame}

SSH is typically used to log into a remote machine and execute commands, but it also supports tunneling, forwarding TCP ports and X11 connections; it can transfer files using the associated SSH file transfer (SFTP) or secure copy (SCP) protocols.[2] SSH uses the client-server model.

\begin{frame}[fragile]
	\frametitle{Secure Shell (SSH)}
	Used for:
      \begin{itemize}
        \item execute commands 
        \item copy files (SFTP, SCP)
        \item forwarding TCP ports and X11 connections
        \item tunneling
      \end{itemize}
	SSH uses the client-server model

	The standard TCP port 22 
\end{frame}

\begin{frame}[fragile]
  \frametitle{Входим удалённо.}

    Много реализаций клиентов и серверов.

    \emph{Как может выглядеть:}
    \newline
    \fbox{\includegraphics[height=3.5cm]{console-ssh-screenshot}}
    \emph{ }
    \fbox{\includegraphics[height=3.5cm]{putty-config-screenshot}}

\end{frame}

\begin{frame}[fragile]
  \frametitle{SSH address}

\begin{block}{Access parameters}
    IP address: 192.168.10.10 \par
    Username: root, val, user \par
    Port: 22 default or any \par
\end{block}
    Full path: root@192.168.10.10:22
\end{frame}

\begin{frame}
  \frametitle{Конфигурация клиента}
\begin{columns}
    \begin{column}{0.5\textwidth}
      {\Large Linux: .ssh/config}
      \lstinputlisting[basicstyle=\tiny]{../../sam-solutions/samples/ssh-config} 
    \end{column}
    \begin{column}{0.5\textwidth}  %%<--- here
   {\Large Windows: PuTTY session}
        \begin{figure}
        \centering
            \includegraphics[scale=0.35]{putty-config-screenshot} 
        \end{figure}
    \end{column}
\end{columns}
\end{frame}

\begin{frame}
  \frametitle{Выход из матрицы}

  \begin{center}
    \includegraphics[height=5.0cm]{matrix-screenshot}
    \pause
    \newline
    \begin{itemize}
      \item Команда \emph{exit}, команда shell \emph{logout}
      \item Hotkey \emph{Ctrl+d}
      \item Закрыть клиент
    \end{itemize}
  \end{center}

\end{frame}
