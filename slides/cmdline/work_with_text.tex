\begin{frame}{Текст в Unix}
  В Unix (и Linux) в виде обычного текста или \alert{plain text} представлены:\pause
  \begin{itemize}
    \item \alert{конфигурационные файлы}, как локальные\footnote{в каталоге \$HOME} , так и общесистемные\footnote{в каталоге /etc} \pause
    \item \alert{системные логи}\footnote{справедливо для \alert{syslog} и совместимых систем}
    \item \alert{исходные тексты программ}, включая скрипты на Shell
    \item \alert{основной формат ввода и (или) вывода данных} для множества программ и утилит
  \end{itemize} 

\end{frame}



\begin{frame}{Текстовый фильтр}

  Определение:\newline \alert{Текстовый фильтр} - программа, обрабатывающая и преобразующая текст. \newline

	Пример: {\tt man | tac | rev }
  \begin{itemize}
    \item Фильтр, запущенный без параметров - читает стандартный ввод.
    \item Параметры фильтра - интерпретируются как имена файлов
    \item Ключи фильтра - управляют режимами работы
  \end{itemize} \pause

  Фильтр почти всегда используется совместно с перенаправлением ввода-вывода Shell (особенно '|', pipes). cmd1 | cmd2

\end{frame}


%\begin{frame}{Регулярные выражения}
%
%  \begin{block}{Как описать текст?}
%    Необходим инструмент и формат описания текста
%  \end{block} \pause

%  \begin{block}{Регулярные выражения}
%    \alert{Регулярные выражения (regular expression или regexp)} -  специальные строки символов, которые задаются для поиска совпадающих фрагментов. Иначе говоря это способ описания наборов букв.
%  \end{block} \pause
%
%  \begin{block}{Универсальный язык описания текста}
%    Все Unix-программы, осуществляющие поиск в тексте, используют регулярные выражения.
%  \end{block}
%
%\end{frame}
%
%\begin{frame}[fragile]{Элементы регулярных выражений}
%    \begin{itemize}
%      \item \alert{литералы} - обычные символы (буквы и цифры) \pause
%      \item \alert{метасимволы} - спецсимволы (количество повторов, группировка фрагментов, позиция в тексте).
%    \end{itemize} \pause
%
%    Примеры регулярных выражений:\newline
%\begin{lstlisting}
%~$ file /bin/* | grep symbolic
%~$ grep -o 'user[0-9]*' /var/log/auth.log
%\end{lstlisting}
%\end{frame}
%
%%\subsection{Метасимволы}
%\begin{frame}{Класс на 1 символ}
%  \begin{itemize}
%    \item \alert{.} (точка)  - заменяет любой символ \newline
%      Пример: \regex{us.r.} = 'user0', 'us rX', 'us9r ' и т.д. \pause
%    \item \alert{[ ]} символьный класс - заменяет любой символ из перечисленных в скобках
%      \begin{enumerate}
%        \item \regex{user[0-9]} = 'user0', 'user5', но не равно 'user'
%        \item \regex{-[abc-]} = '--', '-a', '-b', '-c', но не равно '--a'
%        \item \regex{[\textasciicircum{}abc]1}\footnote{инвертировать символьный класс} = 'd1', '11', но не равно 'a1'
%      \end{enumerate} \pause
%    \item \alert{[:class:]} - дополнительные POSIX-классы для символов, \alert{внутри символьного класса}\footnote{Да, на редкость уродливый синтаксис} \newline
%        Примеры классов: \alert{[:alnum:]}, \alert{[:alpha:]} \alert{[:digit:]} \alert{[:space:]} \alert{[:lower:]}, \alert{[:upper:]}, \alert{[:print:]} \newline
%        Примеры regexp с POSIX классами: \regex{[ы[:digit:]]}
%  \end{itemize}
%\end{frame}
%
%
%\begin{frame}[fragile]{квантификаторы - регулируем повторы}
%
%  \alert{Квантификаторы} указывают, сколько раз может повторяться символ или выражение, после которого указаны.  Не являются шаблонами текста.
%
%  \begin{itemize}
%    \item \alert{?} - необязательный символ \newline
%      пример: \regex{a.?b} - совпадёт с 'ab', 'a9b', 'a b' \pause
%    \item \alert{*} - любое количество символов, включая нулевое \newline
%      примеры: \regex{.*}, \regex{[[:digit:]]*} \pause
%    \item \alert{+} - не менее одного символа \newline
%      примеры: \regex{[a-d]+}, \regex{(02:)+}
%  \end{itemize}
%\end{frame}
