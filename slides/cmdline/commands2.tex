\begin{frame}[fragile]{Сжатие одного файла}

	\begin{block}{Команды: gzip, bzip, xz}
		\begin{itemize}
			\item {\tt -[1-9]} -- изменить уровень сжатия
			\item {\tt -d} -- распаковать
			\item {\tt -c} -- вывод на консоль
		\end{itemize}
	\end{block}

	\begin{block}{Примеры: эффективность сжатия}
		\begin{verbatim}
dd if=/dev/sda of=./big_file.01 bs=1M count=10
cp ./big_file.01 ./big_file.02
gzip ./big_file.01 ; xz ./big_file.02
du -h ./big_file.*
dd if=/dev/sda bs=1M count=1 | gzip -c > backup.gz
    \end{verbatim}
	\end{block}
\end{frame}

\begin{frame}[fragile]{Архивация файлов}
	\begin{block}{Команда: tar}
		\begin{itemize}
			\item {\tt -c} -- создать архив
			\item {\tt -x} -- извлечь из архива
				\begin{itemize}
					\item {\tt -C} -- перейти в директорию
				\end{itemize}
			\item {\tt -f} -- запись в файл
			\item {\tt -x -j -J} -- сжатие gzip, bzip, xz
		\end{itemize}
	\end{block}
	Создать архив:
	\begin{verbatim}
tar -czf /tmp/etc.tar /etc/
        \end{verbatim}
\end{frame}

\begin{frame}[fragile]{Архивация: примеры}
	Создать сжатый архив:
	\begin{verbatim}
tar -czf archive.tar.gz ./*
        \end{verbatim}
	\pause
	Распаковать сжатый архив в директорию {\tt /tmp}:
	\begin{verbatim}
tar -C /tmp/ -xzf archive.tar.gz
        \end{verbatim}
	\pause
	Просмотреть список файлов в архиве:
	\begin{verbatim}
tar -tzf archive.tar.gz
        \end{verbatim}
	%\pause
	%Создать копию текущей директории на другом хосте:
	%\begin{verbatim}
%HostDest: netcat -l 2222 | gzip -dc | tar -C /tmp/copy/ -x
%HostSrc:  tar -c * | gzip -9 | netcat HostDest 2222
%        \end{verbatim}
\end{frame}
