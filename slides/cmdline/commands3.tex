\begin{frame}[fragile]{Поиск файлов}
    \textbf{find} ищет файлы в заданной директории и производит над ним заданную операцию.
	\begin{block}{Параметры поиска}
		\begin{itemize}
			\item {\tt -name}, {\tt -iname} -- имя файлового объекта, включая метасимволы 
			\item {\tt -type} -- тип файлового объекта
			\item {\tt -size} -- размер [cwbkMG]
			\item {\tt -perm} -- права доступа
			\item {\tt -user} -- владелец
			\item {\tt ...} -- другие опции man find 
		\end{itemize}
	\end{block}
	\begin{block}{Действия над результом поиска}
		\begin{itemize}
			\item {\tt -print} -- вывод на stdout (по умолчанию)
			\item {\tt -printf} -- форматированный вывод
			\item {\tt -exec} -- выполнить команду
			\item {\tt -ls} -- замена -exec ls -l \{\} ;
			\item {\tt -delete} -- удалить файл
		\end{itemize}
	\end{block}
\end{frame}

\begin{frame}[fragile]{Поиск файлов}
	\begin{block}{Примеры}
            \begin{verbatim} find . -name '*.o' -print \end{verbatim}
            \begin{verbatim} find -name '*.o' \end{verbatim}
            \begin{verbatim} find -type d -user altlinux \end{verbatim}
            \begin{verbatim} find /root \( -name '*.pyc' -o -name '*.py' \) \
-type f -user root -size +300k -size -1024k \
-exec ls -l \{\} \; \end{verbatim}
	\end{block}
 Дополнительно:
 Позволяет преодолеть лимит на кол-во аргументов в командной строке. 
 Например, когда в директории много файлов и тогда возникают ошибки
 \textquotedblleft Arguments too long.\textquotedblright 
\end{frame}

\begin{frame}[fragile]{xargs}
	\begin{block}{xargs}
			Утилита для создания и запуска команд из стандартного потока ввода:
		\begin{verbatim}
xargs [options] command [command options]
		\end{verbatim}

		\begin{itemize}
			\item {\tt -d} -- разделитель
			\item {\tt -0} -- null-terminated строки
			\item {\tt -I text} -- подстановка
			\item {\tt -n N} -- максимальное количество аргументов
			\item {\tt -P N} -- максимальное количество процессов
		\end{itemize}

	\end{block}
Имя файла может содержать разделители: пробелы, tab, символ новой строки. В этом случае xargs воспримет имя как набор раздельных аргументов. Использовать -print0 в команде find для замены на ASCII NUL в имени файла.
\end{frame}

\begin{frame}[fragile]{xargs}
	\begin{block}{Примеры}

		\begin{verbatim}
file /bin/*  | grep shell | cut -f 1 -d ':' | xargs wc -l # calculate number of strings in all shell scripts
		\end{verbatim}

		\begin{verbatim}
find /etc -type f -size -100k | \
 xargs tar -czf /tmp/archive-100k.tar.gz
		\end{verbatim}


		\begin{verbatim}
find /etc -type f | xargs -I {} echo "Найден {} файл"
		\end{verbatim}

		\begin{verbatim}
find . -type f -name "*.mp3" -print0 | \
 xargs -0 -n 1 -P 0 -I mp3 avconv -i mp3 mp3.ogg
		\end{verbatim}
	
	\end{block}
\end{frame}
