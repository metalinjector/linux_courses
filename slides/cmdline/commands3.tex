\begin{frame}[fragile]{Поиск файлов командой find}
    \alert{find} ищет файлы в заданной директории и производит над ним заданную операцию.
	\begin{block}{Часто используемые параметры поиска}
		\begin{itemize}
			\item {\tt -name}, {\tt -iname} -- имя файлового объекта, включая метасимволы 
			\item {\tt -type} -- тип файлового объекта
			\item {\tt -size} -- размер [cwbkMG]
			\item {\tt -perm} -- права доступа
			\item {\tt -user} -- владелец
			\item {\tt ...} -- другие опции man find 
		\end{itemize}
	\end{block}
\end{frame}

\begin{frame}[fragile]{Файлы найдены}
	\begin{block}{Действия над результом поиска}
		\begin{itemize}
			\item {\tt -print} -- вывод на stdout (по умолчанию)
			\item {\tt -printf} -- форматированный вывод
			\item {\tt -exec} -- выполнить команду
			\item {\tt -ls} -- замена -exec ls -l \{\} ;
			\item {\tt -delete} -- удалить файл
		\end{itemize}
	\end{block}
\end{frame}

\begin{frame}[fragile]{Примеры использования команды find}
            В текущей директории найти все файлы *.o и вывести на экран 
            \begin{verbatim} find . -name '*.o' -print \end{verbatim}
            \begin{verbatim} find -name '*.o' \end{verbatim}
            Поск по типу и владельцу файла.
            \begin{verbatim} find -type d -user altlinux \end{verbatim}
            Составная команда, множество условий
            \begin{verbatim} find /root \( -name '*.pyc' -o -name '*.py' \) \
-type f -user root -size +300k -size -1024k \
-exec ls -l \{\} \; \end{verbatim}
 Дополнительно: позволяет преодолеть лимит на кол-во аргументов в командной строке. 
 \textquotedblleft Arguments too long.\textquotedblright 
\end{frame}
