
\begin{frame}{Регулярные выражения}

  \begin{block}{Как описать текст?}
    Необходим инструмент и формат описания текста
  \end{block} \pause

  \begin{block}{Регулярные выражения}
    \alert{Регулярные выражения (regular expression или regexp)} -  специальные строки символов, которые задаются для поиска совпадающих фрагментов. Иначе говоря это способ описания наборов букв.
  \end{block} \pause

  \begin{block}{Универсальный язык описания текста}
    Все Unix-программы, осуществляющие поиск в тексте, используют регулярные выражения.
  \end{block}

\end{frame}

\begin{frame}[fragile]{Элементы регулярных выражений}
    \begin{itemize}
      \item \alert{литералы} - обычные символы (буквы и цифры) \pause
      \item \alert{метасимволы} - спецсимволы (количество повторов, группировка фрагментов, позиция в тексте).
    \end{itemize} \pause

    Примеры регулярных выражений:\newline
\begin{lstlisting}
~$ file /bin/* | grep symbolic
~$ grep -o 'user[0-9]*' /var/log/auth.log
\end{lstlisting}
\end{frame}

%\subsection{Метасимволы}
\begin{frame}{Класс на 1 символ}
  \begin{itemize}
    \item \alert{.} (точка)  - заменяет любой символ \newline
      Пример: \regex{us.r.} = 'user0', 'us rX', 'us9r ' и т.д. \pause
    \item \alert{[ ]} символьный класс - заменяет любой символ из перечисленных в скобках
      \begin{enumerate}
        \item \regex{user[0-9]} = 'user0', 'user5', но не равно 'user'
        \item \regex{-[abc-]} = '\textendash\textendash', '-a', '-b', '-c'
        \item \regex{[\textasciicircum{}abc]1}\footnote{инвертировать символьный класс} = 'd1', '11', но не равно 'a1'
      \end{enumerate} \pause
    \item \alert{[:class:]} - дополнительные POSIX-классы для символов, \alert{внутри символьного класса}\footnote{Да, на редкость уродливый синтаксис} \newline
        Примеры классов: \alert{[:alnum:]}, \alert{[:alpha:]} \alert{[:digit:]} \alert{[:space:]} \alert{[:lower:]}, \alert{[:upper:]}, \alert{[:print:]} \newline
        Примеры regexp с POSIX классами: \regex{[ы[:digit:]]}
  \end{itemize}
\end{frame}


\begin{frame}[fragile]{квантификаторы - регулируем повторы}

  \alert{Квантификаторы} указывают, сколько раз может повторяться символ или выражение, после которого указаны.  Не являются шаблонами текста.

  \begin{itemize}
    \item \alert{?} - необязательный символ \newline
      пример: \regex{a.?b} - совпадёт с 'ab', 'a9b', 'a b' \pause
    \item \alert{*} - любое количество символов, включая нулевое \newline
      примеры: \regex{.*}, \regex{[[:digit:]]*} \pause
    \item \alert{+} - не менее одного символа \newline
      примеры: \regex{[a-d]+}, \regex{(02:)+}
  \end{itemize}
\end{frame}
