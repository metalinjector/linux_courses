\begin{frame}{Network diagnostic utilities.}
  \begin{tabular}{ | c | c | }
    \hline
    TCP/IP Layer & Utility \\ \hline 
    Data link & arp / ip ne / ip link / ip addr \\ \hline
    Network & ping / mtr / traceroute \\ \hline
    Transport & netstat / ss \\ \hline
    Application & telnet / netcat / nc /  \\ \hline
    Application DNS & host / dig / nslookup  \\ \hline
  \end{tabular}
    \\
   Multipurpose tools:
		\begin{itemize}
			\item Sniffers: tcpdump, wireshark
			\item Scanner: nmap
		\end{itemize}

\end{frame}


%\begin{frame}{Полезные утилиты: практика}
%
%	\begin{columns}
%		\column{0.5\textwidth}
%		\begin{block}{netstat}
%
%			Узнать:
%			\begin{itemize}
%				\item список используемых сокетов
%				\item серверных сокетов
%				\item имена/pid серверов
%				\item узнать номера портов
%			\end{itemize}
%		\end{block}
%	
%		\pause
%		\column{0.5\textwidth}
%		\begin{block}{telnet/netcat}
%
%			\begin{itemize}
%				\item Чат по протоколу TCP с соседом
%				\item Чат по протоколу UDP с соседом
%				\item Передать текстовый и бинарный файлы
%			\end{itemize}
%	
%			При создании чата использовать {\tt netstat} и {\tt tcpdump}
%			для получения информации о соединении.
%		\end{block}
%	
%	\end{columns}
%\end{frame}
%
%nmap
%1. сканирование соседа
%2. сканирование выделенных портов у соседа (поиск сервера чата) 
%3. узнать список открытых портов на всех машинах в 505
%4. узнать список  работающих машин
%
%tcpdump
%0. pcap файлы/libpcap
%1. запуск монитора
%2. запуск чата
%3. монитор-фильтр-анализ
%
