\begin{frame}{Дерево файловой системы - простое}
\includegraphics[height=8cm]{filesystem} 
\end{frame}

\begin{frame}{Детали реализации}
  \begin{itemize}
    \item \alert{VFS - virtual file system} - файлы и каталоги отображаются в единое дерево, независимо от их физического расположения.
  \end{itemize}
  \includegraphics[height=3.5cm]{vfs-and-devices}
\end{frame}

\begin{frame}{Монтирование}
  \begin{itemize}
    \item \alert{Монтирование} - процесс отображения содержимого устройства в указанную папку файловой системы.
    \item Команды:
      \begin{itemize}
        \item монтировать - ( \alert{mount} ) 
        \item размонтировать ( \alert{umount} )
      \end{itemize}
    \item \alert{mount} без параметров - вывести список уже подключенных файловых систем
  \end{itemize}
      \begin{block}{Упражнение. Дерево монтирования.}
     Получить вывод смонтированных блочных устройств в виде дерева с помощью команды: \alert{findmnt}
      \end{block}
  
\end{frame}

