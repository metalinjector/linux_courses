\begin{frame}
 \frametitle{Системы автоматического документирования}
 \begin{itemize}
  \item Donald Knuth, literate programming, 1983
  \item perldoc, etc. 1994
  \item javadoc, 1995
  \item Doxygen, 1997
 \end{itemize}
\end{frame}

\begin{frame}
 \frametitle{Как использовать}
 \begin{itemize}
  \item Добавить комментарии в формате Doxygen в исходники
  \item \texttt{doxygen -g Doxyconfig}
  \item Редактировать Doxyconfig
  \item doxygen Doxyconfig
 \end{itemize}
\end{frame}


\begin{frame}[fragile]
 \frametitle{Некоторые важные параметры в конфигурационном файле}
 \begin{itemize}
  \item \verb+PROJECT_NAME+
  \item \verb+PROJECT_NUMBER+
  \item \verb+INPUT+  
  \item \verb+RECURSIVE+
  \item \verb+EXCLUDE+
  \item \verb+EXCLUDE_PATTERNS+
  \item \verb+EXTRACT_ALL+
 \end{itemize}
\end{frame}

\begin{frame}[fragile]
 \frametitle{Краткий комментарий}
 \begin{block}{До кода}
   Добавить дополнительный символ "/"
   \begin{lstlisting}[language=C]
/// Краткое описание
void function(void);
\end{lstlisting}
 \end{block}
 \begin{block}{После кода}
   Добавить дополнительный символ "/<"
   \begin{lstlisting}[language=C]
void function(void); ///< краткое описание
\end{lstlisting}
 \end{block}
\end{frame}

\begin{frame}[fragile]
 \frametitle{Детальный комментарий}
 \begin{block}{До кода}
   Добавить дополнительный символ "*":
   \begin{lstlisting}[language=C]
/** Детальное описание
  * функции */
void function(void);
\end{lstlisting}
 \end{block}
 \begin{block}{После кода}
   Добавить дополнительный символ "*<":
    \begin{lstlisting}[language=C]
void function(void); /**< Детальное 
  * описание
  * функции */
\end{lstlisting}
 \end{block}
\end{frame}

\begin{frame}[fragile]
 \frametitle{Пример комментариев для файла}
\begin{lstlisting}[language=C]
/**
  @file filename
  @brief Краткое описание

  Детальное описание
  @author Автор
  @copyright Copyright (c) 2016, Author/Company
  @license This file is released under the GNU Public License.
  @bug Нет известных багов
*/
\end{lstlisting}
\end{frame}

\begin{frame}[fragile]
 \frametitle{Образец комментариев для функции}
\begin{lstlisting}[language=C]
/**
  @brief Краткое описание функции

  Детальное описание
  @param[in] parameter1 Описание первого параметра
  @param[out] parameter2 Описание второго параметра
  @see Ссылка на другую функцию
  @return Что возвращает

*/
int function(int parameter1, char *parameter2) 
\end{lstlisting}
\end{frame}

\begin{frame}[fragile]
 \frametitle{Образец комментариев для данных}
\begin{lstlisting}[language=C]
/** 
 * Описание структуры 
 */
typedef struct { 
  /** 
  * @name Coordinates
  */
  /*@{*/
  double x ; /**< the x coordinate*/ 
  double y ; /**< the y coordinate */
  double z ; /**< the z coordinate */
  /*@}*/
  /*@{*/
  char * name ; /**< the name of the point */
  int namelength ; /**< the size of the point name */
  /*@}*/
  } point3d
\end{lstlisting}
\end{frame}


\begin{frame}
 \frametitle{Упражнение}
  \begin{itemize}
    \item В проекте helloworld откомментировать каждую функцию в стиле doxygen
    \item Сгенерировать документацию
  \end{itemize}
\end{frame} 

\begin{frame}
 \frametitle{Домашнее задание}
  \begin{itemize}
    \item Добавить документацию  в стиле doxygen в свою реализацию архиватора Хаффмана
    \item Сгенерировать документацию
  \end{itemize}
\end{frame} 
