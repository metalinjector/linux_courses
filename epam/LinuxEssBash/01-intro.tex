% xetex compatible variant that support TTF fonts according to company rules
\documentclass[ignorenonframetext, professionalfonts, hyperref={unicode}]{beamer}

\usetheme{Epam}

\usepackage{fontspec}
%\setsansfont{SourceSansPro-Regular}
%\setbeamerfont{frametitle}{family=\fontspec{Oswald}}
%\setbeamerfont{block title}{family=\fontspec{Oswald}}

%\setsansfont{Arial}
%setbeamerfont{frametitle}{family=\fontspec{Arial}}
%setbeamerfont{block title}{family=\fontspec{Arial}}

%\setmainfont{Times New Roman}
%\setromanfont{Trebuchet MS}
\defaultfontfeatures{Mapping=tex-text}
\defaultfontfeatures{Ligatures=TeX}


\usepackage{cmap}
\usepackage{graphicx}

\usepackage{textcomp}

\usepackage{beamerthemesplit}

\usepackage{ulem}

\usepackage{verbatim}
\usepackage{import}

\usepackage{listings}
\lstloadlanguages{bash}

\lstset{escapechar=`,
	captionpos=b,
	extendedchars=false,
	language=sh,
%	frame=single,
	tabsize=2, 
	columns=fullflexible, 
%	basicstyle=\scriptsize,
	keywordstyle=\color{blue}, 
	commentstyle=\itshape\color{brown},
%	identifierstyle=\ttfamily, 
	stringstyle=\mdseries\color{green}, 
	showstringspaces=false, 
	numbers=left, 
	numberstyle=\footnotesize, 
	breaklines=true, 
	inputencoding=utf8,
	keepspaces=true,
	morekeywords={u\_short, u\_char, u\_long, in\_addr}
	}

\definecolor{darkgreen}{cmyk}{0.7, 0, 1, 0.5}

\lstdefinelanguage{diff}
{
    morekeywords={+, -},
    sensitive=false,
    morecomment=[l]{//},
    morecomment=[s]{/*}{*/},
    morecomment=[l][\color{darkgreen}]{+},
    morecomment=[l][\color{red}]{-},
    morestring=[b]",
}

\author[Epam]{{\bf Epam}\\Low Level Programming Department}

%\institution[EPAM]{EPAM}
%\logo{\includegraphics[width=1cm]{logo.png}}

\graphicspath{{../../slides/cmdline/clipart/}{../../slides/bash/clipart/}}

\bibliographystyle{unsrt}
\setbeamertemplate{bibliography item}{\insertbiblabel}

\AtBeginSection[]{%
  \begin{frame}<beamer>
    \frametitle{}
    \tableofcontents[
        sectionstyle=show/shaded, hideallsubsections ]
  \end{frame}
  \addtocounter{framenumber}{-1}% If you don't want them to affect the slide number
}

% \regex for regular expressions
\newcommand{\regex}[1]{ %
\expandafter{$\ulcorner{\color{blue}\texttt{#1}}\lrcorner$} %
}



\title{Введение в GNU/Linux}


%%%%%%%%%%%%%%%%%%%%%%%%%%%%%%%%%%%%%%%%%%%%%%%%%
%%%%%%%%%% Begin Document  %%%%%%%%%%%%%%%%%%%%%%
%%%%%%%%%%%%%%%%%%%%%%%%%%%%%%%%%%%%%%%%%%%%%%%%%

\begin{document}

\begin{frame}
	\frametitle{}
	\titlepage
	\vspace{-0.5cm}
	\begin{center}
	%\frontpagelogo
	\end{center}
\end{frame}


\begin{frame}
	\tableofcontents
	[hideallsubsections]
\end{frame}

%%%%%%%%%%%%%%%%%%%%%%%%%%%%%%%%%%%%%%%%%   
%%%%%%%%%% Content starts here %%%%%%%%%%
%%%%%%%%%%%%%%%%%%%%%%%%%%%%%%%%%%%%%%%%%

\section{О курсе}

% add one slide with unix/linux history, age, culture, motivation why it is important to learn
\mode<all>{\begin{frame}{Цель курса}
	\begin{center}
		\large
		Обеспечить быстрый старт в среде GNU/Linux для опытных пользователей
	\end{center} 
    \pause
         \center\includegraphics[scale=0.6]{linux_story} 

    Система сложная, поэтому время потраченное на самостоятельно изучение системы исчисляется годами.
\end{frame}

\begin{frame}{Читает курс}

    \large Vikentsi Lapa 
    \begin{columns}
        \column{0.6\textwidth}
            \begin{itemize}
                \item Senior Software Test Automation Engineer
                \item Work with Linux about 8 years in such areas as
                \begin{itemize}
                    \item High Perfomance Computing
                    \item Cluster file systems: GPFS, Lustre 
                \end{itemize}
                \item MLUG (Minsk Linux User Group Activist)
                \item LVEE conference participant 
            \end{itemize}
        \column{0.3\textwidth}
            \center\includegraphics[width=3cm]{myphoto}
    \end{columns}
\end{frame}

\begin{frame}{Аудитория}

Для кого предназначен этот курс:
\begin{itemize}
    \item IT Engineers, DevOps Engineers
    \item Junior Software Engineers/Software Engineers
    \item Test Engineers, Test Automation Engineers
    \item Students
\end{itemize}
    \pause
Требования к слушателям:
\begin{itemize}
    \item Опыт работы с Linux/Unix меньше 1 года или без опыта
    \pause
    \item Опыт работы с системой виртуализации
    \pause
    \item Знание сетевых технологий
    \pause
    \item Creative thinking and problem solving skills
    \pause
    \item \alert{Feedback is essential: answer to questions, do practical tasks, do homework, ask questions}
\end{itemize}
\end{frame}

\begin{frame}{Проверим обратную связь}
\begin{itemize}
    \item Есть ли опыт работы с Linux? Если есть то сколько?
    \pause
    \item Ваши ожидания. Зачем пришли на курс?
    \pause
    \item Установили дистрибутив? Какой?
    \item Если не установили. Почему?
\end{itemize}
\end{frame}

\begin{frame}[fragile]{Возраст Unix.}
 Посчитаем возраст системы Unix.
\begin{lstlisting}
    $ date --help | grep '%s'
    $ date +%s # show current date in seconds 
    $ echo $(( $(date +%s)/(60*60*24*365))) # show years
\end{lstlisting} 
\pause
Unix systems are characterized by a \alert{modular design} or ``Unix philosophy"
		\begin{columns}
		\column{0.7\textwidth}
\begin{itemize}
    \item simple tools that each perform a limited, well-defined function
    \item shell scripting and command language to combine the tools to perform complex workflows
    \item unified filesystem
\end{itemize}
		\column{0.3\textwidth}
            \includegraphics[width=4cm]{40years}
		\end{columns}
% За это время сложилась культура и философия разработки, которая оказала влияние на интерфейсы взаимодействия, например названия устройств и команд.
% Разнообразие подходов при решении задач. Среда дружественная для программиста и опытного технического пользователя.
\end{frame}


%\begin{frame}{Состав курса}
%	\begin{itemize}
%		\item Представление об архитектуре GNU/Linux дистрибутива
%			\pause
%		\item "Ежедневные" навыки работы в консоли. Конфигурация системы.
%			\pause
%		\item Язык программирования
%			\begin{itemize}
%					\item shell (Bash)
%			\end{itemize}
%			\pause
%		\item Все, чего вы не знали и боялись спросить
%	\end{itemize}
%\end{frame}
}

\section{Архитектура Linux системы.}
% activities ask about OS 
% What does OS do? Why it is necessary? Is it possible to work without OS?

% TODO images
% Состав ОС Linux
\mode<all>{\begin{frame}{Operation system functions.}
\includegraphics[height=2cm]{hw_tower.jpg} 
\includegraphics[height=2cm]{hw_smartphone.jpg}
\includegraphics[height=2cm]{server_1u.jpg}
	\begin{itemize}
            \item Is it possible to work without OS?
            \item Why it is necessary?
	    \item What does OS do?
	\end{itemize}
\end{frame}

\begin{frame}{Operation system GNU/Linux components.}
    \begin{columns}
        \column{0.6\textwidth}
    \includegraphics[scale=1]{unix_arch_highlevel}
        \column{0.4\textwidth}
	\begin{itemize}
		\item Kernel (Linux)
		\item Libraries (glibc)
                \item Compiler (GCC) 
		\item System utilities and applications (Userspace)
	\end{itemize}
    \end{columns}
\end{frame}
}
\mode<all>{\begin{frame}[fragile]{Задачи ядра Linux}
    \begin{columns}
    \column{0.6\textwidth}
        \includegraphics[scale=0.63]{unix_arch_details}
    \column{0.4\textwidth}
	\begin{itemize}
		\item Драйвера устройств
		\item Инициализация системы
		\item Управление
                    \begin{itemize}
                        \item процессами и потоками
                        \item CPU и памятью
                        \item файлами
                    \end{itemize}
		\item IPC (Inter Process Communication)
		\item Разграничение доступа
	\end{itemize}
    \end{columns}
\end{frame}
}
% Задачи ядра Linux
%\mode<all>{\begin{frame}{Задачи ядра Linux}
	\begin{itemize}
		\item Инициализация системы
		\item Управление процессами и потоками
		\item Управление памятью
		\item Управление файлами
		\item IPC (Inter Process Communication)
		\item Разграничение доступа
		\item Сетевые возможности
		\item Интерфейс доступа к возможностям ядра
	\end{itemize}
	%Выполнить команду dmesg и найти строки инициализации памяти, CPU, дисковой подсистемы, CPU.
\end{frame}
}
\subsection{Процесс загрузки ОС Linux}
\mode<all>{\begin{frame}{Процесс загрузки GNU/Linux}
        Demo
        \pause
	\scriptsize
	\begin{enumerate}
		\item BIOS
		\item Master Boot Record (MBR)
			\pause
		\item Загрузка загрузчика 
		\begin{itemize}
		\footnotesize
			\item Stage 1 -- Первичный загрузчик
			\item Stage 1,5 -- Загрузка ядра загрузчика и драйвера ФС
			\item Stage 2 -- загрузчик читает конфигурацию, загружает ядра и образ initrd (initial-RAM disk) в память
                        \item Передает управление ядру
		\end{itemize}

		\item Запуск программы инициализации в initrd, загрузка драйверов файловых систем (LVM, RAID, NFS)
			\pause
		\item Нахождение и монтирование корневого раздела
			\pause
		\item Запуск программы init
		\begin{itemize}
		\footnotesize
			\item Монтирование оставшихся разделов ФС
			\item Запуск демонов для заданного уровня загрузки (runlevel)
			\item Выдает приглашение пользователю. 
		\end{itemize}

	\end{enumerate}
\end{frame}
}

\section{Дистрибутивы ОС Linux}
% TODO ISO example
% TODO setup methods
\mode<all>{\begin{frame}{GNU/Linux Distibution structure}
Linux distro is an \alert{operation system}. 
	\begin{itemize}
		\item Software Collection
		\item Linux Kernel
		\item Package Management System (optional)
	\end{itemize}
\end{frame}
}
\mode<all>{\begin{frame}{Различия между дистрибутивами.}
	\begin{itemize}
		\begin{columns}
		\column{0.4\textwidth}
			\item Система управления пакетами (может отсутствовать)
			\item Формат распространения ПО
			\item Система сборки ПО
			\item Пути к файлам
                        \item Документация
		\column{0.4\textwidth}
			\item Инсталлятор
			\item Первичные настройки
			\item Средства управления
			\item Набор ПО
		\end{columns}
	\end{itemize}
Чем похожи:
    \begin{itemize}
        \item Все дистрибутивы имеют стандартный интерфейс управления - командная строка. 
        \item Ядро Linux представляет собой Unix-like OS. Linux API совместимый со стандартом POSIX, Single UNIX Specification (SUS).
        \item Навыки работы универсальны.
    \end{itemize}
\end{frame}
}
\mode<all>{\begin{frame}{Примеры популярных Linux дистрибутивов.}
	\begin{itemize}
		\begin{columns}
		\column{0.3\textwidth}
			\item RedHat
			\item Fedora Core
			\item CentOS
			\item Scientific Linux
			\item Oracle Unbreakable Linux
		\column{0.3\textwidth}
			\item Slackware 
			\item Gentoo
			\item Arch
			\item OpenSUSE
			\item ALT Linux 
		\column{0.3\textwidth}
			\item Debian
			\item Ubuntu
			\item Mint
			\item Knoppix
			\item BackTrack
		\end{columns}
	\end{itemize}
    \begin{itemize}
        \item По назначению серверный или десктопный; 
        \item По скорости обновления: стабильный или обновляющийся; 
    \end{itemize}
\end{frame}
}
\mode<all>{\begin{frame}{Installation methods}
	\begin{itemize}
            \item Install from existing Linux 
            \item Network installation 
            \item Use preinstalled virtual machine image (cloud)
            \item Write the image on flash media or optical disc
	\end{itemize}
\end{frame}
}
% activity ISO list what info can you get from ISO names?
\mode<all>{\begin{frame}{Установка с ISO образа}
    Что можно сказать о дистрибутиве по имени образа?
	\begin{enumerate}
            \item debian-8.5.0-amd64-netinst.iso
            \item CentOS-7-x86\_64-DVD.iso
            \item debian-8.5.0-powerpc-DVD-1.iso
            \item openSUSE-Leap-42.1-DVD-x86\_64.iso
            \item archlinux-2016.08.01-dual.iso
	\end{enumerate}
\end{frame}
}


% activity login to system. putty, ssh, my IP address
\section{Вход в систему.}
\mode<all>{% пользовательская сессия
\begin{frame}
  \frametitle{Пользовательская сессия}

  \begin{center}
    \begin{block}<1->{Многопользовательская система?}
      Надо представиться системе. Логин и пароль.
    \end{block}

    \begin{block}<2->{Как может выглядеть}
      \includegraphics[height=2.5cm]{console-login-screenshot}
      \includegraphics[height=2.5cm]{gdm-login-screenshot}
      \includegraphics[height=2.5cm]{putty-login-screenshot}
    \end{block}

    \begin{block}<3->{Виды сессий}
      \begin{itemize}
        \item локальные и удалённые (сетевые)
        \item текстовые и графические
      \end{itemize}
    \end{block}

  \end{center}

\end{frame}

\begin{frame}
  \frametitle{Входим удалённо. SSH}

  \begin{center}

    SSH - \emph{S}ecure \emph{SH}ell
    \newline \pause

    Протокол удалённой работы по сети для Linux.

    Много реализаций клиентов и серверов.
    \newline
    \pause

    \emph{Как может выглядеть:}
    \newline
    \fbox{\includegraphics[height=3.5cm]{console-ssh-screenshot}}
    \emph{ }
    \fbox{\includegraphics[height=3.5cm]{putty-config-screenshot}}
  \end{center}

\end{frame}

\begin{frame}
  \frametitle{Выход из матрицы}

  \begin{center}
    \includegraphics[height=5.0cm]{matrix-screenshot}
    \pause
    \newline
    \begin{itemize}
      \item Команда \emph{exit}, команда shell \emph{logout}
      \item Hotkey \emph{Ctrl+d}
      \item Закрыть клиент
    \end{itemize}
  \end{center}

\end{frame}
}
\subsection{Беспарольный вход}
\mode<all>{\begin{frame}
  \frametitle{Методы авторизации SSH. Ключи}

  \alert{Не хочешь вводить пароли?}
  \pause

  \alert{Не вводи!}
  \pause

  \begin{center}
    Aвторизация в SSH.

    Пара: открытый + секретный ключ.

    Вместо пароля.
  \end{center}

\end{frame}

\begin{frame}
  \frametitle{Ключи SSH. Создание}

  \begin{center}

    \begin{tabular}{ l r }
      \hbox{Linux: ssh-keygen} & \hbox{Windows: PuTTY keygen} \\
      \includegraphics[height=3cm]{ssh-keygen-screenshot} & \includegraphics[height=4cm]{putty-keygen-screenshot} \\
      Ключи в папке .ssh/: & Ключи там, где сохранили \\
      id\_rsa и id\_rsa.pub &
    \end{tabular}

  \end{center}

\end{frame}


\begin{frame}
  \frametitle{Ключи SSH. Копирование}

  \begin{itemize}
    \item \alert{Что копировать?} Публичный ключ (id\_rsa.pub)  \pause
    \item \alert{Куда складывать?} в \$HOME/.ssh/authorized\_keys \newline удалённой машины \pause
    \item \alert{Как перенести?} \pause
      \begin{itemize}
        \item Linux: ssh-copy-id username@host \pause
        \item Copy-paste в редактор \pause
      \end{itemize}
  \end{itemize}

\end{frame}

\begin{frame}
\begin{columns}
    \begin{column}{0.5\textwidth}
      {\Large Linux: .ssh/config}
      \lstinputlisting[basicstyle=\tiny]{../../sam-solutions/samples/ssh-config} 
    \end{column}
    \begin{column}{0.5\textwidth}  %%<--- here
   {\Large Windows: PuTTY session}
        \begin{figure}
        \centering
            \includegraphics[scale=0.35]{putty-config-screenshot} 
        \end{figure}
    \end{column}
\end{columns}
\end{frame}
}

% activity question. ask about Examples of command line usage 
\section{Интерфейс командной строки}
\mode<all>{\input{../../slides/cmdline/clui_vs_gui.tex}}

\section{Получение помощи}
\mode<all>{\begin{frame}[fragile]{From FAQ How To Ask Questions The Smart Way}
Before You Ask try to find an answer
  \begin{itemize}
	  \item by reading (RTFM): manual, FAQ, archives of the forum, by searching the Web;
	  \item by inspection or experimentation;
	  \item by asking a skilled friend;
	  \item by reading the source code;
    \end{itemize}
\end{frame}


\begin{frame}[fragile]{Встроенная документация}
\begin{itemize}
    \item \textbf{man} - помощь по внешним командам
    \pause
    \item \textbf{info} - расширенная помощь по некоторым командам (texinfo format)
    \pause
    \item \textbf{find /usr/share/doc/} - файлы документации поставляемые вместе с приложением (примеры конфигов, README)
    \item \textbf{-h, --help option} - встроенная в приложение справка
    \item \textbf{help} - встроенная помощь по внутренним командам bash (также man bash)
\end{itemize}
\end{frame}

\begin{frame}[fragile]{Основное о man}
      \textbf{man \textless command\_name\textgreater }

	\begin{block}{Example: show uptime manual page}
		{\tt man man}
	\end{block}

		\begin{itemize}
			\item Прочитайте {\tt man man} !
		\end{itemize}

\end{frame}

\begin{frame}[fragile]{man page navigation}
		\begin{itemize}
			\item \textbf{up, down} - scroll one line
			\item \textbf{q} - exit
			\item \textbf{/pattern} - search pattern
			\item \textbf{n} - next text pattern
			\item \textbf{N} - repeat search in back direction
			\item \textbf{h} - help
		\end{itemize}
\end{frame}

\begin{frame}[fragile]{Page structure}
		\begin{itemize}
			\item NAME
			\item SYNOPSIS
			\item DESCRIPTION
			\item EXAMPLES
			\item SEE ALSO
		\end{itemize}
\end{frame}

\begin{frame}[fragile]{Разделы помощи}
	\begin{itemize}
		\item[1] Основная секция(юзерские программы)
		\item[2] Syscalls
		\item[3] С library
		\item[5] Конфигурационные файлы
		\item[8] Sysadmih commands
	\end{itemize}
\end{frame}

\begin{frame}[fragile]{More than one section of the manual}
	name(section)  \\ 
	\textbf{man(1)} and \textbf{man(7)}, or \textbf{exit(2)} and \textbf{exit(3)} \\
     \begin{block}{Example: show manual in section 5 and 1}
        \begin{lstlisting}
man -f passwd #or whatis passwd 
man 5 passwd; man 1 passwd; man -wa passwd
        \end{lstlisting}
    \end{block}
\end{frame}

\begin{frame}[fragile]{Поиск по страницам помощи}
     \begin{block}{Упражнение. Поиск страниц с ключевым словом.}
        \begin{lstlisting}
man -f passwd #or whatis passwd 
man -k passwd #or apropos passwd 
whatis  -l -w '*'
man -s 3 -Kw passwd
        \end{lstlisting}
    \end{block}
\end{frame}


%\begin{frame}[fragile]{Чему научились}
%  \begin{itemize}
%  \item Как спрашивать у сообщества
%  \item Умеем использовать 3 источника получения информации man, info, help
%  \item Как перемещаться по страницам помощи info и man
%  \item Иcкать в системе помощи man и запрашивать из одного из 8-ми разделов 
%  \end{itemize}
%\end{frame}
}

\section{Интерактивная работа в командной оболочке (shell)}
\mode<all>{\begin{frame}{Делаем жизнь в консоли приятной}
  \begin{itemize}
	\item Меньше ошибок
	\item Легче вспомнить 
	\item Быстрее набрать и отредактировать
  \end{itemize}
\end{frame}

\begin{frame}{Автодополненение}
  Клавиша \textbf{Tab} 
  \begin{itemize}
	\item напоминалка имени команды и аргументов;
	\item файловый менеджер;
	\item ускоритель ввода.
  \end{itemize}
\end{frame}

\begin{frame}{История команд}
  Просмотреть историю \alert{{\tt history}}
  \begin{itemize}
	\item ближняя
	      \begin{itemize}
		\item Клавиши курсора, \alert{Ctrl-P}, \alert{Ctrl-N} -- навигация по истории
	      \end{itemize}
	\item дальняя 
	      \begin{itemize}
		\item \alert{\$ !10}  -- по номеру команды
		\item \alert{Ctrl-R} -- поиск в истории по фрагменту
	      \end{itemize}
	\item часть команды 
	      \begin{itemize}
		\item \alert{Alt-.}  -- последний аргумент предыдущей команды
	      \end{itemize}
  \end{itemize}
\end{frame}

\begin{frame}{Работаем с блоками текста}
    Удаление
      \begin{itemize}
        \item \alert{Ctrl-W}, \alert{Alt-Backspace} удалить слово, назад
        \item \alert{Alt-D}, удалить слово под курсором 
      \end{itemize}
    Перемещение
      \begin{itemize}
        \item \alert{Alt-F} слово вправо
        \item \alert{Alt-B} слово влево
        \item \alert{Ctrl-A} перейти в начало 
        \item \alert{Ctrl-E} перейти в конец
      \end{itemize}
\end{frame}

\begin{frame}{Управление терминалом}
      \begin{itemize}
        \item Ctrl-S -- заморозка терминала
        \item Ctrl-Q -- разморозка терминала
        \item \alert{Shift-PgUp}, \alert{Shift-PgDown} -- прокрутка терминала
        \item \alert{Ctrl-L} -- очистка терминала
      \end{itemize}
\end{frame}


\begin{frame}{Alias}
  \begin{block}{Bash alias}
    Alias в Bash -- это не более, чем клавиатурное сокращение или своего рода аббревиатура, 
    позволяющая сократить количество нажимаемых клавиш для ввода длинных команд.

    \begin{itemize}
        \item {\tt alias} -- просмотр сокращений
	\item {\tt alias <name>=''cmd1;cmd2''} -- добавление/модификация сокращений \\
	      {\tt alias tls=''netstat -lpnt | grep 127.0.0.1''}
        \item {\tt unalias <cmd>} -- удаление сокращений
    \end{itemize}
  \end{block}

  \pause
  \begin{block}{Практическое задание}
  Создать сокращение для команды {\tt ls} так, чтобы она всегда вызывалась с параметрами {\tt -l} и {\tt -a}.
  \end{block}

\end{frame}
}

\section{Навигация по файловой системе}
\mode<all>{\begin{frame}{Файловая система.}
\includegraphics[height=4cm]{hw_hdd.jpg} 
\includegraphics[height=4cm]{hw_usb_stick.jpg} 
\end{frame}

\begin{frame}{Древовидная структура хранения файлов.}
\includegraphics[height=4cm]{unix_tree} 
  \begin{itemize}
    \item В UNIX (и Linux) файлы организованы в виде \emph{единой древовидной структуры} (дерева), называемой \alert{файловой системой}.
    \item Корнем дерева является \alert{корневой каталог} (root directory), имеющий имя \alert{"/"}.
    \item Каждый файл имеет \alert{имя}, определяющее его расположение в дереве FS.
  \end{itemize}
\end{frame}


\begin{frame}[fragile]{Особенности именования файлов.}

                \begin{itemize}
                    \item регистр имеет значение \alert{fileA} и \alert{Filea} - два разных имени
                    \item один файл может иметь \alert{несколько} имен (hardlink) 
                    \item \alert{/} разделяет директории. Не получиться присвоить.
                    \item \alert{- \textbackslash <Space>} специальные символы 
                    \item \alert{.}file - скрытый файл. По умолчанию пропускаются.
                    \item максимальная длина имени \alert{255} символов
                \end{itemize}
\end{frame}

\begin{frame}[fragile]{Хранение файлов группами.}
      \begin{block}{Directory.}
Directory is a file system cataloging structure which contains references to other computer \alert{files}, and possibly other \alert{directories}
      \end{block}
 Для каждой запущенной программы в системе определена \alert{текущая директория} (working directory or current working directory) 
  \begin{itemize}
    \item \alert{относительный путь} - от текущей директории \newline
      Примеры: ../user10/.bashrc; ./script; ls script
    \item \alert{полный путь} начинается с \alert{/}(корневой директории), директории разделяются символом \alert{/}. \newline
  \end{itemize}
\end{frame}

\begin{frame}{Примеры использования имен.}
        /home/user/.ssh/authorized\_key
        \alert{./} - текущая директория
        \alert{../} - предыдущая директория
        \alert{../../} - две предыдущих директории
\end{frame}

\begin{frame}{Абсолютное имя файла в дереве файловой системы.}
\includegraphics[height=8cm]{filesystem} 
\end{frame}

\begin{frame}[fragile]{Перемещение по файловой системе}
      \begin{itemize}
		  \item {\tt pwd} -- имя текущей директории (help pwd)
		  \item {\tt ls} -- список файлов в директории. По умолчанию в текущей (man ls)
		  \item {\tt cd} -- смена текущей директории (help cd)
      \end{itemize}
      \begin{block}{Упражнение. Заходим в /usr/bin/ и просматриваем список доступных команд.}
	\begin{lstlisting}
        pwd
        cd /usr/bin/
        pwd
        ls
        cd -
        pwd
\end{lstlisting}
      \end{block}
\end{frame}

%\begin{frame}[fragile]{Просмотр типов файлов}
%      \begin{block}{Упражнение. Символьное обозначение типа файла.}
%Первая буква вывода команды ls -l обозначает тип файла. 
%Tip. Команды file позволяет определить тип файла. Ключ -d для работы с
%директорией. 
%
%/dev/zero
%
%/dev/sda 
%
%.. 
%
%/bin/sh 
%
%/dev/log
%
%/dev/stdout
%      \end{block}
%\end{frame}

\begin{frame}{Типы файлов в Unix}
  Для пользователя:
  \begin{itemize}
    \item \alert{Обычные файлы (regular file)}: .bashrc, /bin/bash
    \item \alert{Каталоги (directory)}: /home/user1, /usr, /, /usr/local  \pause
    \item \alert{Символические ссылки (symbolic links)}: /bin/sh, /dev/stdout
  \end{itemize} \pause
  Для администратора:
  \begin{itemize}
    \item \alert{Файлы устройств (device special file)}:
      \begin{itemize}
        \item \alert{блочные}: /dev/sda5, /dev/loop0, /dev/sr0
        \item \alert{символьные}: /dev/null, /dev/mem, /dev/tty
      \end{itemize}
  \end{itemize} \pause
  Для программиста:
  \begin{itemize}
    \item \alert{FIFO (named pipe)}: /dev/xconsole
    \item \alert{Socket}: /dev/log
  \end{itemize}

\end{frame}

\begin{frame}[fragile]{Создание (текстовых) файлов}
  \pause
  \begin{enumerate}
    \item Любимым редактором: \alert{vi/vim}, \alert{nano}, \alert{mcedit} \pause
    \item \alert{echo}
\begin{lstlisting}
echo какой-то текст >file
\end{lstlisting}\pause
    \item \alert{cat} с перенаправлением\footnote{До ``конца ввода'', т.е. нажатия Ctrl+D}
\lstinputlisting[basicstyle=\small]{../../sam-solutions/samples/cat-input-file} \pause
    \item \alert{touch} - создать пустой файл
\begin{lstlisting}
~$ touch file4
~$
\end{lstlisting}
  \end{enumerate}
\end{frame}

\begin{frame}[fragile]{Команды для работы с файлами}
Any ideas what is purpose of this commands? 
	\begin{itemize}
		\begin{columns}
		\column{0.2\textwidth}
			\item cp
			\item mv
			\item rm
			\item file
		\column{0.2\textwidth}
			\item touch
			\item ln
			\item mkdir
			\item mknod
			\item mkfifo
		\end{columns}
	\end{itemize}
\end{frame}


\begin{frame}[fragile]{Операции над каталогами (и файлами)}
  \begin{itemize}
    \item \alert{mkdir} - создать каталог
\begin{lstlisting}
~$ mkdir dir1 /tmp/somedir
~$ mkdir -p dir/and/existant/parts/in/path
\end{lstlisting} \pause
    \item \alert{rmdir} - удалить (пустой) каталог
\begin{lstlisting}
~$ rmdir dir1 /tmp/somedir
~$ rmdir -p deep/empty/dir/structure/
\end{lstlisting} \pause
    \item \alert{cp} - копирование файлов\footnotemark[8]
    \item \alert{mv} - перемещение и переименование файлов
    \item \alert{rm} - удаление файлов\footnotemark[17]
\begin{lstlisting}
~$ rm -rf dir1 /tmp/somedir
~$ cp /etc/passwd /tmp/passwd # copy file to file 
~$ cp -r /etc/  /tmp/ # copy directory to directory
\end{lstlisting} \pause
  \end{itemize}
\footnotetext[8]{с ключом \alert{-r}) и каталогов}
\end{frame}
}
\mode<all>{\begin{frame}{Дерево файловой системы - простое}
\includegraphics[height=8cm]{filesystem} 
\end{frame}

\begin{frame}{Детали реализации}
  \begin{itemize}
    \item \alert{VFS - virtual file system} - файлы и каталоги отображаются в единое дерево, независимо от их физического расположения.
  \end{itemize}
  \includegraphics[height=3.5cm]{vfs-and-devices}
\end{frame}

\begin{frame}{Монтирование}
  \begin{itemize}
    \item \alert{Монтирование} - процесс отображения содержимого устройства в указанную папку файловой системы.
    \item Команды:
      \begin{itemize}
        \item монтировать - ( \alert{mount} ) 
        \item размонтировать ( \alert{umount} )
      \end{itemize}
    \item \alert{mount} без параметров - вывести список уже подключенных файловых систем
  \end{itemize}
      \begin{block}{Упражнение. Дерево монтирования.}
     Получить вывод смонтированных блочных устройств в виде дерева с помощью команды: \alert{findmnt}
      \end{block}
  
\end{frame}

}
% globbing
% activity  ask how to copy files that start from letter a to /tmp dir?
% add example with 1000 files
\mode<all>{\begin{frame}[fragile]
  \frametitle{Подстановочные символы путей (globbing)}

  \alert{Wildcard characters} - спецсимволы в параметрах команд, раскрываемые в путь и имя файла самим интерпретатором перед тем, как запустить команду на выполнение. \pause


  \begin{itemize}
    \item \alert{*} - любое количество любых символов
\begin{lstlisting}[basicstyle=\normalsize]
        echo *
        ls /u*
        ls /sys/*/net/
\end{lstlisting} \pause
    \item \alert{[]} - символ из перечисления\footnote{об интервалах - в разделе о регулярных выражениях}
\begin{lstlisting}[basicstyle=\normalsize]
        echo .[bp]*
\end{lstlisting} \pause
    \item \alert{?} - любой одиночный символ
\begin{lstlisting}[basicstyle=\normalsize]
        echo ?i*
\end{lstlisting} 
  \end{itemize}

\end{frame}

}
\mode<all>{\begin{frame}[fragile]{Сжатие одного файла}

	\begin{block}{Команды: gzip, bzip, xz}
		\begin{itemize}
			\item {\tt -[1-9]} -- изменить уровень сжатия
			\item {\tt -d} -- распаковать
			\item {\tt -c} -- вывод на консоль
		\end{itemize}
	\end{block}

	\begin{block}{Примеры: эффективность сжатия}
		\begin{verbatim}
dd if=/dev/sda of=./big_file.01 bs=1M count=10
cp ./big_file.01 ./big_file.02
gzip ./big_file.01 ; xz ./big_file.02
du -h ./big_file.*
dd if=/dev/sda bs=1M count=1 | gzip -c > backup.gz
    \end{verbatim}
	\end{block}
\end{frame}

\begin{frame}[fragile]{Архивация файлов}
	\begin{block}{Команда: tar}
		\begin{itemize}
			\item {\tt -c} -- создать архив
			\item {\tt -x} -- извлечь из архива
				\begin{itemize}
					\item {\tt -C} -- перейти в директорию
				\end{itemize}
			\item {\tt -f} -- запись в файл
			\item {\tt -x -j -J} -- сжатие gzip, bzip, xz
		\end{itemize}
	\end{block}
	Создать архив:
	\begin{verbatim}
tar -czf /tmp/etc.tar /etc/
        \end{verbatim}
\end{frame}

\begin{frame}[fragile]{Архивация: примеры}
	Создать сжатый архив:
	\begin{verbatim}
tar -czf archive.tar.gz ./*
        \end{verbatim}
	\pause
	Распаковать сжатый архив в директорию {\tt /tmp}:
	\begin{verbatim}
tar -C /tmp/ -xzf archive.tar.gz
        \end{verbatim}
	\pause
	Просмотреть список файлов в архиве:
	\begin{verbatim}
tar -tzf archive.tar.gz
        \end{verbatim}
	%\pause
	%Создать копию текущей директории на другом хосте:
	%\begin{verbatim}
%HostDest: netcat -l 2222 | gzip -dc | tar -C /tmp/copy/ -x
%HostSrc:  tar -c * | gzip -9 | netcat HostDest 2222
%        \end{verbatim}
\end{frame}
}
\end{document}
\bye
