% xetex compatible variant that support TTF fonts according to company rules
\documentclass[ignorenonframetext, professionalfonts, hyperref={unicode}]{beamer}

\usetheme{Epam}

\usepackage{fontspec}
%\setsansfont{SourceSansPro-Regular}
%\setbeamerfont{frametitle}{family=\fontspec{Oswald}}
%\setbeamerfont{block title}{family=\fontspec{Oswald}}

%\setsansfont{Arial}
%setbeamerfont{frametitle}{family=\fontspec{Arial}}
%setbeamerfont{block title}{family=\fontspec{Arial}}

%\setmainfont{Times New Roman}
%\setromanfont{Trebuchet MS}
\defaultfontfeatures{Mapping=tex-text}
\defaultfontfeatures{Ligatures=TeX}


\usepackage{cmap}
\usepackage{graphicx}

\usepackage{textcomp}

\usepackage{beamerthemesplit}

\usepackage{ulem}

\usepackage{verbatim}
\usepackage{import}

\usepackage{listings}
\lstloadlanguages{bash}

\lstset{escapechar=`,
	captionpos=b,
	extendedchars=false,
	language=sh,
%	frame=single,
	tabsize=2, 
	columns=fullflexible, 
%	basicstyle=\scriptsize,
	keywordstyle=\color{blue}, 
	commentstyle=\itshape\color{brown},
%	identifierstyle=\ttfamily, 
	stringstyle=\mdseries\color{green}, 
	showstringspaces=false, 
	numbers=left, 
	numberstyle=\footnotesize, 
	breaklines=true, 
	inputencoding=utf8,
	keepspaces=true,
	morekeywords={u\_short, u\_char, u\_long, in\_addr}
	}

\definecolor{darkgreen}{cmyk}{0.7, 0, 1, 0.5}

\lstdefinelanguage{diff}
{
    morekeywords={+, -},
    sensitive=false,
    morecomment=[l]{//},
    morecomment=[s]{/*}{*/},
    morecomment=[l][\color{darkgreen}]{+},
    morecomment=[l][\color{red}]{-},
    morestring=[b]",
}

\author[Epam]{{\bf Epam}\\Low Level Programming Department}

%\institution[EPAM]{EPAM}
%\logo{\includegraphics[width=1cm]{logo.png}}

\graphicspath{{../../slides/cmdline/clipart/}{../../slides/bash/clipart/}}

\bibliographystyle{unsrt}
\setbeamertemplate{bibliography item}{\insertbiblabel}

\AtBeginSection[]{%
  \begin{frame}<beamer>
    \frametitle{}
    \tableofcontents[
        sectionstyle=show/shaded, hideallsubsections ]
  \end{frame}
  \addtocounter{framenumber}{-1}% If you don't want them to affect the slide number
}

% \regex for regular expressions
\newcommand{\regex}[1]{ %
\expandafter{$\ulcorner{\color{blue}\texttt{#1}}\lrcorner$} %
}



\title{Введение в GNU/Linux}


%%%%%%%%%%%%%%%%%%%%%%%%%%%%%%%%%%%%%%%%%%%%%%%%%
%%%%%%%%%% Begin Document  %%%%%%%%%%%%%%%%%%%%%%
%%%%%%%%%%%%%%%%%%%%%%%%%%%%%%%%%%%%%%%%%%%%%%%%%




\begin{document}

\begin{frame}
	\frametitle{}
	\titlepage
	\vspace{-0.5cm}
	\begin{center}
	%\frontpagelogo
	\end{center}
\end{frame}


%%%%%%%%%%%%%%%%%%%%%%%%%%%%%%%%%%%%%%%%%   
%%%%%%%%%% Content starts here %%%%%%%%%%
%%%%%%%%%%%%%%%%%%%%%%%%%%%%%%%%%%%%%%%%%


\section{Система управления пакетами}
\mode<all>{\input{../../slides/packaging/intro}}
\mode<all>{\begin{frame}{Система управления пакетами: для чего это нужно}
\begin{itemize}
 \item Dependency hell
 \item Общие задачи пакетного менеджера:
   \begin{itemize}
     \item Проверка целостности пакетов
     \item Проверка зависимостей пакетов
        \item Поддержание списка установленных пакетов
        \item Автоматическое удаление пакетов
     \item Предоставление доступа к репозиторию пакетов
     \item Разрешение зависимостей
   \end{itemize}
\end{itemize}
\end{frame}


\begin{frame}{RPM: структура пакета}
	\begin{itemize}
		\item Метаданные
			\begin{itemize}
				\item Имя
				\item Версия/Релиз
				\item Группа
				\item Описание
                                \item Зависимости
				\item ...
			\end{itemize}
		\item Архив с файлами
			\begin{itemize}
				\item cpio
			\end{itemize}
		\item Скрипты
			\begin{itemize}
				\item Pre Install
				\item Post Install
				\item Pre Uninstall
				\item Post Uninstall \bigskip
				\item Triggers
			\end{itemize}
	\end{itemize}
\end{frame}

\begin{frame}{Два уровня пакетных менеджеров}

\begin{tabular}{| l | c | r |}
      \hline
          Level &  RedHat-based & Debian-based \\ 
      \hline
          {\bf Low} & rpm & dpkg \\ 
      \hline
          {\bf High} & yum, dnf & apt, aptitude \\
      \hline
    \end{tabular}
    \\


    \alert{Низкоуровневые} используются для установки, удаления, получения информации о пакете. \\
    \alert{Высокоуровневые} предоставляют дополнительные функции такие как поиск по репозиторию, копирование пакета из репозитория, разрешение зависимостей, обновление системы.

\end{frame}
}
\mode<all>{\begin{frame}{RPM: команды}
	\begin{block}{Установка пакета}
		{\tt rpm -i [rpm-file1] ... [[url://]rpm-fileN] }
	\end{block}
	\begin{block}{Удаление пакета}
		{\tt rpm -e pkgname1 ... pkgnameN }
	\end{block}
	\begin{block}{Обновление пакета}
		{\tt rpm -U [rpm-file1] ... [[url://]rpm-fileN] }
	\end{block}
	\begin{block}{Проверка пакета}
		{\tt rpm -V pkgname1 ... pkgnameN }
	\end{block}
\end{frame}

\begin{frame}{RPM -q: часто используемые опции опроса}

	\begin{itemize}
		\item {\tt pkgname} -- выбор пакета, установленного в системе
		\item {\tt -a} -- все пакеты, установленные в системе
		\item {\tt -p} -- использовать файл RPM
	\end{itemize}


	\begin{itemize}
		\item {\tt -i} -- показать информацию пакета\\
			{\tt rpm \alert{-q} \alert{-i} glibc }
		\item {\tt -l} -- показать список файлов пакета \\
			{\tt rpm \alert{-q -l} glibc }
		\item {\tt -{}-whatprovides} -- \\
			{\tt rpm \alert{-q --whatprovides} java}
		\item {\tt -{}-whatrequires} -- \\
			{\tt rpm \alert{-q --whatrequires} /bin/bash}
		\item {\tt -{}-queryformat} -- формат вывода\\
			{\tt rpm \alert{-q -{}-whatrequires} /bin/bash \alert{-{}-queryformat ''\%\{name\} ''} }

	\end{itemize}

\end{frame}


\begin{frame}{Команды пакетных менеджеров}
        \begin{tabular}{ll}
            \multicolumn{2}{c}{Установка пакета }   \tabularnewline
            Debian & {\tt apt-get \alert{install} pkgname } \\
            CentOS & {\tt yum \alert{install} pkgname } \\
            \multicolumn{2}{c}{Обновление пакета }  \tabularnewline
            Debian & {\tt apt-get \alert{install} pkgname } \\
            CentOS & {\tt yum \alert{update} pkgname }  \\
            \multicolumn{2}{c}{Удаление пакета }   \tabularnewline
            Debian & {\tt apt-get \alert{remove} pkgname } \\ 
            CentOS & {\tt yum \alert{remove} pkgname }  \\
            \multicolumn{2}{c}{Поиск. По имени пакета}   \tabularnewline
            Debian & {\tt apt-cache \alert{search} pkgname } \\
            CentOS & {\tt yum \alert{list} pkgname }  \\
            \multicolumn{2}{c}{Поиск. По имени файла}   \tabularnewline
            Debian & {\tt apt-file \alert{search} path } \\
            CentOS & {\tt yum \alert{provides} file} 
        \end{tabular}
\end{frame}
}
\mode<all>{%\newcounter{tmpc}

\begin{frame}{Репозиторий}
	\begin{block}{Репозиторий пакетов}
		Место, где хранятся и поддерживаются пакеты, а также сопутствующая мета-информация, предназначенное для использования пакетным менеджером.
	\end{block}
	\begin{block}{Пример: Fedora Core}
		\begin{itemize}
			\item Packages/*.rpm
			\item RPM-GPG-KEY-*
			\item repodata
			\begin{itemize}
				\item множество сжатых и несжатых XML файлов для YUM
			\end{itemize}
		\end{itemize}

		Описание репозтория для YUM на локальной системе хранится по пути
		{\tt /etc/yum.repos.d/*.repo}
	\end{block}
		
\end{frame}



%\begin{frame}[fragile]{Упражнение}
%  \begin{enumerate}
%      \item Создать на {\tt /dev/sda} раздел размером примерно 10Gb
%      \item Создать на этом разделе ext3 ФС и смонтировать раздел в {\tt /mnt/chroot}
%      \item Развернуть {\tt /media/nfs/pub/CentOS/precreated/centOS.tar.gz} в {\tt /mnt/chroot}
%      \item Смонтировать {\tt proc, sysfs} а также {\tt /dev} в соответствующие места {\tt /mnt/chroot}
%      \item {\tt chroot /mnt/chroot}
%      \item Отредактировать {\tt /etc/resolv.conf} -- скопировать туда информацию из {\tt resolv.conf} основной системы
%      \item Отредактировать {\tt /etc/yum.conf} Добавить следующий раздел
%\begin{minipage}{0.5\textwidth}
%\begin{verbatim}
%[base]
%  name = CentOS 6
%  baseurl = ftp://192.168.11.15/CentOS
%  gpgcheck = 0
%\end{verbatim}
%\end{minipage}
%\setcounter{tmpc}{\theenumi}
%\end{enumerate}
%\end{frame}
%\begin{frame}{Продолжение упражнения}
%  \begin{enumerate}
%      %\setcounter{enumi}{\thetmpc}
%      \item Обновить информацию о пакетах {\tt update}
%      \item Удалить пакет vim
%      \item Установить заново пакет vim
%      \item Посмотреть списки файлов для пакетов {\tt rpm, vim}
%      \item Найти, к какому пакету относится команда {\tt ls, top}
%      \item Найти пакет предоставляющий сервис ssh и установить его
%    \end{enumerate}
%\end{frame}
}

\section{Find, extract and edit text data}
\subsection{find и xargs}
\mode<all>{\begin{frame}[fragile]{Поиск файлов}
    \textbf{find} ищет файлы в заданной директории и производит над ним заданную операцию.
	\begin{block}{Параметры поиска}
		\begin{itemize}
			\item {\tt -name}, {\tt -iname} -- имя файлового объекта, включая метасимволы 
			\item {\tt -type} -- тип файлового объекта
			\item {\tt -size} -- размер [cwbkMG]
			\item {\tt -perm} -- права доступа
			\item {\tt -user} -- владелец
			\item {\tt ...} -- другие опции man find 
		\end{itemize}
	\end{block}
	\begin{block}{Действия над результом поиска}
		\begin{itemize}
			\item {\tt -print} -- вывод на stdout (по умолчанию)
			\item {\tt -printf} -- форматированный вывод
			\item {\tt -exec} -- выполнить команду
			\item {\tt -ls} -- замена -exec ls -l \{\} ;
			\item {\tt -delete} -- удалить файл
		\end{itemize}
	\end{block}
\end{frame}

\begin{frame}[fragile]{Поиск файлов}
	\begin{block}{Примеры}
            \begin{verbatim} find . -name '*.o' -print \end{verbatim}
            \begin{verbatim} find -name '*.o' \end{verbatim}
            \begin{verbatim} find -type d -user altlinux \end{verbatim}
            \begin{verbatim} find /root \( -name '*.pyc' -o -name '*.py' \) \
-type f -user root -size +300k -size -1024k \
-exec ls -l \{\} \; \end{verbatim}
	\end{block}
 Дополнительно:
 Позволяет преодолеть лимит на кол-во аргументов в командной строке. 
 Например, когда в директории много файлов и тогда возникают ошибки
 \textquotedblleft Arguments too long.\textquotedblright 
\end{frame}

\begin{frame}[fragile]{xargs}
	\begin{block}{xargs}
			Утилита для создания и запуска команд из стандартного потока ввода:
		\begin{verbatim}
xargs [options] command [command options]
                \end{verbatim}
		\begin{itemize}
			\item {\tt -d} -- разделитель
			\item {\tt -0} -- null-terminated строки
			\item {\tt -I text} -- подстановка
			\item {\tt -n N} -- максимальное количество аргументов
			\item {\tt -P N} -- максимальное количество процессов
		\end{itemize}

	\end{block}
Имя файла может содержать разделители: пробелы, tab, символ новой строки. В этом случае xargs воспримет имя как набор раздельных аргументов. Использовать -print0 в команде find для замены на ASCII NUL в имени файла.
\end{frame}

\begin{frame}[fragile]{xargs}
	\begin{block}{Примеры}

		\begin{verbatim}
file /bin/*  | grep shell | cut -f 1 -d ':' | xargs wc -l 
# calculate number of strings in all shell scripts
                \end{verbatim}

		\begin{verbatim}
find /etc -type f -size -100k | \
 xargs tar -czf /tmp/archive-100k.tar.gz
                \end{verbatim}

		\begin{verbatim}
find /etc -type f | xargs -I {} echo "Найден {} файл"
                \end{verbatim}

		\begin{verbatim}
find . -type f -name "*.mp3" -print0 | \
 xargs -0 -n 1 -P 0 -I mp3 avconv -i mp3 mp3.ogg
                \end{verbatim}
	
	\end{block}
\end{frame}
}
\subsection{Extract data}
\mode<all>{\input{../../slides/cmdline/commands1.tex}}
\subsection{Редакторы}
\mode<all>{\input{../../slides/cmdline/editors-intro}}

\section{Multiuser UNIX}
\mode<all>{\input{../../slides/multiuser/multiuser-model.tex}}
\mode<all>{\input{../../slides/multiuser/fs-permissions.tex}}

\section{User management}
\mode<all>{\input{../../slides/multiuser/account_files.tex}}
\mode<all>{\input{../../slides/multiuser/pam.tex}}

\section{Configure network settings}
\mode<all>{\input{../../slides/networking/intro}}

\subsection{Управление интерфейсами}
\mode<all>{\input{../../slides/networking/interface-management}}

\subsection{Полезные программы}
\mode<all>{\begin{frame}{Network diagnostic utilities.}
  \begin{tabular}{ | c | c | }
    \hline
    TCP/IP Layer & Utility \\ \hline 
    Data link & arp / ip ne / ip link / ip addr \\ \hline
    Network & ping / mtr / traceroute \\ \hline
    Transport & netstat / ss \\ \hline
    Application & telnet / netcat / nc /  \\ \hline
    Application DNS & host / dig / nslookup  \\ \hline
  \end{tabular}
    \\
   Multipurpose tools:
		\begin{itemize}
			\item Sniffers: tcpdump, wireshark
			\item Scanner: nmap
		\end{itemize}

\end{frame}


%\begin{frame}{Полезные утилиты: практика}
%
%	\begin{columns}
%		\column{0.5\textwidth}
%		\begin{block}{netstat}
%
%			Узнать:
%			\begin{itemize}
%				\item список используемых сокетов
%				\item серверных сокетов
%				\item имена/pid серверов
%				\item узнать номера портов
%			\end{itemize}
%		\end{block}
%	
%		\pause
%		\column{0.5\textwidth}
%		\begin{block}{telnet/netcat}
%
%			\begin{itemize}
%				\item Чат по протоколу TCP с соседом
%				\item Чат по протоколу UDP с соседом
%				\item Передать текстовый и бинарный файлы
%			\end{itemize}
%	
%			При создании чата использовать {\tt netstat} и {\tt tcpdump}
%			для получения информации о соединении.
%		\end{block}
%	
%	\end{columns}
%\end{frame}
%
%nmap
%1. сканирование соседа
%2. сканирование выделенных портов у соседа (поиск сервера чата) 
%3. узнать список открытых портов на всех машинах в 505
%4. узнать список  работающих машин
%
%tcpdump
%0. pcap файлы/libpcap
%1. запуск монитора
%2. запуск чата
%3. монитор-фильтр-анализ
%
}

\section{Маршрутизация}
\mode<all>{\input{../../slides/networking/routing}}

\section{iptables}
\mode<all>{\input{../../slides/networking/iptables}}

\end{document}
