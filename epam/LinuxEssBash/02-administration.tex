% xetex compatible variant that support TTF fonts according to company rules
\documentclass[ignorenonframetext, professionalfonts, hyperref={unicode}]{beamer}

\usetheme{Epam}

\usepackage{fontspec}
\setsansfont{SourceSansPro-Regular}
%\setbeamerfont{frametitle}{family=\fontspec{Oswald}}
\setbeamerfont{frametitle}{family=\fontspec{Oswald}}
\setbeamerfont{block title}{family=\fontspec{Oswald}}

%\setmainfont{Times New Roman}
\defaultfontfeatures{Mapping=tex-text}
\defaultfontfeatures{Ligatures=TeX}

%\setsansfont{Arial}
%\setromanfont{Trebuchet MS}

\usepackage{cmap}
\usepackage{graphicx}

\usepackage{textcomp}

\usepackage{beamerthemesplit}

\usepackage{ulem}

\usepackage{verbatim}
\usepackage{import}

\usepackage{listings}
\lstloadlanguages{bash}

\lstset{escapechar=`,
	captionpos=b,
	extendedchars=false,
	language=sh,
%	frame=single,
	tabsize=2, 
	columns=fullflexible, 
%	basicstyle=\scriptsize,
	keywordstyle=\color{blue}, 
	commentstyle=\itshape\color{brown},
%	identifierstyle=\ttfamily, 
	stringstyle=\mdseries\color{green}, 
	showstringspaces=false, 
	numbers=left, 
	numberstyle=\footnotesize, 
	breaklines=true, 
	inputencoding=utf8,
	keepspaces=true,
	morekeywords={u\_short, u\_char, u\_long, in\_addr}
	}

\definecolor{darkgreen}{cmyk}{0.7, 0, 1, 0.5}

\lstdefinelanguage{diff}
{
    morekeywords={+, -},
    sensitive=false,
    morecomment=[l]{//},
    morecomment=[s]{/*}{*/},
    morecomment=[l][\color{darkgreen}]{+},
    morecomment=[l][\color{red}]{-},
    morestring=[b]",
}

\author[Epam]{{\bf Epam}\\Low Level Programming Department}

%\institution[EPAM]{EPAM}
%\logo{\includegraphics[width=1cm]{logo.png}}

\graphicspath{{../../slides/cmdline/clipart/}{../../slides/bash/clipart/}}

\bibliographystyle{unsrt}
\setbeamertemplate{bibliography item}{\insertbiblabel}

\AtBeginSection[]{%
  \begin{frame}<beamer>
    \frametitle{}
    \tableofcontents[
        sectionstyle=show/shaded, hideallsubsections ]
  \end{frame}
  \addtocounter{framenumber}{-1}% If you don't want them to affect the slide number
}

% \regex for regular expressions
\newcommand{\regex}[1]{ %
\expandafter{$\ulcorner{\color{blue}\texttt{#1}}\lrcorner$} %
}



\title{Введение в GNU/Linux}


%%%%%%%%%%%%%%%%%%%%%%%%%%%%%%%%%%%%%%%%%%%%%%%%%
%%%%%%%%%% Begin Document  %%%%%%%%%%%%%%%%%%%%%%
%%%%%%%%%%%%%%%%%%%%%%%%%%%%%%%%%%%%%%%%%%%%%%%%%




\begin{document}

\begin{frame}
	\frametitle{}
	\titlepage
	\vspace{-0.5cm}
	\begin{center}
	%\frontpagelogo
	\end{center}
\end{frame}


%%%%%%%%%%%%%%%%%%%%%%%%%%%%%%%%%%%%%%%%%   
%%%%%%%%%% Content starts here %%%%%%%%%%
%%%%%%%%%%%%%%%%%%%%%%%%%%%%%%%%%%%%%%%%%

%\section{Connecting to remote systems with Secure Shell}
%\mode<all>{\begin{frame}{Secure shell}

	\begin{block}{ssh -- терминал}
		{\tt ssh [user@]host[:port]}\\
		{\tt ssh host [-l user] [-p port]}
		\begin{itemize}
			\item -v -- "разговорчивый" режим 
			\item -t -- насильное назначение псевдотерминала (для автоматизации)
		\end{itemize}
		Вся конфигурация пользователя: {\tt \$HOME/.ssh}
	\end{block}

	\pause

	\begin{block}{... и не только}
		\begin{itemize}
			\item -X -- "проброс" графики 
			\item -L [bindip:]port:rhost:rport -- "пробрасывание" порта с удаленной машины на локальную
			\item -R [bindip:]port:lhost:lport -- "пробрасывание" порта с локальной машины на удаленную
			\item -W host:port -- stdin/stdout с указанным хостом
			\item -D port -- динамический прокси
		\end{itemize}
	\end{block}
\end{frame}


}

\section{Find, extract and edit text data}
\subsection{find и xargs}
\mode<all>{\begin{frame}[fragile]{Поиск файлов командой find}
    \alert{find} ищет файлы в заданной директории и производит над ним заданную операцию.
	\begin{block}{Часто используемые параметры поиска}
		\begin{itemize}
			\item {\tt -name}, {\tt -iname} -- имя файлового объекта, включая метасимволы 
			\item {\tt -type} -- тип файлового объекта
			\item {\tt -size} -- размер [cwbkMG]
			\item {\tt -perm} -- права доступа
			\item {\tt -user} -- владелец
			\item {\tt ...} -- другие опции man find 
		\end{itemize}
	\end{block}
\end{frame}

\begin{frame}[fragile]{Файлы найдены}
	\begin{block}{Действия над результом поиска}
		\begin{itemize}
			\item {\tt -print} -- вывод на stdout (по умолчанию)
			\item {\tt -printf} -- форматированный вывод
			\item {\tt -exec} -- выполнить команду
			\item {\tt -ls} -- замена -exec ls -l \{\} ;
			\item {\tt -delete} -- удалить файл
		\end{itemize}
	\end{block}
\end{frame}

\begin{frame}[fragile]{Примеры использования команды find}
            В текущей директории найти все файлы *.o и вывести на экран 
            \begin{verbatim} find . -name '*.o' -print \end{verbatim}
            \begin{verbatim} find -name '*.o' \end{verbatim}
            Поск по типу и владельцу файла.
            \begin{verbatim} find -type d -user altlinux \end{verbatim}
            Составная команда, множество условий
            \begin{verbatim} find /root \( -name '*.pyc' -o -name '*.py' \) \
-type f -user root -size +300k -size -1024k \
-exec ls -l \{\} \; \end{verbatim}
 Дополнительно: позволяет преодолеть лимит на кол-во аргументов в командной строке. 
 \textquotedblleft Arguments too long.\textquotedblright 
\end{frame}

\begin{frame}[fragile]{xargs}
			Утилита для создания и запуска команд из стандартного потока ввода:
		\begin{verbatim}
xargs [options] command [command options]
                \end{verbatim}
		\begin{itemize}
			\item {\tt -d} -- разделитель
			\item {\tt -0} -- null-terminated строки
			\item {\tt -I text} -- подстановка
			\item {\tt -n N} -- максимальное количество аргументов
			\item {\tt -P N} -- максимальное количество процессов
		\end{itemize}

%Для работы с разделителями в имени файла: пробелы, tab, символ новой строки. 
Использовать -print0 в команде find для замены на ASCII NUL в имени файла.
\end{frame}

\begin{frame}[fragile]{xargs}
	\begin{block}{Примеры}
		\begin{verbatim}
file /bin/*  | grep shell | cut -f 1 -d ':' | xargs wc -l 
# calculate number of strings in all shell scripts
                \end{verbatim}
		\begin{verbatim}
find /etc -type f -size -100k | \
 xargs tar -czf /tmp/archive-100k.tar.gz
                \end{verbatim}
		\begin{verbatim}
find /etc -type f | xargs -I {} echo "Найден {} файл"
                \end{verbatim}

		\begin{verbatim}
find . -type f -name "*.mp3" -print0 | \
 xargs -0 -n 1 -P 0 -I mp3 avconv -i mp3 mp3.ogg
                \end{verbatim}
	
	\end{block}
\end{frame}
}
\subsection{Extract data}
\mode<all>{\begin{frame}{Дополнительный набор команд}
  \begin{itemize}
    \item {\tt cat} - Вывод файла в stdout, соединение нескольких файлов в stdout
    \item {\tt wc} - подсчет статистики символов в файле или в stdin 
    \item {\tt sort} - сортировка строк файла
    \item {\tt uniq} - объединение одинаковых строк в одну
    \item {\tt tr} - замена набора символов
    \item {\tt less} - программа-пейджер
    \item {\tt grep} - поиск строк, соответствующих регулярному выражению
    \item {\tt cut} - выделение полей из строк stdin
    \item {\tt awk} - небольшой язык программирования (также полезен для выделения полей)
  \end{itemize}
\end{frame}

\begin{frame}[fragile]{Некоторые примеры использования}
\begin{lstlisting}[language=bash]
cat /proc/1/environ | tr '\0' '\n' | less
ls  | wc -l # подсчет числа файлов
man uniq | tr  '[:space:]' '\n' | sort | uniq -c | sort -n | less # подсчет количества слов в тексте man uniq
history | wc -l # подсчет ранее введенных команд
cat /etc/udev/rules.d/* | wc -l
ls -s *.jpg | awk 'BEGIN{s=0};/^[ ]*[0-9]/{s+=`\$1`};END{print s}' 
\end{lstlisting}
  \pause
  \begin{block}{Упражнение}
    Посчитать статистику использования команд в history
  \end{block}
\end{frame}

\begin{frame}{Дополнительный набор команд для работы с текстом}
	\begin{itemize}
	  \item {\tt head} -- вывести первые строки
	  \item {\tt tail} -- вывести последние строки
		\begin{itemize}
			\item {\tt -f} -- отслеживать добавление данных в файл 
		\end{itemize}
	  \item {\tt tee} -- копировать стандартный вывод в файл
	  \item {\tt grep} -- печать текста, соответствующего шаблону
		\begin{itemize}
			\item {\tt -i}	
			\item {\tt -v}
			\item {\tt -o}
		\end{itemize}
	\end{itemize}
\end{frame}

}
\subsection{Редакторы}
\mode<all>{\begin{frame}{Текстовые редакторы}
	\begin{itemize}
		\item Интерактивные
			\begin{itemize}
				\item vi
				\item vim
				\item emacs
			\end{itemize}
		\item Поточные
			\begin{itemize}
				\item {\tt ed}
				\item {\tt sed}
				\item {\tt awk}
			\end{itemize}
	\end{itemize}
\end{frame}

%%\begin{frame}[fragile]{Метасимволы}
%	\begin{block}{grep, sed, awk}
%	\end{block}
%	\begin{itemize}
%		\item {\tt .} -- любой символ за исключением пустой строки
%		\item {\tt *} -- любоe количество символов, которые стоят перед {\tt *}
%		\item {\tt \^{}} -- начало строки
%		\item {\tt \$} -- конец строки
%		\item {\tt [...]} -- любой символ из заключенных в скобки
%	\end{itemize}
%\end{frame}

\begin{frame}[fragile]{sed}
	\begin{block}{Сценарии}
		{\tt [ addr [ ,  addr ] ] cmd [ args ]}
	\end{block}

	\tiny
	\begin{block}{Команды}
		\begin{itemize}
		  \item {\tt d} -- удалить строку
			  \begin{verbatim} who | sed -e '2,4 d' \end{verbatim}
			  \begin{verbatim} who | sed -e '/pts/ d' \end{verbatim}
		  \item {\tt s} -- замена по регулярному выражению
			  \begin{verbatim} who | sed -e "s/USER/user/g" \end{verbatim}
		  \item {\tt a, i} -- добавить строку после (перед) текущей
			  \begin{verbatim} who | sed -e 'a Text' \end{verbatim}
		\end{itemize}
	\end{block}
%	\pause
%	\begin{block}{Задача}
%		С помощью {\tt find} найти все вложенные директории в {\tt /etc} и 
%		''переделать'' их в windows-style
%	\end{block}
\end{frame}
}

\section{Multiuser UNIX}
\mode<all>{\begin{frame}{Многопользовательская модель}   
 \begin{itemize}
   \item Linux -- многопользовательская система
   \item Привилегии пользователей
     \begin{itemize}
       \item root
       \item other users
      \end{itemize}
     \end{itemize}
\end{frame}

%\section{Механизмы разделения привилегий}
%\subsection{Классический UNIX}

\begin{frame}{Пользователи, группы и файлы}
\begin{itemize}
  \item Каждый пользователь принадлежит одной или нескольким \textbf{группам}
  \item Каждый файл и директория принадлежит
    \begin{itemize}
      \item Одному пользователю 
      \item Одной группе
    \end{itemize}
  \pause
  \item  Разрешения что либо делать с файлом определяются по отношению к
    \begin{enumerate}
      \item Пользователю-владельцу файла
      \item Группе владеющей файлом
      \item Всем остальным пользователям
    \end{enumerate}

\end{itemize}
\pause
\begin{columns}
  \column{0.48\textwidth}
  \begin{itemize}
    \item {\tt ls -l} 3,4 поле 
    \item {\tt groups}
   \end{itemize}
  \column{0.48\textwidth}
  \begin{block}{Попробовать}
    {\tt ls -l /usr/bin/}

    {\tt groups}

    {\tt groups root}
  \end{block}
\end{columns}
\end{frame}
}
\mode<all>{\begin{frame}{Типы разрешений для файлов}
	\begin{columns}
		\column{0.48\textwidth}
		\begin{center}
			\textbf{Разрешения для файла}
		\end{center}
		\begin{itemize}
			\item Три типа разрешений
				\begin{enumerate}
					\item чтение read(r)
					\item запись write(w)
					\item выполнение execute(x)
				\end{enumerate}
		\end{itemize}
		\column{0.48\textwidth}
		\begin{center}
			\textbf{Разрешения для директорий}
		\end{center}
		\begin{itemize}
			\item Три типа разрешений
				\begin{enumerate}
					\item поиск файлов в директории read(r) 
					\item добавление и удаление файлов write(w)
					\item заход в директорию execute(x)
				\end{enumerate}
		\end{itemize}
	\end{columns}

	\pause

	Попробовать {\tt ls -l /usr/bin}

	\pause

	Пересчет мнемонического разрешения в битовую маску 

	$r\to4, w\to2 , x\to1$ 

	rwxrw-r-x$\to$765
\end{frame}

\begin{frame}{Команды для управления пользователями и разрешениями файлов}
	\begin{columns}
		\column{0.48\textwidth}
		\begin{itemize}
			\item {\tt chown}
			\item {\tt chmod}
		\end{itemize}
		\column{0.48\textwidth}
		\begin{itemize}
			\item {\tt useradd, usermod, userdel}
			\item {\tt groupadd, groupmod, groupdel}
			\item {\tt su, sudo}
		\end{itemize}
	\end{columns}
\end{frame}

%\begin{frame}
%    \frametitle{}
%	\begin{block}{Упражнения}
%		\begin{enumerate}
%			\item Создать директорию без r разрешения но с x разрешением, внутри нее создать поддиректорию с rwx разрешениями (для пользователя \defaultuser)
%			\item Создать нового пользователя testuser.
%			\item Скопировать {\tt /bin/bash} (под именем mysh) в домашнюю директорию пользователя \defaultuser  и поставить r-x разрешение только для other
%			\item Попробовать выполнить скопированный файл от имени пользователя \defaultuser, затем от имени пользователя testuser
%       \end{enumerate}
%    \end{block}
%\end{frame}
%\begin{frame}
%    \frametitle{}
%	\begin{block}{Упражнения}
%		\begin{enumerate}
%			\item Создать новую группу testgroup
%			\item Изменить группу владеющую mysh на testgroup и сделать {\tt chmod 474 mysh}
%			\item Попробовать выполнить mysh от имени \defaultuser и root. 
%			\item Добавить пользователя \defaultuser в группу testgroup и попробовать выполнить mysh еще раз
%			\item Получить список групп которым принадлежат устройства в {\tt /dev}
%		\end{enumerate}
%	\end{block}
%\end{frame}

\begin{frame}{SUID программы}
	\begin{block}{Попробовать}
		{\tt id}

		{\tt ls -l `which su`}
	\end{block}
	\pause
	\begin{itemize}
		\item Некоторые программы должны выполняться от имени обычного пользователя, но иметь больше привилегий
		\item Для этого у них устанавливается suid или sgid биты
		\item Установка suid (например {\tt chmod 4710 <file>})
	\end{itemize}
	\pause
	\begin{block}{Упражнение}
		\begin{itemize}
			\item Под root создать копию утилиты {\tt id} (назвать, например, {\tt id2}) в директории /usr/bin/
			\item Установить suid бит для этой утилиты
			%\item Запустить {\tt id2} от имени пользователя defaultuser
			%\item тоже с sgid битом
		\end{itemize}
	\end{block}
\end{frame}

\begin{frame}{Опасности SUID}
	\begin{itemize}
		\item Возможность backdoor через suid программу
			\begin{itemize}
				\item Shell игнорирует effective uid
				\item Скрипты обычно тоже игнорируют
				\item nosuid mount option
			\end{itemize}
		\item Атака через buffer overflow в существующей suid программе
			\begin{itemize}
				\item не использовать strcpy, sprintf, ... в security critical
				\item А если все же не уследили
					\begin{itemize}
						\item рандомизация стека
						\item grsecurity
						\item частично selinux
					\end{itemize}
			\end{itemize}
	\end{itemize}
\end{frame}


\begin{frame}{SUID, SGID и sticky bit для директорий}
	\begin{itemize}
		\item sgid для директорий -- все поддиректории и файлы внутри имеют тот же group id
		\item suid -- игнорируется
		\item Sticky bit (\tt{chmod +t mydir})
          \begin{itemize}
            \item Файлы из обычной директории может удалять любой пользователь с правами на запись в \emph{директорию}
            \item Файлы из директории со sticky bit может удалять только владелец директории, владелец файла или root.
          \end{itemize} 
	\end{itemize}
\end{frame}

\begin{frame}[fragile]
 \frametitle{UMASK}

	\begin{block}{umask}
		маска режима создания пользовательских файлов
	\end{block}

	Права доступа файлов, вычисляются c помощью побитовых операций:
    \begin{itemize}
      \item библиотечный вызов \tt{fopen} создает файл с разрешениями 
     \verb+ 0666 & ~umask +
      \item Системный вызов \tt{open(pathname,flags,mode)} создает файл с разрешениями \verb+ mode & ~umask +
   \end{itemize}
        

\end{frame}



}

\section{User management}
\mode<all>{\input{../../slides/multiuser/account_files.tex}}
\mode<all>{\input{../../slides/multiuser/pam.tex}}

\section{Configure network settings}
\mode<all>{\begin{frame}{Сетевая подсистема Linux}

	\begin{block}{Cетевой интерфейс}

		Сетевой интерфейс в Linux -– это абстрактный \alert{именованный} объект,  используемый для передачи 
		данных через некоторую линию связи без привязки к ее (линии связи) реализации.
	\end{block}
\end{frame}

\begin{frame}{Сетевая подсистема Linux}

	\center\includegraphics[width=0.9\textwidth]{../../slides/networking/06-netstack.png}

\end{frame}


}

\subsection{Управление интерфейсами}
\mode<all>{\input{../../slides/networking/interface-management}}

\subsection{Полезные программы}
\mode<all>{\begin{frame}{Полезные утилиты}
	\begin{center}
		\begin{itemize}
			\item netstat / ss
			\item nslookup / dig
			\item ping
			\item traceroute
			\item tcpdump
			\item telnet
			\item netcat
			\item nmap
		\end{itemize}
	\end{center}

\end{frame}


%\begin{frame}{Полезные утилиты: практика}
%
%	\begin{columns}
%		\column{0.5\textwidth}
%		\begin{block}{netstat}
%
%			Узнать:
%			\begin{itemize}
%				\item список используемых сокетов
%				\item серверных сокетов
%				\item имена/pid серверов
%				\item узнать номера портов
%			\end{itemize}
%		\end{block}
%	
%		\pause
%		\column{0.5\textwidth}
%		\begin{block}{telnet/netcat}
%
%			\begin{itemize}
%				\item Чат по протоколу TCP с соседом
%				\item Чат по протоколу UDP с соседом
%				\item Передать текстовый и бинарный файлы
%			\end{itemize}
%	
%			При создании чата использовать {\tt netstat} и {\tt tcpdump}
%			для получения информации о соединении.
%		\end{block}
%	
%	\end{columns}
%\end{frame}
%
%nmap
%1. сканирование соседа
%2. сканирование выделенных портов у соседа (поиск сервера чата) 
%3. узнать список открытых портов на всех машинах в 505
%4. узнать список  работающих машин
%
%tcpdump
%0. pcap файлы/libpcap
%1. запуск монитора
%2. запуск чата
%3. монитор-фильтр-анализ
%
}


\section{Маршрутизация}
\mode<all>{\input{../../slides/networking/routing}}

\section{iptables}
\mode<all>{\input{../../slides/networking/iptables}}

\end{document}
