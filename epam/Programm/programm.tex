\documentclass[12pt,a4paper,oneside]{article}
\usepackage[left=0.5cm,right=0.5cm,top=2cm,bottom=2cm,bindingoffset=0cm]{geometry}

\usepackage[russian]{babel}
\usepackage[utf8]{inputenc}
\usepackage[T1]{fontenc}

\usepackage[usenames, dvipsnames]{color}
\usepackage[usenames, dvipsnames]{xcolor}

\usepackage{listingsutf8}
\lstloadlanguages{C, make, bash}

\lstset{escapechar=`,
	language=C, 
	tabsize=2, 
	columns=fullflexible, 
	basicstyle=\ttfamily, 
	keywordstyle=\color{blue}, 
	commentstyle=\itshape\color{brown},
%	identifierstyle=\ttfamily, 
	stringstyle=\mdseries\color{OliveGreen}, 
	showstringspaces=false, 
	numbers=left, 
	numberstyle=\tiny, 
	breaklines=true, 
	inputencoding=utf8/latin1, 
	morekeywords={u\_short, u\_char, u\_long, in\_addr}
	}

\usepackage{verbatim}
\usepackage{moreverb}

\usepackage{longtable}
\usepackage[nottoc,numbib]{tocbibind}

\usepackage{html}   %  *always* load this for LaTeX2HTML
\begin{htmlonly}
\providecommand{\lstinputlisting}[2][]{\verbatiminput{#2}}
\end{htmlonly}

\bibliographystyle{unsrt}

\begin{document}

% Формируем титульную страницу
\title{Программа курсов по ОС Linux}
\author{Epam Low Level Linux Department}
%\date{2011}
\maketitle

\renewcommand{\contentsname}{Оглавление}

\tableofcontents
%\setcounter{tocdepth}{2}
%\newpage

\section{Цель}
%\addcontentsline{toc}{chapter}{Цель}

Целью курса является обучение студентов базовыми знаниями и навыками работы в ОС Linux.

Курс рассчитан на студентов, которые уже знакомы с программированием, как таковым.
Предпочтительным языком программирования является C.

\newpage
\section{Программа}

\subsection{Введение в GNU/Linux}

\begin{longtable}{|p{0.2\linewidth}|p{0.5\linewidth}|p{0.07\textwidth}|p{0.1\textwidth}|}
        \hline
		\textbf{Тема} & \textbf{Содержание} & \textbf{Теория} & \textbf{Практика} \\ \hline
		\endfirsthead
		\multicolumn{4}{c}%
		{\tablename\ \thetable\ -- \textit{Продолжение}} \\
		\hline
		\textbf{Тема} & \textbf{Содержание} & \textbf{Теория} & \textbf{Практика} \\ \hline
		\endhead
		\hline \multicolumn{3}{r}{\textit{Продолжение на следующей странице}} \\
		\endfoot
		\hline
		\endlastfoot

        Введение & Обзор свободных лицензий. Принципы проектирования переносимых программ & 1 & 0 \\ \hline
        ОС Linux & Дистрибутивы. Процесс загрузки ОС Linux. Управление загрузкой & 1.5 & 0.5 \\ \hline
        Командная строка & Понятие shell, console. Окружение. Потоки ввода/вывода. Базовые утилиты. Sed. & 2.5 & 2.5 \\ \hline
        Управление ПО & ``DLL Hell''. Пакетные менеджеры. Установка, обновление и удаление пакетов. & 1 & 1 \\ \hline
        Пользовательская модель & Управление пользователями, группами & 1 & 1 \\ \hline
        Файловая структура & FHS. Файловые объекты. Права доступа. Утилиты. & 1 & 1 \\ \hline
        Дисковая подсистема & Блочные устройства. Файловые системы. RAID. LVM. & 1 & 1 \\ \hline
        Сетевое администрирование & Управление интерфейсами. Iptables. Удаленный доступ. & 1 & 1 \\ \hline
	\textbf{Итого:} & 18 ч.  & 10 & 8 \\ \hline

\end{longtable}



%%\newpage

\subsection{Командная оболочка и язык сценариев Bash}
\begin{longtable}{|p{0.2\linewidth}|p{0.5\linewidth}|p{0.07\textwidth}|p{0.1\textwidth}|}
        \hline
		\textbf{Тема} & \textbf{Содержание} & \textbf{Теория} & \textbf{Практика} \\ \hline
		\endfirsthead
		\multicolumn{4}{c}%
		{\tablename\ \thetable\ -- \textit{Продолжение}} \\
		\hline
		\textbf{Тема} & \textbf{Содержание} & \textbf{Теория} & \textbf{Практика} \\ \hline
		\endhead
		\hline \multicolumn{3}{r}{\textit{Продолжение на следующей странице}} \\
		\endfoot
		\hline
		\endlastfoot



        Синтаксис & Переменные. Операторы. Экранирование. Коды возврата. Тесты и сравнения.	Ветвления. Циклы. Арифметические операции. Массивы & 2 & 3 \\ \hline
        Регулярные выражения & Использование. Awk. & 1 & 1 \\ \hline
        Сценарии  & Функции. Встроенные и внешние команды. Подстановка результатов выполнения команд. Фильтры. Перенаправление ввода/вывода.	exec & 2 & 3 \\ \hline
        \textbf{Итого:} & 12         & 5 & 7 \\ \hline

\end{longtable}

%%\newpage
\subsection{Инструменты разработчика}
\begin{longtable}{|p{0.2\linewidth}|p{0.5\linewidth}|p{0.07\textwidth}|p{0.1\textwidth}|}
        \hline
		\textbf{Тема} & \textbf{Содержание} & \textbf{Теория} & \textbf{Практика} \\ \hline
		\endfirsthead
		\multicolumn{4}{c}%
		{\tablename\ \thetable\ -- \textit{Продолжение}} \\
		\hline
		\textbf{Тема} & \textbf{Содержание} & \textbf{Теория} & \textbf{Практика} \\ \hline
		\endhead
		\hline \multicolumn{3}{r}{\textit{Продолжение на следующей странице}} \\
		\endfoot
		\hline
		\endlastfoot


        GNU Toolchain  & Компилятор. Линкер. Статические и динамические приложения и библиотеки. & 2 & 2 \\ \hline
        Управление сборкой проекта & Make. Введение в синтаксис makefile. Autotools. & 2 & 2 \\ \hline
        Базовый инструментарий & Работа с исходным кодом. Анализ исполняемого файла. Утилиты для анализа. & 2 & 2 \\ \hline
        Отладка & Трассировка. Backtrace. GDB. Valgrind. Дампы. & 3 & 3 \\ \hline
        Профайлинг & Определение ``узких мест''. Утилиты для определения ``узких мест''. & 2 & 2 \\ \hline
        Покрытие исходного кода & gcov & 1 & 1 \\ \hline
        Системы документирования исходного кода & Doxygen & 1 & 1 \\ \hline
        Совместная разработка & Централизованные и распределенные системы контроля версий. SVN. GIT. История изменений. Бранчи, слияния, тэги. & 1 & 1 \\ \hline
        Установка ПО & Установка из исходников. Введение в пакетирование RPM. Run-time и build-time окружения. & 1 & 1 \\ \hline
        \textbf{Итого:} & 30         & 15 & 15 \\ \hline


\end{longtable}

\end{document}

