% xetex compatible variant that support TTF fonts according to company rules
\documentclass[ignorenonframetext, professionalfonts, hyperref={unicode}]{beamer}

\usetheme{Epam}

\usepackage{fontspec}
\setsansfont{SourceSansPro-Regular}
%\setbeamerfont{frametitle}{family=\fontspec{Oswald}}
\setbeamerfont{frametitle}{family=\fontspec{Oswald}}
\setbeamerfont{block title}{family=\fontspec{Oswald}}

%\setmainfont{Times New Roman}
\defaultfontfeatures{Mapping=tex-text}
\defaultfontfeatures{Ligatures=TeX}

%\setsansfont{Arial}
%\setromanfont{Trebuchet MS}

\usepackage{cmap}
\usepackage{graphicx}

\usepackage{textcomp}

\usepackage{beamerthemesplit}

\usepackage{ulem}

\usepackage{verbatim}
\usepackage{import}

\usepackage{listings}
\lstloadlanguages{bash}

\lstset{escapechar=`,
	captionpos=b,
	extendedchars=false,
	language=sh,
%	frame=single,
	tabsize=2, 
	columns=fullflexible, 
%	basicstyle=\scriptsize,
	keywordstyle=\color{blue}, 
	commentstyle=\itshape\color{brown},
%	identifierstyle=\ttfamily, 
	stringstyle=\mdseries\color{green}, 
	showstringspaces=false, 
	numbers=left, 
	numberstyle=\footnotesize, 
	breaklines=true, 
	inputencoding=utf8,
	keepspaces=true,
	morekeywords={u\_short, u\_char, u\_long, in\_addr}
	}

\definecolor{darkgreen}{cmyk}{0.7, 0, 1, 0.5}

\lstdefinelanguage{diff}
{
    morekeywords={+, -},
    sensitive=false,
    morecomment=[l]{//},
    morecomment=[s]{/*}{*/},
    morecomment=[l][\color{darkgreen}]{+},
    morecomment=[l][\color{red}]{-},
    morestring=[b]",
}

\author[Epam]{{\bf Epam}\\Low Level Programming Department}

%\institution[EPAM]{EPAM}
%\logo{\includegraphics[width=1cm]{logo.png}}

\graphicspath{{../../slides/cmdline/clipart/}{../../slides/bash/clipart/}}

\bibliographystyle{unsrt}
\setbeamertemplate{bibliography item}{\insertbiblabel}

\AtBeginSection[]{%
  \begin{frame}<beamer>
    \frametitle{}
    \tableofcontents[
        sectionstyle=show/shaded, hideallsubsections ]
  \end{frame}
  \addtocounter{framenumber}{-1}% If you don't want them to affect the slide number
}

% \regex for regular expressions
\newcommand{\regex}[1]{ %
\expandafter{$\ulcorner{\color{blue}\texttt{#1}}\lrcorner$} %
}



\title{Введение в GNU/Linux}


%%%%%%%%%%%%%%%%%%%%%%%%%%%%%%%%%%%%%%%%%%%%%%%%%
%%%%%%%%%% Begin Document  %%%%%%%%%%%%%%%%%%%%%%
%%%%%%%%%%%%%%%%%%%%%%%%%%%%%%%%%%%%%%%%%%%%%%%%%




\begin{document}

\begin{frame}
	\frametitle{}
	\titlepage
	\vspace{-0.5cm}
	\begin{center}
	%\frontpagelogo
	\end{center}
\end{frame}


%%%%%%%%%%%%%%%%%%%%%%%%%%%%%%%%%%%%%%%%%   
%%%%%%%%%% Content starts here %%%%%%%%%%
%%%%%%%%%%%%%%%%%%%%%%%%%%%%%%%%%%%%%%%%%

\section{Перенаправление ввода-вывода}
\mode<all>{

\begin{frame}{Конвееры}
%  \textbf{Цель} -- максимальная модульность: большое количество простых приложений, взаимодействующих друг с другом для решения задач
  \only<1>{
  \begin{center}
    \includegraphics[width=1.2in]{../../slides/cmdline/process}
  \end{center}
  }
  \only<2>{
    \begin{center}
      \includegraphics[width=3.6in]{../../slides/cmdline/processes}
    \end{center}
  }
  \begin{itemize}
    \item <1-> Каждое приложение открывает 3 стандартных файловых дескриптора (file descriptor) \alert{stdin (fd 0)}, \alert{stdout(fd 1)}, \alert{stderr (fd 2) }
    \item <2-> Приложения могут работать как фильтр из \alert{STDIN} в \alert{STDOUT}, можно объединять несколько приложений в конвейер
    \item <2-> Синтаксис {\tt <app1> | <app2>}
  \end{itemize}
\end{frame}
}
\mode<all>{\begin{frame}{Перенаправления в файл}

\begin{itemize}
  \item Перенаправление stdout FD=1
    \begin{itemize}
      \item С созданием нового файла

        {\tt command > file}\\
		Например {\tt cat file1 file2 > file3}
      \item С дополнением существующего

		  {\tt command >\phantom{}>  file}
    \end{itemize}
    \pause
  \item Перенаправления stdin FD=0

    {\tt command < file}
    \pause
  \item Перенаправления stderr FD=2

    {\tt command1 2>\&1 | command2}

   {\tt command 1>file 2>\&1}

   {\tt command 2>file 1>\&2}
\end{itemize}

\end{frame}
}
\mode<all>{\input{../../slides/cmdline/io-redirection-here.tex}}

\section{Полезные команды}

\mode<all>{\begin{frame}{Дополнительный набор команд}
  \begin{itemize}
    \item {\tt cat} - Вывод файла в stdout, соединение нескольких файлов в stdout
    \item {\tt wc} - подсчет статистики символов в файле или в stdin 
    \item {\tt sort} - сортировка строк файла
    \item {\tt uniq} - объединение одинаковых строк в одну
    \item {\tt tr} - замена набора символов
    \item {\tt less} - программа-пейджер
    \item {\tt grep} - поиск строк, соответствующих регулярному выражению
    \item {\tt cut} - выделение полей из строк stdin
    \item {\tt awk} - небольшой язык программирования (также полезен для выделения полей)
  \end{itemize}
\end{frame}

\begin{frame}[fragile]{Некоторые примеры использования}
\begin{lstlisting}[language=bash]
cat /proc/1/environ | tr '\0' '\n' | less
ls  | wc -l # подсчет числа файлов
man uniq | tr  '[:space:]' '\n' | sort | uniq -c | sort -n | less # подсчет количества слов в тексте man uniq
history | wc -l # подсчет ранее введенных команд
cat /etc/udev/rules.d/* | wc -l
ls -s *.jpg | awk 'BEGIN{s=0};/^[ ]*[0-9]/{s+=`\$1`};END{print s}' 
\end{lstlisting}
  \pause
  \begin{block}{Упражнение}
    Посчитать статистику использования команд в history
  \end{block}
\end{frame}

\begin{frame}{Дополнительный набор команд для работы с текстом}
	\begin{itemize}
	  \item {\tt head} -- вывести первые строки
	  \item {\tt tail} -- вывести последние строки
		\begin{itemize}
			\item {\tt -f} -- отслеживать добавление данных в файл 
		\end{itemize}
	  \item {\tt tee} -- копировать стандартный вывод в файл
	  \item {\tt grep} -- печать текста, соответствующего шаблону
		\begin{itemize}
			\item {\tt -i}	
			\item {\tt -v}
			\item {\tt -o}
		\end{itemize}
	\end{itemize}
\end{frame}

}

\subsection{Архиваторы}
\mode<all>{\begin{frame}[fragile]{Архивация}
	\begin{block}{Архивация: tar}
		\begin{itemize}
			\item {\tt -c} -- создать архив
			\item {\tt -x} -- извлечь из архива
				\begin{itemize}
					\item {\tt -C} -- перейти в директорию
					\item {\tt -{}-strip-components=N} -- пропустить N уровней
				\end{itemize}
			\item {\tt -f} -- запись в файл
		\end{itemize}
	\end{block}

	\begin{block}{Сжатие: gzip, bzip, xz}
		\begin{itemize}
			\item {\tt -[1-9]} -- изменить уровень сжатия
			\item {\tt -d} -- распаковать
			\item {\tt -c} -- вывод на консоль
		\end{itemize}
		\begin{verbatim}
dd if=/dev/sda bs=1M count=1 | gzip -c > backup.gz
    \end{verbatim}
	\end{block}

\end{frame}

\begin{frame}[fragile]{Архивация: примеры}
	Создать сжатый архив:
	\begin{verbatim}
tar -czf archive.tar.gz *
        \end{verbatim}
	\pause
	Распаковать сжатый архив в директорию {\tt /tmp}:
	\begin{verbatim}
tar -C /tmp/ -xzf archive.tar.gz
        \end{verbatim}
	\pause
	Создать копию текущей директории в директории {\tt /tmp/copy/}:
	\begin{verbatim}
tar -c * | tar -C /tmp/copy -x
tar -cf - * | tar -C /tmp -xf -
        \end{verbatim}
	\pause
	Создать копию текущей директории на другом хосте:
	\begin{verbatim}
HostDest: netcat -l 2222 | gzip -dc | tar -C /tmp/copy/ -x
HostSrc:  tar -c * | gzip -9 | netcat HostDest 2222
        \end{verbatim}
\end{frame}
}

\subsection{find и xargs}
\mode<all>{\begin{frame}[fragile]{Поиск файлов командой find}
    \alert{find} ищет файлы в заданной директории и производит над ним заданную операцию.
	\begin{block}{Часто используемые параметры поиска}
		\begin{itemize}
			\item {\tt -name}, {\tt -iname} -- имя файлового объекта, включая метасимволы 
			\item {\tt -type} -- тип файлового объекта
			\item {\tt -size} -- размер [cwbkMG]
			\item {\tt -perm} -- права доступа
			\item {\tt -user} -- владелец
			\item {\tt ...} -- другие опции man find 
		\end{itemize}
	\end{block}
\end{frame}

\begin{frame}[fragile]{Файлы найдены}
	\begin{block}{Действия над результом поиска}
		\begin{itemize}
			\item {\tt -print} -- вывод на stdout (по умолчанию)
			\item {\tt -printf} -- форматированный вывод
			\item {\tt -exec} -- выполнить команду
			\item {\tt -ls} -- замена -exec ls -l \{\} ;
			\item {\tt -delete} -- удалить файл
		\end{itemize}
	\end{block}
\end{frame}

\begin{frame}[fragile]{Примеры использования команды find}
            В текущей директории найти все файлы *.o и вывести на экран 
            \begin{verbatim} find . -name '*.o' -print \end{verbatim}
            \begin{verbatim} find -name '*.o' \end{verbatim}
            Поск по типу и владельцу файла.
            \begin{verbatim} find -type d -user altlinux \end{verbatim}
            Составная команда, множество условий
            \begin{verbatim} find /root \( -name '*.pyc' -o -name '*.py' \) \
-type f -user root -size +300k -size -1024k \
-exec ls -l \{\} \; \end{verbatim}
 Дополнительно: позволяет преодолеть лимит на кол-во аргументов в командной строке. 
 \textquotedblleft Arguments too long.\textquotedblright 
\end{frame}

\begin{frame}[fragile]{xargs}
			Утилита для создания и запуска команд из стандартного потока ввода:
		\begin{verbatim}
xargs [options] command [command options]
                \end{verbatim}
		\begin{itemize}
			\item {\tt -d} -- разделитель
			\item {\tt -0} -- null-terminated строки
			\item {\tt -I text} -- подстановка
			\item {\tt -n N} -- максимальное количество аргументов
			\item {\tt -P N} -- максимальное количество процессов
		\end{itemize}

%Для работы с разделителями в имени файла: пробелы, tab, символ новой строки. 
Использовать -print0 в команде find для замены на ASCII NUL в имени файла.
\end{frame}

\begin{frame}[fragile]{xargs}
	\begin{block}{Примеры}
		\begin{verbatim}
file /bin/*  | grep shell | cut -f 1 -d ':' | xargs wc -l 
# calculate number of strings in all shell scripts
                \end{verbatim}
		\begin{verbatim}
find /etc -type f -size -100k | \
 xargs tar -czf /tmp/archive-100k.tar.gz
                \end{verbatim}
		\begin{verbatim}
find /etc -type f | xargs -I {} echo "Найден {} файл"
                \end{verbatim}

		\begin{verbatim}
find . -type f -name "*.mp3" -print0 | \
 xargs -0 -n 1 -P 0 -I mp3 avconv -i mp3 mp3.ogg
                \end{verbatim}
	
	\end{block}
\end{frame}
}

\subsection{Редакторы}
\mode<all>{\begin{frame}{Текстовые редакторы}
	\begin{itemize}
		\item Интерактивные
			\begin{itemize}
				\item vi
				\item vim
				\item emacs
			\end{itemize}
		\item Поточные
			\begin{itemize}
				\item {\tt ed}
				\item {\tt sed}
				\item {\tt awk}
			\end{itemize}
	\end{itemize}
\end{frame}

%%\begin{frame}[fragile]{Метасимволы}
%	\begin{block}{grep, sed, awk}
%	\end{block}
%	\begin{itemize}
%		\item {\tt .} -- любой символ за исключением пустой строки
%		\item {\tt *} -- любоe количество символов, которые стоят перед {\tt *}
%		\item {\tt \^{}} -- начало строки
%		\item {\tt \$} -- конец строки
%		\item {\tt [...]} -- любой символ из заключенных в скобки
%	\end{itemize}
%\end{frame}

\begin{frame}[fragile]{sed}
	\begin{block}{Сценарии}
		{\tt [ addr [ ,  addr ] ] cmd [ args ]}
	\end{block}

	\tiny
	\begin{block}{Команды}
		\begin{itemize}
		  \item {\tt d} -- удалить строку
			  \begin{verbatim} who | sed -e '2,4 d' \end{verbatim}
			  \begin{verbatim} who | sed -e '/pts/ d' \end{verbatim}
		  \item {\tt s} -- замена по регулярному выражению
			  \begin{verbatim} who | sed -e "s/USER/user/g" \end{verbatim}
		  \item {\tt a, i} -- добавить строку после (перед) текущей
			  \begin{verbatim} who | sed -e 'a Text' \end{verbatim}
		\end{itemize}
	\end{block}
%	\pause
%	\begin{block}{Задача}
%		С помощью {\tt find} найти все вложенные директории в {\tt /etc} и 
%		''переделать'' их в windows-style
%	\end{block}
\end{frame}
}

\end{document}
