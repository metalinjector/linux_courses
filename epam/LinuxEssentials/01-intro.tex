% xetex compatible variant that support TTF fonts according to company rules
\documentclass[ignorenonframetext, professionalfonts, hyperref={unicode}]{beamer}

\usetheme{Epam}

\usepackage{fontspec}
\setsansfont{SourceSansPro-Regular}
%\setbeamerfont{frametitle}{family=\fontspec{Oswald}}
\setbeamerfont{frametitle}{family=\fontspec{Oswald}}
\setbeamerfont{block title}{family=\fontspec{Oswald}}

%\setmainfont{Times New Roman}
\defaultfontfeatures{Mapping=tex-text}
\defaultfontfeatures{Ligatures=TeX}

%\setsansfont{Arial}
%\setromanfont{Trebuchet MS}

\usepackage{cmap}
\usepackage{graphicx}

\usepackage{textcomp}

\usepackage{beamerthemesplit}

\usepackage{ulem}

\usepackage{verbatim}
\usepackage{import}

\usepackage{listings}
\lstloadlanguages{bash}

\lstset{escapechar=`,
	captionpos=b,
	extendedchars=false,
	language=sh,
%	frame=single,
	tabsize=2, 
	columns=fullflexible, 
%	basicstyle=\scriptsize,
	keywordstyle=\color{blue}, 
	commentstyle=\itshape\color{brown},
%	identifierstyle=\ttfamily, 
	stringstyle=\mdseries\color{green}, 
	showstringspaces=false, 
	numbers=left, 
	numberstyle=\footnotesize, 
	breaklines=true, 
	inputencoding=utf8,
	keepspaces=true,
	morekeywords={u\_short, u\_char, u\_long, in\_addr}
	}

\definecolor{darkgreen}{cmyk}{0.7, 0, 1, 0.5}

\lstdefinelanguage{diff}
{
    morekeywords={+, -},
    sensitive=false,
    morecomment=[l]{//},
    morecomment=[s]{/*}{*/},
    morecomment=[l][\color{darkgreen}]{+},
    morecomment=[l][\color{red}]{-},
    morestring=[b]",
}

\author[Epam]{{\bf Epam}\\Low Level Programming Department}

%\institution[EPAM]{EPAM}
%\logo{\includegraphics[width=1cm]{logo.png}}

\graphicspath{{../../slides/cmdline/clipart/}{../../slides/bash/clipart/}}

\bibliographystyle{unsrt}
\setbeamertemplate{bibliography item}{\insertbiblabel}

\AtBeginSection[]{%
  \begin{frame}<beamer>
    \frametitle{}
    \tableofcontents[
        sectionstyle=show/shaded, hideallsubsections ]
  \end{frame}
  \addtocounter{framenumber}{-1}% If you don't want them to affect the slide number
}

% \regex for regular expressions
\newcommand{\regex}[1]{ %
\expandafter{$\ulcorner{\color{blue}\texttt{#1}}\lrcorner$} %
}



\title{Введение в GNU/Linux}


%%%%%%%%%%%%%%%%%%%%%%%%%%%%%%%%%%%%%%%%%%%%%%%%%
%%%%%%%%%% Begin Document  %%%%%%%%%%%%%%%%%%%%%%
%%%%%%%%%%%%%%%%%%%%%%%%%%%%%%%%%%%%%%%%%%%%%%%%%




\begin{document}

\begin{frame}
	\frametitle{}
	\titlepage
	\vspace{-0.5cm}
	\begin{center}
	%\frontpagelogo
	\end{center}
\end{frame}


\begin{frame}
	\tableofcontents
	[hideallsubsections]
\end{frame}

\section{О курсах}

\mode<all>{\input{../../epam/slides/intro-epam}}

\section*{GNU/Linux}

\begin{frame}{Основы ОС Linux}

	\begin{block}{Вопрос}
	Почему Linux является самой популярной
	свободной операционной системой?
	\end{block}

	\pause

	\begin{block}{Ответ}
	\begin{itemize}
		\item \textcopyleft -- Copyleft
		\item ``Философия'' Unix
		\item Открытые стандарты
	\end{itemize}
	\end{block}

\end{frame}


%%%%%%%%%%%%%%%%%%%%%%%%%%%%%%%%%%%%%%%%%   
%%%%%%%%%% Content starts here %%%%%%%%%%
%%%%%%%%%%%%%%%%%%%%%%%%%%%%%%%%%%%%%%%%%

\section[Принципы]{Базовые принципы ОС Linux}

\subsection{GNU/Linux}

\mode<all>{\input{../../slides/intro/vocabulary}}

\subsection{Лицензии}

\mode<all>{\begin{frame}{Авторское право и лицензии}

	\begin{block}{Авторское право}
		 Возникает по факту создания ПО 

		\begin{itemize}
			\item Неимущественные права
			\item Имущественные права
		\end{itemize}
	\end{block}

	\pause

	\begin{block}{Лицензии}
		Лицензия -- средство передать какие-либо права на продукт либо его часть.

		Необходима для защиты авторских прав. 
		Средство для возможности законно пресечь несанкционирование копирование,  использование или распространение ПО. 
	\end{block}
\end{frame}


\begin{frame}{Лицензии: открытые и свободные}
	\begin{block}{ Р.Столлман: 4 свободы}
		\begin{itemize}
			\item Свобода 0: Свобода запускать программу в любых целях.
			\item Свобода 1: Свобода изучения работы программы и адаптация её к вашим нуждам. 
				Доступ к исходным текстам является необходимым условием.
			\item Свобода 2: Свобода распространять копии,  так что вы можете помочь вашему товарищу.
			\item Свобода 3: Свобода улучшать программу и публиковать ваши улучшения,
				так что всё общество выиграет от этого.
				Доступ к исходным текстам является необходимым условием.
		\end{itemize}
	\end{block}
\end{frame}


\begin{frame}{Лицензии: permissive}
	\begin{columns}
	\column{0.3\textwidth}
		\center\includegraphics[width=2cm,natwidth=144,natheight=144]{../../slides/intro/three-arrows@2x.png}

	\column{0.6\textwidth}

	\begin{itemize}
		\item BSD
		\item MIT
		\item Apache
	\end{itemize}
	\end{columns}

	\begin{block}{I want it simple and permissive.}
		\begin{itemize}
			\item практически не ограничивают свободу действий пользователей ПО и разработчиков, работающих с исходным кодом.
			\item По своему духу, распространение работы под пермиссивной лицензией схоже с помещением работы в общественное
				достояние, но не требует отказа от авторского права.
		\end{itemize}
	\end{block}

\end{frame}


\begin{frame}{\textcopyleft -- Copyleft}

	\begin{columns}
	\column{0.3\textwidth}
		\center\includegraphics[width=2cm,natwidth=144,natheight=138]{../../slides/intro/circular@2x.png}

	\column{0.6\textwidth}

	\begin{itemize}
		\item GPL
		\item LGPL
		\item AGPL
	\end{itemize}
	\end{columns}


	\begin{block}{I care about sharing improvements.}
	
	Авторское лево -- концепция и практика использования законов авторского права для обеспечения 
	невозможности ограничить любому человеку право использовать,  изменять и распространять как 
	исходное произведение,  так и произведения,  производные от него.
	\end{block}


	При копилефте все производные произведения должны распространяться под той же лицензией,
	что и оригинальное произведение.

\end{frame}



}

%%\subsection{Принципы проектирования переносимых программ}

%%\mode<all>{\input{../../slides/intro/unixway}}

\section{Дистрибутивы ОС Linux}

\mode<all>{\begin{frame}{Дистрибутив ОС GNU/Linux}
	\begin{block}{ Определение}
		\only<1>{\center{\bf{?}}}
		\pause
		\only<2->{Набор программного обеспечения на базе ядра Linux, распространяющийся как единое целое.}
	\end{block}
\end{frame}


\begin{frame}{Задачи дистрибутива}
	\begin{itemize}
		\item Предоставление комплекта ПО (ядро + утилиты)
		\item Средства установки и настройки
		\item Средства обновления
	\end{itemize}
\end{frame}

\begin{frame}{Различия между дистрибутивами}

	\only<1>{\Large\center{\bf{?}}}
	\pause
	\only<2->{\Large\center{\bf{Цели!!!}}}

	\bigskip
	\normalsize

	\pause

	\begin{itemize}
		\begin{columns}
		\column{0.4\textwidth}
			\item Инсталлятор
			\item Первичные настройки
			\item Средства управления
			\item Набор ПО
		\column{0.4\textwidth}
			\item Менеджер пакетов
			\item Формат распространения ПО
			\item Пути к файлам
			\item Система сборки ПО
		\end{columns}
	\end{itemize}
\end{frame}

\begin{frame}{Дистрибутивы}
	\begin{itemize}
		\begin{columns}
		\column{0.3\textwidth}
			\item RedHat
			\item Fedora Core
			\item CentOS
			\item Scientific Linux
			\item Oracle Unbreakable Linux
		\column{0.3\textwidth}
			\item Slackware 
			\item Gentoo
			\item Arch
			\item OpenSUSE
			\item ALT Linux 
		\column{0.3\textwidth}
			\item Debian
			\item Ubuntu
			\item Mint
			\item Knoppix
			\item BackTrack
		\end{columns}
	\end{itemize}
\end{frame}
}

\section{Процесс загрузки ОС Linux}

\subsection{Этапы загрузки}

\mode<all>{\begin{frame}{Процесс загрузки GNU/Linux}
	\scriptsize
	\begin{enumerate}
		\item BIOS
		\item Master Boot Record (MBR)
			\pause
		\item Загрузка загрузчика 
		\begin{itemize}
		\footnotesize
			\item Stage 1 -- Первичный загрузчик
			\item Stage 1,5 -- Загрузка ядра загрузчика и драйвера ФС
			\item Stage 2 -- загрузчик читает конфигурацию, загружает ядра и образ initrd (initial-RAM disk) в память
                        \item Передает управление ядру
		\end{itemize}

		\item Запуск программы инициализации в initrd, загрузка драйверов файловых систем (LVM, RAID, NFS)
			\pause
		\item Нахождение и монтирование корневого раздела
			\pause
		\item Запуск программы init
		\begin{itemize}
		\footnotesize
			\item Монтирование оставшихся разделов ФС
			\item Запуск демонов для заданного уровня загрузки (runlevel)
			\item Выдает приглашение пользователю. 
		\end{itemize}

	\end{enumerate}
\end{frame}
}

\subsection{Ядро Linux}

\mode<all>{\begin{frame}{Задачи ядра Linux}
	\begin{itemize}
		\item Инициализация системы
		\item Управление процессами и потоками
		\item Управление памятью
		\item Управление файлами
		\item IPC (Inter Process Communication)
		\item Разграничение доступа
		\item Сетевые возможности
		\item Интерфейс доступа к возможностям ядра
	\end{itemize}
	%Выполнить команду dmesg и найти строки инициализации памяти, CPU, дисковой подсистемы, CPU.
\end{frame}


\begin{frame}{Ядро}

	Ядро ОС Linux является модульным. 

	\begin{block}{Модули}
		\begin{itemize}
			\item В виде отдельных файлов
			\item "Вкомпилированные" в ядро
		\end{itemize}
	\end{block}

	\bigskip
	%Список загруженных модулей: {\tt cat /proc/modules } либо {\tt lsmod}
\end{frame}


\begin{frame}{Параметры ядра}
	
	Полный список: {\tt Documentation/kernel-parameters.txt}

	\begin{block}{Некоторые часто применяемые параметры}
		\begin{itemize}
			\begin{columns}
			\column{0.3\textwidth}
				\item console=ttyS0,9600
				\item debug
				\item init=/sbin/init
				\item loglevel=[0-7]
				\item maxcpus=[num]
			\column{0.3\textwidth}
				\item mem=nn[KMG]
				\item noacpi
				\item noapic
				\item panic=nn (sec)
				\item resume=/dev/sda2
			\column{0.3\textwidth}
				\item ro
				\item rw
				\item root=/dev/sda1
				\item rootdelay=nn (sec)
				\item rootwait
				\item vga=<num>|ask
			\end{columns}
		\end{itemize}
	\end{block}

	Параметры переданные ядру во время загрузки: {\tt /proc/cmdline} \\

        Программы могут использовать /cmdline, например установщик ОС Anaconda \\

	Модулям можно передавать параметры используя синтаксис: {\tt module.param=value} \\
\end{frame}
}

\subsection{Userspace}

\mode<all>{\input{../../slides/intro/initrd}}

\mode<all>{\begin{frame}{init}
	Менеджер управления работой системой и сервисами.
	
	\bigskip

	\center{\large PID = 1}

	\bigskip

	\begin{block}{Наиболее известные}
		\begin{itemize}
			\item SysVInit
			\item systemd
			\item upstart
		\end{itemize}
	\end{block}
\end{frame}

\begin{frame}{SysVInit}
	\begin{block}{Управление}
		\begin{itemize}
			\item kernel boot parameters: <N> -- runlevel
			\item утилита {\tt runlevel}
			\item утилита {\tt init}
		\end{itemize}
	\end{block}

	\scriptsize
	\begin{block}{Runlevel}
		\begin{table}
			\begin{tabular}{| c | l | }
			\hline
			Runlevel & Описание\\
			\hline
			0	& Выключить систему \\
			1,s,single & Однопользовательский режим \\
			2	& Многопользовательский режим без графики. Без сетевых сервисов.\\
			3	& Многопользовательский режим без графики. Полноценная сеть. \\
			4	& Определяется на хосте\\
			5	& Многопользовательский режим с графикой.\\
			6	& Перезагрузка\\
			emergency & Аварийная оболочка \\
			\hline
			\end{tabular}
		\end{table}
	\end{block}
\end{frame}

\begin{frame}{SysVInit: сервисы}
	\begin{block}{Управление}
		\begin{itemize}
			\item утилита {\tt service}
			\item утилита {\tt chkconfig}
		\end{itemize}
	\end{block}

	\begin{block}{Сервисы}
		\begin{itemize}
			\item {\tt /etc/rc.d/init.d}
			\item {\tt /etc/rc.d/rc.N}\footnote{N=runlevel}
		\end{itemize}
	\end{block}
\end{frame}

\begin{frame}{systemd}
	\begin{block}{Управление}
		\begin{itemize}
			\item kernel boot parameters\\
				{\tt systemd.unit=rescue.target} \\
			\item утилита {\tt systemctl} \\
				{\tt systemctl isolate multi-user.target} \\
				{\tt systemctl set-default single.target}
		\end{itemize}
	\end{block}

	\begin{block}{targets}
		\tiny
		\begin{table}
			\begin{tabular}{| c | l | l | }
			\hline
			Runlevel & Описание\\
			\hline
			0	& poweroff.target & Выключить систему \\
			1,s,single & rescue.target  & Однопользовательский режим \\
			2	& multi-user.target & Многопользовательский режим без графики. Без сетевых сервисов.\\
			3	& multi-user.target & Многопользовательский режим без графики. Полноценная сеть. \\
			4	& multi-user.target & Определяется на хосте\\
			5	& graphical.target & Многопользовательский режим с графикой.\\
			6	& reboot.target & Перезагрузка\\
			emergency & emergency.target & Аварийная оболочка \\
			\hline
			\end{tabular}
		\end{table}
	\end{block}
\end{frame}

\begin{frame}{systemd: сервисы}
	\begin{block}{Управление}
		\begin{itemize}
			\item утилита {\tt systemctl}
		\end{itemize}
	\end{block}

	\begin{block}{Сервисы}
		\begin{itemize}
			\item {\tt /lib/systemd/system/}
			\item {\tt /etc/systemd/system/}
		\end{itemize}
	\end{block}
\end{frame}
}

\subsection{Практика}

\mode<all>{\begin{frame}{Практическое задание}
	\begin{enumerate}
		\item Загрузить ОС по умолчанию
		\item Посмотреть используемые параметры ядра 
		\item Посмотреть список загруженных модулей
			\pause
		\item Переопределить init на sh
		\item SysRq. {\bf R}eboot {\bf E}ven {\bf I}f {\bf S}ystem {\bf U}tterly {\bf B}roken
			\pause
		\item Загрузить ядро с "урезанным" количеством памяти
		\item Отключить 1 или несколько процессоров
			\pause
		\item Посмотреть текущий runlevel
		\item Посмотреть список сервисов
	\end{enumerate}
\end{frame}
}

\end{document}
