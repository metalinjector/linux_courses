\begin{frame}{Цель курса}
	\begin{center}
		\large
		Обеспечить быстрый старт в среде GNU/Linux для опытных пользователей
	\end{center} 
    \pause
         \center\includegraphics[scale=0.6]{linux_story} 

    Система сложная, поэтому время потраченное на самостоятельно изучение системы исчисляется годами.
\end{frame}

\begin{frame}{Читает курс}

    \large Vikentsi Lapa 
    \begin{columns}
        \column{0.6\textwidth}
            \begin{itemize}
                \item Senior Software Test Automation Engineer
                \item Work with Linux about 8 years in such areas as
                \begin{itemize}
                    \item High Perfomance Computing
                    \item Cluster file systems: GPFS, Lustre 
                \end{itemize}
                \item MLUG (Minsk Linux User Group Activist)
                \item LVEE conference participant 
            \end{itemize}
        \column{0.3\textwidth}
            \center\includegraphics[width=3cm]{myphoto}
    \end{columns}
\end{frame}

\begin{frame}{Аудитория}

Для кого предназначен этот курс:
\begin{itemize}
    \item IT Engineers, DevOps Engineers
    \item Junior Software Engineers/Software Engineers
    \item Test Engineers, Test Automation Engineers
    \item Students
\end{itemize}
    \pause
Требования к слушателям:
\begin{itemize}
    \item Опыт работы с Linux/Unix меньше 1 года или без опыта
    \pause
    \item Опыт работы с системой виртуализации
    \pause
    \item Знание сетевых технологий
    \pause
    \item Creative thinking and problem solving skills
    \pause
    \item \alert{Feedback is essential: answer to questions, do practical tasks, do homework, ask questions}
\end{itemize}
\end{frame}

\begin{frame}{Проверим обратную связь}
\begin{itemize}
    \item Есть ли опыт работы с Linux? Если есть то сколько?
    \pause
    \item Ваши ожидания. Зачем пришли на курс?
    \pause
    \item Установили дистрибутив? Какой?
    \item Если не установили. Почему?
\end{itemize}
\end{frame}

\begin{frame}[fragile]{Возраст Unix.}
 Посчитаем возраст системы Unix.
\begin{lstlisting}
    $ date --help | grep '%s'
    $ date +%s # show current date in seconds 
    $ echo $(( $(date +%s)/(60*60*24*365))) # show years
\end{lstlisting} 
\pause
Unix systems are characterized by a \alert{modular design} or ``Unix philosophy"
		\begin{columns}
		\column{0.7\textwidth}
\begin{itemize}
    \item simple tools that each perform a limited, well-defined function
    \item shell scripting and command language to combine the tools to perform complex workflows
    \item unified filesystem
\end{itemize}
		\column{0.3\textwidth}
            \includegraphics[width=4cm]{40years}
		\end{columns}
% За это время сложилась культура и философия разработки, которая оказала влияние на интерфейсы взаимодействия, например названия устройств и команд.
% Разнообразие подходов при решении задач. Среда дружественная для программиста и опытного технического пользователя.
\end{frame}


%\begin{frame}{Состав курса}
%	\begin{itemize}
%		\item Представление об архитектуре GNU/Linux дистрибутива
%			\pause
%		\item "Ежедневные" навыки работы в консоли. Конфигурация системы.
%			\pause
%		\item Язык программирования
%			\begin{itemize}
%					\item shell (Bash)
%			\end{itemize}
%			\pause
%		\item Все, чего вы не знали и боялись спросить
%	\end{itemize}
%\end{frame}
