\begin{frame}{Цель курса}
	\begin{center}
		\Huge
		Обеспечить быстрый старт в среде GNU/Linux для опытных пользователей
	\end{center}
    \pause
    Курить маны, читать логи, находить ответы самостоятельно.
\end{frame}

\begin{frame}{Читает курс}

    \large Vikentsi Lapa 
    \begin{columns}
        \column{0.6\textwidth}
            \begin{itemize}
                \item Senior Software Test Automation Engineer
                \item Work with Linux about 8 years in such areas as
                \begin{itemize}
                    \item High Perfomance Computing
                    \item Cluster file systems: GPFS, Lustre 
                \end{itemize}
                \item MLUG (Minsk Linux User Group Activist)
                \item LVEE conference participant 
            \end{itemize}
        \column{0.3\textwidth}
            \center\includegraphics[width=3cm]{myphoto}
    \end{columns}
\end{frame}

\begin{frame}{Аудитория}

Для кого предназначен этот курс:
\begin{itemize}
    \item IT Engineers, DevOps Engineers
    \item Junior Software Engineers/Software Engineers
    \item Test Engineers, Test Automation Engineers
    \item Students
\end{itemize}

Если опыт работы с Linux/Unix меньше 1 года, то подойдет для систематизации знаний.

Желательно чтобы был навык работы с системой виртуализации. Самостоятельная настройка сети.

\end{frame}

%\begin{frame}{Состав курса}
%	\begin{itemize}
%		\item Представление об архитектуре GNU/Linux дистрибутива
%			\pause
%		\item "Ежедневные" навыки работы в консоли. Конфигурация системы.
%			\pause
%		\item Язык программирования
%			\begin{itemize}
%					\item shell (Bash)
%			\end{itemize}
%			\pause
%		\item Все, чего вы не знали и боялись спросить
%	\end{itemize}
%\end{frame}
